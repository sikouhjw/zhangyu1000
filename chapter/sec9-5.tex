\subsection{级数展开与求和}

	\begin{ti}
		函数 $f(x) = \frac{1}{x}$ 展开成的 $x - 1$ 的幂级数为\htwo.
	\end{ti}

	\begin{ti}
		$\ee^{x}$ 展开成的 $x - 3$ 的幂级数为\htwo.
	\end{ti}

	\begin{ti}
		将函数 $f(x) = \frac{1}{x^{2} - 3x + 2}$ 展开成 $x$ 的幂级数,并指出其收敛区间.
	\end{ti}

	\begin{ti}
		将 $y = \sin x$ 展开为 $x - \frac{\uppi}{4}$ 的幂级数.
	\end{ti}

	\begin{ti}
		将 $f(x) = \frac{1}{x^{2}}$ 展开为 $x + 1$ 的幂级数.
	\end{ti}

	\begin{ti}
		设 $f(x) = \frac{1}{1 - 2x - x^{2}}$.
		\begin{enumerate}
			\item 将 $f(x)$ 展开为 $x$ 的幂级数;
			\item 分别判断级数 $\sum_{n=0}^{\infty} \frac{n!}{f^{(n)}(0)}, \sum_{n=0}^{\infty} \frac{f^{(n)}(0)}{n!}$ 的敛散性.
		\end{enumerate}
	\end{ti}

	\begin{ti}
		将函数 $f(x) = \ln \bigl| \frac{x}{x - 3} \bigr|$ 展开成 $x - 2$ 的幂级数,并求出其收敛范围.
	\end{ti}

	\begin{ti}
		\begin{enumerate}
			\item 证明 $\sum_{n=1}^{\infty} (-1)^{n-1} \frac{1}{3n - 2} = \int_{0}^{1} \frac{1}{1 + x^{3}} \dd{x}$;
			\item 求 $\sum_{n=1}^{\infty} (-1)^{n-1} \frac{1}{3n - 2}$.
		\end{enumerate}
	\end{ti}

	\begin{ti}
		\begin{enumerate}
			\item 设 $f(x)$ 为任意阶可导函数,且
			\[
				f(x) = \sum_{n=0}^{\infty} a_{n} x^{n},
			\]
			若 $f(x)$ 为奇函数,证明
			\[
				f(x) = \sum_{n=1}^{\infty} a_{2n-1} x^{2n-1};
			\]
			\item 将函数 $f(x) = \int_{0}^{x} \ee^{x^{2} - t^{2}} \dd{t}$ 展开为 $x$ 的幂级数.
		\end{enumerate}
	\end{ti}

	\begin{ti}
		设函数 $f(x) = \frac{7 + 2x}{2 - x - x^{2}}$,当 $-1 < x < 1$ 时,其幂级数展开式为 $f(x) = \sum_{n=0}^{\infty} a_{n} x^{n}$.
		\begin{enumerate}
			\item 求 $a_{n} (n = 0,1,2,\cdots)$;
			\item 求级数 $\sum_{n=0}^{\infty} \frac{a_{n+1} - a_{n}}{(a_{n} - 2) (a_{n+1} - 2)}$ 的和.
		\end{enumerate}
	\end{ti}

	\begin{ti}
		级数 $\sum_{n=0}^{\infty} \frac{(\ln 3)^{n}}{2^{n}}$ 的和为\htwo.
	\end{ti}

	\begin{ti}
		当 $|x| < 1$ 时,级数 $\sum_{n=1}^{\infty} \frac{1}{n} x^{n}$ 的和函数是\kuo.

		\twoch{$\ln(1 - x)$}{$\ln \frac{1}{1 - x}$}{$\ln(x - 1)$}{$- \ln (x - 1)$}
	\end{ti}

	\begin{ti}
		幂级数 $\frac{1}{a} + \frac{2x}{a^{2}} + \cdots + \frac{nx^{n-1}}{a^{n}} + \cdots$ 在收敛区间 $(-a,a)$ 内的和函数 $S(x)$ 为\htwo.
	\end{ti}

	\begin{ti}
		求幂级数 $\sum_{n=0}^{\infty} \frac{(-1)^{n}}{3n + 1} x^{3n}$ 的收敛域与和函数,并求 $\sum_{n=0}^{\infty} \frac{(-1)^{n}}{3n + 1}$ 的和.
	\end{ti}

	\begin{ti}
		已知 $f_{n}(x)$ 满足 $f_{n}'(x) = f_{n}(x) + x^{n-1} \ee^{x}$($n$ 为正整数),且 $f_{n}(1) = \frac{\ee}{n}$,求函数项级数 $\sum_{n=1}^{\infty} f_{n}(x)$ 的和函数.
	\end{ti}

	\begin{ti}
		\begin{enumerate}
			\item 求函数项级数 $\ee^{-x} + 2 \ee^{-2x} + \cdots + n \ee^{-nx} + \cdots$ 收敛时 $x$ 的取值范围;
			\item 当上述级数收敛时,求其和函数 $S(x)$,并求
			\[
				\int_{\ln 2}^{\ln 3} S(x) \dd{x}.
			\]
		\end{enumerate}
	\end{ti}

	\begin{ti}
		求级数 $\sum_{n=2}^{\infty} \frac{x^{n-2}}{n \cdot 3^{n}}$ 的收敛域及其和函数.
	\end{ti}

	\begin{ti}
		求数项级数 $\sum_{n=1}^{\infty} (-1)^{n} \frac{n(n + 1)}{2^{n}}$ 的和.
	\end{ti}

	\begin{ti}
		设 $x_{1} = r > 0, x_{n+1} = x_{n} + x_{n}^{3}, n = 1,2,3,\cdots$. 则数项级数 $\sum_{n=1}^{\infty} \frac{x_{n}}{1 + x_{n}^{2}} = $\htwo.
	\end{ti}

	\begin{ti}
		求级数 $\sum_{n=0}^{\infty} \frac{x^{2n}}{(2n)!}$ 的和函数.
	\end{ti}

	\begin{ti}
		求数列 $\{ a_{n} \}$ 满足 $a_{1} = a_{2} = 1$,且 $a_{n+1} = a_{n} + a_{n-1}, n = 2,3,\cdots$. 证明在 $|x| < \frac{1}{2}$ 时幂级数 $\sum_{n=1}^{\infty} a_{n} x^{n-1}$ 收敛,并求其和函数与系数 $a_{n}$.
	\end{ti}

	\begin{ti}
		$\sum_{n=1}^{\infty} (-1)^{n-1} \frac{2n^{2}}{(2n)!} \frac{1}{2^{n}}$ 的和为\htwo.
	\end{ti}

	\begin{ti}
		已知 $y(x) = 2 + \sum_{n=1}^{\infty} \frac{x^{2n}}{(2n)!}$ 在 $(-\infty,+\infty)$ 是微分方程 $y'' - y = a$ 的解.
		\begin{enumerate}
			\item 求常数 $a$;
			\item 求 $y(x)$.
		\end{enumerate}
	\end{ti}

	\begin{ti}
		设 $y = f(x)$ 由方程组 $\begin{cases}
			x = \sum_{n=1}^{\infty} \frac{(t - 1)^{n}}{n},\\
			y = \sum_{n=1}^{\infty} \frac{n t^{n-1}}{2^{n}}
		\end{cases}$ 所确定,求 $\frac{\dd{y}}{\dd{x}}\bigr|_{t = 1}$.
	\end{ti}

	\begin{ti}
		设 $A_{n}$ 是曲线 $y = x^{n}$ 与 $y = x^{n+1}(n = 1,2,\cdots)$ 所围区域的面积,记 $S_{1} = \sum_{n=1}^{\infty} A_{n}, S_{2} = \sum_{n=1}^{\infty} A_{2n-1}$,求 $S_{1}$ 与 $S_{2}$ 的值.
	\end{ti}

	\begin{ti}
		已知 $\sum_{n=1}^{\infty} \frac{1}{n^{2}} = \frac{\uppi^{2}}{6}$.
		\begin{enumerate}
			\item 设 $f(x) = \sum_{n=1}^{\infty} \frac{1}{n^{2}} x^{n}$,证明当 $0 < x < 1$ 时,$f(x) + f(1 - x) + \ln x \ln (1 - x) = \frac{\uppi^{2}}{6}$;
			\item 求 $I = \int_{0}^{1} \frac{1}{2 - x} \ln \frac{1}{x} \dd{x}$.
		\end{enumerate}
	\end{ti}