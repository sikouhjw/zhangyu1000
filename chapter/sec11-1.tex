\chapter{概率论与数理统计}
	概率论与数理统计是硕士研究生招在考试考查内容之一,主要考查考生对研究随机规律性的基本概念、基本理论和基本方法的理解,以及运用概率统计方法分析和解决实际问题的能力。在考研数学一试卷中分值为 $34$ 分,约占 \SI{22}{\percent}。

\section{事件与概率}
	\begin{titwo}
		设 $A$ 与 $B$ 是两随机事件,$P(B) = 0.6$ 且 $P(A|B) = 0.5$,则 $P \bigl( A \cup \overline{B} \bigr) = $ \kuo.

		\fourch{$0.1$}{$0.3$}{$0.5$}{$0.7$}
	\end{titwo}

	\begin{titwo}
		设 $10$ 件产品中有 $4$ 件不合格品,从中任取两件,已知所取两件产品中有一件是不合格品,则另一件也是不合格品的概率是 \kuo.

		\fourch{$\frac{1}{2}$}{$\frac{2}{3}$}{$\frac{1}{5}$}{$\frac{2}{5}$}
	\end{titwo}

	\begin{titwo}
		设 $A$, $B$ 是任意两个事件,且 $A \subset B$, $P(B) > 0$,则必有 \kuo.

		\twoch{$P(A) \leq P(A|B)$}{$P(A) < P(A|B)$}{$P(A) \geq P(A|B)$}{$P(A) > P(A|B)$}
	\end{titwo}

	\begin{titwo}
		设 $0 < P(B) < 1$, $P(A_{1}) P(A_{2}) > 0$ 且 $P(A_{1} \cup A_{2} | B) = P(A_{1}|B) + P(A_{2}|B)$,则下列等式成立的是 \kuo.

		\onech{$P \bigl( A_{1} \cup A_{2} | \overline{B} \bigr) = P \bigl( A_{1} | \overline{B} \bigr) + \bigl( A_{2} | \overline{B} \bigr)$}%
		{$P( A_{1}B \cup A_{2}B ) = P(A_{1}B) + P(A_{2}B)$}%
		{$P(A_{1} \cup A_{2}) = P(A_{1}|B) + P(A_{2}|B)$}%
		{$P(B) = P(A_{1}) P(B|A_{1}) + P(A_{2}) P(B|A_{2})$}
	\end{titwo}

	\begin{titwo}
		设事件 $A$, $B$ 满足 $A B = \varnothing$,则下列结论中一定正确的是 \kuo.

		\twoch{$\overline{A}$, $\overline{B}$ 互不相容}{$\overline{A}$, $\overline{B}$ 相容}{$P(AB) = P(A) P(B)$}{$P(A - B) = P(A)$}
	\end{titwo}

	\begin{titwo}
		以下结论,错误的是 \kuo.
		
		\onech{若 $0 < P(B) < 1$, $P(A|B) + P \bigl( \overline{A} | \overline{B} \bigr) = 1$,则 $A$, $B$ 相互独立}%
		{若 $A$, $B$ 满足 $P(B|A) = 1$,则 $P(A-B) = 0$}%
		{设 $A$, $B$ 是两个事件,则 $(A - B) \cup B = A \cup B$}%
		{若当事件 $A$, $B$ 同时发生时,事件 $C$ 必发生,则 $P(C) < P(A) + P(B) - 1$}
	\end{titwo}

	\begin{titwo}
		$A$, $B$, $C$ 为随机事件,$A$ 发生导致 $B$ 与 $C$ 最多有一个发生,则有 \kuo.

		\fourch{$A \subset BC$}{$A \supset BC$}{$A \subset \overline{BC}$}{$A \supset \overline{BC}$}
	\end{titwo}

	\begin{titwo}
		设 $P(B) > 0$, $A_{1}$, $A_{2}$ 互不相容,则下列各式中不一定正确的是 \kuo.

		\onech{$P(A_{1}A_{2} | B) = 0$}{$P(A_{1} \cup A_{2} | B) = P(A_{1} | B) + P(A_{2} | B)$}{$P \bigl( \overline{A_{1}} \overline{A_{2}} | B \bigr) = 1$}{$P \bigl( \overline{A_{1}} \cup \overline{A_{2}} | B \bigr) = 1$}
	\end{titwo}

	\begin{titwo}
		一种零件的加工由相互独立的两道工序组成,第一道工序的废品率为 $p_{1}$,第二道工序的废品 $p_{2}$,则该零件加工的成品率为 \kuo.

		\twoch{$1 - p_{1} - p_{2}$}{$1 - p_{1}p_{2}$}{$1 - p_{1} - p_{2} + p_{1} p_{2}$}{$(1 - p_{1}) + (1 - p_{2})$}
	\end{titwo}

	\begin{titwo}
		以下 $4$ 个结论:\\
		\circled{1} 教室中有 $r$ 个学生,则他们的生日都不相同的概率是 $\frac{\AA_{365}^{r}}{365^{r}}$;\\
		\circled{2} 教室中有 $4$ 个学生,则至少有两个人的生日在同一个月的概率是 $\frac{41}{96}$;\\
		\circled{3} 将 $C,C,E,E,I,N,S$ 共 $7$ 个字母随机地排成一行,恰好排成英文单词 $SCIENCE$ 的概率是 $\frac{1}{315}$;\\
		\circled{4} 袋中有编号为 $1$ 到 $10$ 的 $10$ 个球,今从袋中任取 $3$ 个球,则 $3$ 个球的最小号码为 $5$ 的概率为 $\frac{1}{12}$.\\
		正确的个数为 \kuo.

		\fourch{1}{2}{3}{4}
	\end{titwo}

	\begin{titwo}
		设两个相互独立的事件 $A$ 与 $B$ 至少有一个发生的概率为 $\frac{8}{9}$,且 $A$ 发生 $B$ 不发生的概率与 $B$ 发生 $A$ 不发生的概率相等,则 $P(A) = $ \htwo.
	\end{titwo}

	\begin{titwo}
		设事件 $A$, $B$, $C$ 两两独立,三个事件不能同时发生,且它们的概率相等,则 $P(A \cup B \cup C)$ 的最大值为 \htwo.
	\end{titwo}

	\begin{titwo}
		事件 $A$ 与 $B$ 相互独立,$P(A) = a$, $P(B) = b$,如果事件 $C$ 发生必然导致 $A$ 与 $B$ 同时发生,则 $A$, $B$, $C$ 都不发生的概率为 \htwo.
	\end{titwo}

	\begin{titwo}
		设 $A$, $B$ 是任意两个事件,则 $P \bigl[ \bigl( \overline{A} \cup B \bigr) \bigl( A \cup B \bigr) \bigl( \overline{A} \cup \overline{B} \bigr) \bigl( A \cup \overline{B} \bigr) \bigr] = $ \htwo.
	\end{titwo}

	\begin{titwo}
		设随机事件 $A$, $B$, $C$ 满足 $P(A|B) + P \bigl( \overline{A} | \overline{B} \bigr) = 1$,且 $P\bigl( A \overline{B} \bigr) = P\bigl( \overline{A} B \bigr) = \frac{1}{4}$, $C = A \cup B$,则 $P(AB|C) = $ \htwo.
	\end{titwo}

	\begin{titwo}
		一批产品共有 $10$ 个正品和 $2$ 个次品,任意抽取两次,每次抽一个,抽出后不再放回,则第二次抽出的是次品的概率为 \htwo.
	\end{titwo}

	\begin{titwo}
		设有大小相同、标号分别为 $1$, $2$, $3$, $4$, $5$ 的五个球,同时有标号为 $1$, $2$, $\cdots$, $10$ 的十个空盒. 将五个球随机放入这十个空盒中,设每个球放入任何一个盒子的可能性都是一样的,并且每个空盒可以放多个球,计算下列事件的概率:
		\begin{enumerate}
			\item $A = $ \{某指定的五个盒子中各有一个球\};
			\item $B = $ \{每个盒子中最多只有一个球\};
			\item $C = $ \{某个指定的盒子不空\}.
		\end{enumerate}
	\end{titwo}

	\begin{titwo}
		设 $AB \subset C$. 试证明:$P(A) + P(B) - P(C) \leq 1$.
	\end{titwo}

	\begin{titwo}
		设 $P(A) > 0$, $P(B) > 0$. 证明:$A$, $B$ 互不相容与 $A$, $B$ 相互独立不能同时成立.
	\end{titwo}

	\begin{titwo}
		证明:若三事件 $A$, $B$, $C$ 相互独立,则 $A \cup B$ 及 $A - B$ 都与 $C$ 相互独立.
	\end{titwo}

	\begin{titwo}
		袋中有 $5$ 只白球 $6$ 只黑球,从袋中一次取出 $3$ 个球,发现都是同一颜色,求这颜色是黑色的概率.
	\end{titwo}

	\begin{titwo}
		甲袋中有 $3$ 个白球 $2$ 个黑球,乙袋中有 $4$ 个白球 $4$ 个黑球,今从甲袋中任取 $2$ 球放入乙袋,再从乙袋中任取一球,求该球是白球的概率.
	\end{titwo}

	\begin{titwo}
		某彩票每周开奖一次,每次提供十万分之一的中奖机会,且各周开奖是相互独立的. 某彩民每周买一次彩票,坚持十年(每年 $52$ 周),那么他从未中奖的概率是多少?
	\end{titwo}

	\begin{titwo}
		随机地取两个正数 $x$ 和 $y$,这两个数中的每一个都不超过 $1$,试求 $x$ 与 $y$ 之和不超过 $1$,积不小于 $0.09$ 的概率.
	\end{titwo}

	\begin{titwo}
		设有甲、乙两名射击运动员,甲命中目标的概率是 $0.6$,乙命中目标的概率是 $0.5$,求下列事件的概率:
		\begin{enumerate}
			\item 从甲、乙中任选一人去射击,若目标被命中,则是甲命中的概率;
			\item 甲、乙两人各自独立射击,若目标被命中,则是甲命中的概率.
		\end{enumerate}
	\end{titwo}

	\begin{titwo}
		验收成箱包装的玻璃器皿,每箱 $24$ 只装. 统计资料表明,每箱最多有 $2$ 只残品,且含 $0$, $1$, $2$ 件残品的箱各占 \SI{80}{\percent}, \SI{15}{\percent}, \SI{5}{\percent}. 现在随机抽取一箱,随机检验其中 $4$ 只; 若未发现残品则通过验收,否则要逐一检验并更换. 试求:
		\begin{enumerate}
			\item 一次通过验收的概率;
			\item 通过验收的箱中确实无残品的概率.
		\end{enumerate}
	\end{titwo}

	\begin{titwo}
		甲、乙、丙三人向一架飞机进行射击,他们的命中率分别为 $0.4$, $0.5$, $0.7$. 设飞机中一弹而被击落的概率为 $0.2$,中两弹而被击落的概率为 $0.6$,中三弹必然被击落,今三人各射击一次,求飞机被击落的概率.
	\end{titwo}

	\begin{titwo}
		某学生想借张宇编著的《张宇高等数学 18 讲》,决定到三个图书馆去借,对每一个图书馆而言,有无这本书的概率均为 $0.5$; 若有,能否借到的概率也均为 $0.5$,假设这三个图书馆采购、出借图书相互独立,求该生能借到此书的概率.
	\end{titwo}

	\begin{titwo}
		假设有四张同样的卡片,其中三张上分别只印有 $a_{1}$, $a_{2}$, $a_{3}$,而另一张上同时印有 $a_{1}$, $a_{2}$, $a_{3}$. 现在随意抽取一张卡片,令 $A_{k} = $ \{卡片上印有 $a_{k}$\}. 证明:事件 $A_{1}$, $A_{2}$, $A_{3}$ 两两独立但不相互独立.
	\end{titwo}

	\begin{titwo}
		在电视剧《乡村爱情》中,谢广坤家中生了一对龙凤胎,专业上叫异性双胞胎. 假设双胞胎中第一个是男孩的概率为 \SI{51}{\percent},同性双胞胎是异性双胞胎的 $3$ 倍,已知一双胞胎第一个是男孩,试求第二个也是男孩的概率.
	\end{titwo}