\section{大数定律与中心极限定理}
	\begin{titwo}
		已知随机变量 $X_{n}(n = 1,2,\cdots)$ 相互独立且都在 $(-1,1)$ 上服从均匀分布,根据独立同分布中心极限定理有 $\lim_{n \to \infty} P \biggl\{ \sum_{i=1}^{n} X_{i} \leq \sqrt{n} \biggr\} = $ \kuo.

		\fourch{$\varPhi(0)$}{$\varPhi(1)$}{$\varPhi\bigl(\sqrt{3}\bigr)$}{$\varPhi(2)$}
	\end{titwo}

	\begin{titwo}
		设随机变量 $X$ 的数学期望 $EX = 75$,方差 $DX = 5$,由切比雪夫不等式估计得
		\[
			P\{ |X - 75| \geq k \} \leq 0.05,
		\]
		则 $k = $ \htwo.
	\end{titwo}

	\begin{titwo}
		设 $X_{1}$, $X_{2}$, $\cdots$, $X_{n}$ 是相互独立的随机变量序列,且都服从参数为 $\lambda$ 的泊松分布,则 $\lim_{n \to \infty} P \Biggl\{ \frac{ \sum_{i=1}^{n} X_{i} - n \lambda }{\sqrt{n \lambda}} \leq x \Biggr\} = $ \htwo.
	\end{titwo}

	\begin{titwo}
		设 $Y \sim \chi^{2}(200)$,则由中心极限定理得 $P\{ Y \leq 200 \}$ 近似等于 \htwo.
	\end{titwo}

	\begin{titwo}
		设 $\{ X_{n} \}$ 是一随机变量序列,$X_{n}$ $(n = 1$, $2$, $\cdots)$ 的概率密度为
		\[
			f_{n}(x) = \frac{n}{\uppi( 1 + n^{2} x^{2} )}, -\infty < x < +\infty,
		\]
		试证:$X_{n}  \xrightarrow{P} 0$.
	\end{titwo}

	\begin{titwo}
		设 $X_{1}$, $X_{2}$, $\cdots$, $X_{n}$ 是独立同分布的随机变量序列,$EX_{i} = \mu$, $DX_{i} = \sigma^{2}$, $i = 1$, $2$, $\cdots$, $n$,令 $Y_{n} = \frac{2}{n(n + 1)} \* \sum_{i=1}^{n} iX_{i}$. 证明:随机变量序列 $\{Y_{n}\}$ 依概率收敛于 $\mu$.
	\end{titwo}

	\begin{titwo}
		若 $DX = 0.004$,利用切比雪夫不等式估计概率 $P\{ |X - EX| < 0.2 \}$.
	\end{titwo}

	\begin{titwo}
		用切比雪夫不等式确定,掷一均质硬币时,需掷多少次,才能保证“正面”出现的频率在 $0.4$ 至 $0.6$ 之间的概率不小于 $0.9$.
	\end{titwo}

	\begin{titwo}
		设事件 $A$ 出现的概率为 $p = 0.5$,试利用切比雪夫不等式,估计在 $1000$ 次独立重复试验中事件 $A$ 出现的次数在 $450$ 到 $550$ 次之间的概率 $\alpha$.
	\end{titwo}

	\begin{titwo}
		设随机变量 $X$ 的概率密度为 $f(x)$,已知方差 $DX = 1$,而随机变量 $Y$ 的概率密度为 $f(-y)$,且 $X$ 与 $Y$ 的相关系数为 $-\frac{1}{4}$. 记 $Z = X + Y$.
		\begin{enumerate}
			\item 求 $EZ$, $DZ$;
			\item 用切比雪夫不等式估计 $P\{|Z| \geq 2\}$.
		\end{enumerate}
	\end{titwo}

	\begin{titwo}
		某计算机系统有 $100$ 个终端,每个终端有 \SI{20}{\percent} 的时间在使用,若各个终端使用与否相互独立,试求有 $10$ 个或更多个终端在使用的概率.
	\end{titwo}

	\begin{titwo}
		某保险公司接受了 \num{10000} 辆电动自行车的保险,每辆车每年的保费为 $12$ 元. 若车丢失,则赔偿车主 $1000$ 元. 假设车的丢失率为 $0.006$,对于此项业务,试利用中心极限定理,求保险公司:
		\begin{enumerate}
			\item 亏损的概率 $\alpha$;
			\item 一年获利润不少于 \num{40000} 元的概率 $\beta$;
			\item 一年获利润不少于 \num{60000} 元的概率 $\gamma$.
		\end{enumerate}
	\end{titwo}

	\begin{titwo}
		一生产线生产的产品成箱包装,每箱的重量是随机的,假设每箱平均重量 $50$ 千克,标准差为 $5$ 千克,若用最大载重为 $5$ 吨的汽车承运,试用中心极限定理说明每辆车最多可装多少箱,才能保证不超载的概率大于 $0.977$ $(\varPhi(2) = 0.977)$.
	\end{titwo}