\section{一维随机变量及其分布}
	\begin{titwo}
		设 $X_{1}$, $X_{2}$ 为相互独立的随机变量,分布函数分别为 $F_{1}(x)$, $F_{2}(x)$,则下列选项一定是某一随机变量的分布函数的为 \kuo.

		\twoch{$F_{1}(x) + F_{2}(x)$}{$F_{1}(x) - F_{2}(x)$}{$F_{1}(x) F_{2}(x)$}{$F_{1}(x) / F_{2}(x)$}
	\end{titwo}

	\begin{titwo}
		设随机变量 $X$ 的分布函数为 $F(x)$,概率密度为 $f(x) = af_{1}(x) + bf_{2}(x)$,其中 $f_{1}(x)$ 是正态分布 $N\bigl(0,$ $\sigma^{2}\bigr)$ 的概率密度,$f_{2}(x)$ 是参数为 $\lambda$ 的指数分布的概率密度,已知 $F(0) = \frac{1}{8}$,则 \kuo.

		\twoch{$a = 1$, $b = 0$}{$a = \frac{3}{4}$, $b = \frac{1}{4}$}{$a = \frac{1}{2}$, $b = \frac{1}{2}$}{$a = \frac{1}{4}$, $b = \frac{3}{4}$}
	\end{titwo}

	\begin{titwo}
		设随机变量 $X$ 的分布函数为 $F(x)$,概率密度为
		\[
			f(x) = \begin{cases}
				Ax(1 - x), & 0 \leq x \leq 1, \\
				0, & \text{其他},
			\end{cases}
		\]
		其中 $A$ 为常数,则 $F\bigl( \frac{1}{2} \bigr) = $ \kuo.

		\fourch{$\frac{1}{2}$}{$\frac{1}{3}$}{$\frac{1}{4}$}{$\frac{1}{5}$}
	\end{titwo}

	\begin{titwo}
		设随机变量 $X$ 的概率密度为
		\[
			f(x) = \begin{cases}
				A \ee^{-x}, & x > \lambda, \\
				0, & \text{其他}
			\end{cases}(\lambda > 0),
		\]
		则概率 $P\{ \lambda < X < \lambda + a \}(a > 0)$ 的值 \kuo.

		\onech{与 $a$ 无关,随 $\lambda$ 增大而增大}{与 $a$ 无关,随 $\lambda$ 增大而减小}{与 $\lambda$ 无关,随 $a$ 增大而增大}{与 $\lambda$ 无关,随 $a$ 增大而减小}
	\end{titwo}

	\begin{titwo}
		设随机变量 $X$ 与 $Y$ 均服从正态分布,$X \sim N\bigl(\mu,$ $4^{2}\bigr)$, $Y \sim N\bigl( \mu,5^{2} \bigr)$,记 $p_{1} = P \{ X \leq \mu - 4 \}$, $p_{2} = P \{ Y \geq \mu + 5 \}$,则 \kuo.

		\onech{对任意实数 $\mu$,都有 $p_{1} = p_{2}$}{对任意实数 $\mu$,都有 $p_{1} < p_{2}$}{只对 $\mu$ 的个别值,才有 $p_{1} = p_{2}$}{对任意实数 $\mu$,都有 $p_{1} > p_{2}$}
	\end{titwo}

	\begin{titwo}
		设随机变量 $X$ 的概率密度为 $f_{X}(x) = \frac{1}{\uppi (1 + x^{2})}$,则 $Y = 2X$ 的概率密度为 \kuo.

		\fourch{$\frac{1}{\uppi (1 + 4 y^{2})}$}{$\frac{1}{\uppi (4 + y)^{2}}$}{$\frac{2}{\uppi (4 + y^{2})}$}{$\frac{2}{\uppi (1 + y^{2})}$}
	\end{titwo}

	\begin{titwo}
		$X \sim N(\mu,\sigma^{2})$, $F(x)$ 为其分布函数,则随机变量 $Y = F(X)$ 的分布函数 \kuo.

		\twoch{处处可导}{恰有 $1$ 个不可导点}{恰有 $2$ 个不可导点}{恰有 $3$ 个不可导点}
	\end{titwo}

	\begin{titwo}
		设随机变量 $X$ 的分布函数为
		\[
		F(x) = \begin{cases}
			0, & x \leq 0, \\
			A \sin x + B, & 0 < x \leq \frac{\uppi}{2}, \\
			1, & x > \frac{\uppi}{2}.
		\end{cases}
		\]
		则 $A$, $B$ 的值依次为 \htwo.
	\end{titwo}

	\begin{titwo}
		设随机变量 $X$ 服从泊松分布,且 $P\{ X \leq 1 \} = 4 \* P \{ X = 2 \}$,则 $P\{ X = 3 \} = $ \htwo.
	\end{titwo}

	\begin{titwo}
		将一枚硬币重复掷五次,则正面、反面都至少出现两次的概率为 \htwo.
	\end{titwo}

	\begin{titwo}
		已知每次试验“成功”的概率为 $p$,现进行 $n$ 次独立试验,则在没有全部失败的条件下,“成功”不止一次的概率为 \htwo.
	\end{titwo}

	\begin{titwo}
		设 $X$ 服从参数为 $\lambda$ 的指数分布,对 $X$ 作三次独立重复观察,至少有一次观测值大于 $2$ 的概率为 $\frac{7}{8}$,则 $\lambda = $ \htwo.
	\end{titwo}

	\begin{titwo}
		设随机变量 $X$ 与 $-X$ 服从同一均匀分布 $U(a,$ $b)$,已知 $X$ 的概率密度 $f(x)$ 的平方 $f^{2}(x)$ 也是概率密度,则 $b = $ \htwo.
	\end{titwo}

	\begin{titwo}
		设随机变量 $X$ 服从正态分布,其概率密度为
		\[
			f(x) = k \ee^{-x^{2} + 2x - 1} (-\infty < x < +\infty),
		\]
		则常数 $k = $ \htwo.
	\end{titwo}

	\begin{titwo}
		市场上某产品由甲、乙两厂各生产 $\frac{1}{2}$,已知甲厂和乙厂的产品指标分别服从分布函数 $F_{1}(x)$ 和 $F_{2}(x)$,现从市场上任取一件产品,则其指标服从的分布函数为 \htwo.
	\end{titwo}

	\begin{titwo}
		设随机变量 $X$ 服从 $(0,2)$ 上的均匀分布,则随机变量 $Y = X^{2}$ 在 $(0,4)$ 内的概率密度 $f_{Y}(y) = $ \htwo.
	\end{titwo}

	\begin{titwo}
		一汽车沿一街道行驶,需要通过三个设有红绿信号灯的路口,每个信号灯为红或绿相互独立,且每一信号灯红绿两种信号显示的概率均为 $\frac{1}{2}$,以 $X$ 表示该汽车首次遇到红灯前已通过的路口的个数,求 $X$ 的概率分布.
	\end{titwo}

	\begin{titwo}
		一实习生用一台机器接连生产了三个同种零件,第 $i$ 个零件是不合格品的概率 $p_{i} = \frac{1}{i + 1} (i = 1,2$, $3)$,以 $X$ 表示三个零件中合格品的个数,求 $X$ 的分布律.
	\end{titwo}

	\begin{titwo}
		设随机变量 $X$ 的分布函数为
		\[
			F(x) = A + B \arctan x, -\infty < x < +\infty,
		\]
		求:
		\begin{enumerate}
			\item 系数 $A$ 与 $B$;
			\item $P\{ -1 < X \leq 1 \}$;
			\item $X$ 的概率密度.
		\end{enumerate}
	\end{titwo}

	\begin{titwo}
		设电子管寿命 $X$ 的概率密度为
		\[
			f(x) = \begin{cases}
				\frac{100}{x^{2}}, & x > 100, \\
				0, & \text{其他},
			\end{cases}
		\]
		若一台收音机上装有三个这种电子管,求:
		\begin{enumerate}
			\item 使用的最初 $150$ 小时内,至少有两个电子管被烧坏的概率;
			\item 在使用的最初 $150$ 小时内烧坏的电子管数 $Y$ 的分布律;
			\item $Y$ 的分布函数.
		\end{enumerate}
	\end{titwo}

	\begin{titwo}
		设顾客在某银行窗口等待服务的时间 $X$ (单位:分)服从参数为 $\frac{1}{5}$ 的指数分布. 若等待时间超过 $10$ 分钟,他就离开. 设他一个月内要来银行 $5$ 次,以 $Y$ 表示一个月内他没有等到服务而离开窗口的次数,求 $Y$ 的分布律及 $P\{ Y \geq 1 \}$.
	\end{titwo}

	\begin{titwo}
		向半径为 $r$ 的圆内随机抛一点,求此点到圆心的距离 $X$ 的分布函数 $F(x)$,并求 $P \bigl\{ X > \frac{2r}{3} \bigr\}$.
	\end{titwo}

	\begin{titwo}
		设随机变量 $X$ 的概率密度为
		\[
			f(x) = \begin{cases}
				x, & 0 \leq x < 1, \\
				2 - x, & 1 \leq x < 2, \\
				0, & \text{其他},
			\end{cases}
		\]
		求 $X$ 的分布函数.
	\end{titwo}

	\begin{titwo}
		假设随机变量 $X$ 服从参数为 $\lambda$ 的指数分布,求随机变量 $Y = 1 - \ee^{-\lambda X}$ 的概率密度 $f_{Y}(y)$.
	\end{titwo}

	\begin{titwo}
		设随机变量 $X$ 的概率密度为
		\[
		f(x) = \begin{cases}
			\frac{1}{3 \sqrt[3]{x^{2}}}, & 1 \leq x < 8, \\
			0, & \text{其他},
		\end{cases}
		\]
		$F(x)$ 是 $X$ 的分布函数,求随机变量 $Y = F(X)$ 的分布函数.
	\end{titwo}

	\begin{titwo}
		在独立的伯努利试验中,若 $p$ 为一次试验中成功的概率. 以 $X$ 记为第 $r$ 次成功出现时的试验次数,则 $X$ 是随机变量,取值为 $r$, $r+1$, $\cdots$,称为负二项分布,记为 $Nb(r,p)$,其概率分布为:
		\[
			P\{X = k\} = \CC_{k-1}^{r-1} p^{r} (1 - p)^{k-r}, k = r,r+1,\cdots.
		\]
		\begin{enumerate}
			\item 记 $Y_{1}$ 表示首次成功的试验次数,$Y_{2}$ 表示第 $1$ 次成功后到第 $2$ 次成功为止共进行的试验次数,证明 $X = Y_{1} + Y_{2} \sim Nb(2,p)$;
			\item 设试验成功的概率为 $\frac{3}{4}$,失败的概率为 $\frac{1}{4}$,独立重复试验直到成功两次为止,求试验次数的数学期望、方差.
		\end{enumerate}
	\end{titwo}