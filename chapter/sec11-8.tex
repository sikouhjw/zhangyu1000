\section{区间估计与假设检验}
	\begin{titwo}
		设总体 $X \sim N \bigl( \mu,\sigma^{2} \bigr)$,来自 $X$ 的一个样本为 $X_{1}$, $X_{2}$, $\cdots$, $X_{2n}$,记 $\overline{X} = \frac{1}{n} \sum_{i=1}^{n} X_{i}$, $T = \sum_{i=1}^{n} \bigl( X_{i} - \overline{X} \bigr)^{2} + \sum_{i=n+1}^{2n} (X_{i} - \mu)^{2}$,当 $\mu$ 已知时,基于 $T$ 构造估计 $\sigma^{2}$ 的置信水平为 $1 - \alpha$ 的置信区间为 \kuo.

		\onech{$\Biggl( \frac{T}{\chi_{1 - \frac{\alpha}{2}}^{2}(2n)}, \frac{T}{\chi_{\frac{\alpha}{2}}^{2} (2n) } \Biggr)$}%
		{$\Biggl( \frac{T}{\chi_{\frac{\alpha}{2}}^{2}(2n)}, \frac{T}{\chi_{1 - \frac{\alpha}{2}}^{2} (2n) } \Biggr)$}%
		{$\Biggl( \frac{T}{\chi_{1 - \frac{\alpha}{2}}^{2}(2n-1)}, \frac{T}{\chi_{\frac{\alpha}{2}}^{2} (2n-1) } \Biggr)$}%
		{$\Biggl( \frac{T}{\chi_{\frac{\alpha}{2}}^{2}(2n-1)}, \frac{T}{\chi_{1 - \frac{\alpha}{2}}^{2} (2n-1) } \Biggr)$}
	\end{titwo}

	\begin{titwo}
		设 $\overline{X}$ 为来自总体 $X \sim N \bigl( \mu,\sigma^{2} \bigr)$ ($\sigma^{2}$ 未知)的一个简单随机样本的样本均值,若已知在置信水平 $1 - \alpha$ 下,$\mu$ 的置信区间长度为 $2$,则在显著性水平 $\alpha$ 下,对于假设检验问题 $H_{0}: \mu = 1$, $H_{1}: \mu \ne 1$,要使得检验结果接受 $H_{0}$,则应有 \kuo.

		\twoch{$\overline{X} \in (-1,1)$}{$\overline{X} \in (-1,3)$}{$\overline{X} \in (-2,2)$}{$\overline{X} \in (0,2)$}
	\end{titwo}

	\begin{titwo}
		设总体 $X$ 的方差为 $1$,根据来自 $X$ 的容量为 $100$ 的简单随机样本测得样本均值为 $5$,则 $X$ 的数学期望的置信度近似等于 $0.95$ 的置信区间为 \htwo.
	\end{titwo}

	\begin{titwo}
		设总体 $X \sim N(\mu,8)$, $X_{1}$, $X_{2}$, $\cdots$, $X_{36}$ 是来自 $X$ 的简单随机样本,$\overline{X}$ 是它的均值. 如果 $\bigl( \overline{X} - 1,\overline{X} + 1 \bigr)$ 是未知参数 $\mu$ 的置信区间,则置信水平为 \htwo.
	\end{titwo}

	\begin{titwo}
		设总体 $X \sim N(\mu,8)$, $\mu$ 为未知参数,$X_{1}$, $X_{2}$, $\cdots$, $X_{32}$ 是取自总体 $X$ 的一个简单随机样本,如果以区间 $\bigl[ \overline{X} - 1,\overline{X} + 1 \bigr]$ 作为 $\mu$ 的置信区间,则置信水平为 \htwo. (精确到 $3$ 位小数,参考数值:$\varPhi(2) = 0.977$, $\varPhi(3) \approx 0.999$, $\varPhi(4) \approx 1$)
	\end{titwo}

	\begin{titwo}
		某种零件的尺寸方差为 $\sigma^{2} = 1.21$,抽取一批这类零件中的 $6$ 件检查,得尺寸数据如下(单位:毫米):
		\[
			32.56,29.66,31.64,30.00,21.87,31.03~.
		\]
		设零件尺寸服从正态分布,问这批零件的平均尺寸能否认为是 $32.50$ 毫米 ($\alpha = 0.05$).
	\end{titwo}

	\begin{titwo}
		经测定某批矿砂的 $5$ 个样品中镍含量为 $X$ (\si{\percent}):
		\[
			3.25,3.27,3.24,3.26,3.24,
		\]
		设测定值服从正态分布,问能否认为这批矿砂的镍含量为 $3.25$ ($\alpha = 0.01$)?
	\end{titwo}

	\begin{titwo}
		从一批轴料中取 $15$ 件测量其椭圆度,计算得 $S = 0.025$,问该批轴料椭圆度的总体方差与规定的 $\sigma^{2} = 0.0004$ 有无显著差别. ($\alpha = 0.05$,椭圆度服从正态分布)
	\end{titwo}

	\begin{titwo}
		设某产品的指标服从正态分布,它的标准差为 $\sigma = 100$,今抽了一个容量为 $26$ 的样本,计算平均值为 $1580$,问在显著性水平 $\alpha = 0.05$ 下,能否认为这批产品的指标的期望值 $\mu$ 不低于 $1600$.
    \end{titwo}
    \guanggao\newpage\guanggao