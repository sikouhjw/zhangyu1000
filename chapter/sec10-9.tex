\section{相似}
	\begin{titwo}
		下列矩阵中能相似于对角矩阵的是 \kuo.

		\twoch{$\bm A = \begin{bsmallmatrix}
			1 & 2 & 0 \\
			0 & 1 & 0 \\
			0 & 0 & 2
		\end{bsmallmatrix}$}{$\bm B = \begin{bsmallmatrix}
			1 & 0 & 2 \\
			0 & 2 & 0 \\
			0 & 0 & 1
		\end{bsmallmatrix}$}{$\bm C = \begin{bsmallmatrix}
			1 & 2 & 0 \\
			0 & 2 & 0 \\
			0 & 0 & 1
		\end{bsmallmatrix}$}{$\bm D = \begin{bsmallmatrix}
			1 & 1 & 1 \\
			0 & 1 & 0 \\
			0 & 0 & 2
		\end{bsmallmatrix}$}
	\end{titwo}

	\begin{titwo}
		下列矩阵中不能相似于对角矩阵的是 \kuo.

		\twoch{$\bm A = \begin{bsmallmatrix}
			1 & 1 & 0 \\
			0 & 1 & 0 \\
			0 & 0 & 2
		\end{bsmallmatrix}$}{$\bm B = \begin{bsmallmatrix}
			1 & 0 & 0 \\
			0 & 1 & 1 \\
			0 & 0 & 2
		\end{bsmallmatrix}$}{$\bm C = \begin{bsmallmatrix}
			1 & 1 & 0 \\
			0 & 2 & 1 \\
			0 & 0 & 3
		\end{bsmallmatrix}$}{$\bm D = \begin{bsmallmatrix}
			1 & 0 & 0 \\
			0 & 1 & 1 \\
			0 & 1 & 2
		\end{bsmallmatrix}$}
	\end{titwo}

	\begin{titwo}
		设 $\bm A = \begin{bsmallmatrix}
			3 & 0 & 0 \\
			0 & 1 & 2 \\
			2 & 0 & 5
		\end{bsmallmatrix}$, $\bm B = \begin{bsmallmatrix}
			2 & 1 & 0 \\
			0 & 2 & 0 \\
			3 & 0 & 5
		\end{bsmallmatrix}$, $\bm C = \begin{bsmallmatrix}
			3 & 0 & 2 \\
			0 & 1 & 0 \\
			2 & 0 & 5
		\end{bsmallmatrix}$, $\bm D = \begin{bsmallmatrix}
			2 & 0 & 0 \\
			0 & 2 & 0 \\
			3 & 1 & 5
		\end{bsmallmatrix}$,其中与对角矩阵相似的有 \kuo.

		\fourch{$\bm A,\bm B,\bm C$}{$\bm B,\bm D$}{$\bm A,\bm C,\bm D$}{$\bm A,\bm C$}
	\end{titwo}

	\begin{titwo}
		设 $\bm A$ 是 $n$ 阶矩阵,满足 $\bm A^{2} = \bm A$,且 $r(\bm A) = r (0 < r \leq n)$. 证明:
		\[
			\bm A \sim \begin{bsmallmatrix}
				\bm E_{r} & \bm O \\
				\bm O & \bm O
			\end{bsmallmatrix},
		\]
		其中 $\bm E_{r}$ 是 $r$ 阶单位矩阵.
	\end{titwo}

	\begin{titwo}
		设 $\bm A$, $\bm B$ 均为 $n$ 阶矩阵,$\bm A$ 有 $n$ 个互不相同的特征值,且 $\bm A \bm B = \bm B \bm A$. 证明:$\bm B$ 相似于对角矩阵.
	\end{titwo}

	\begin{titwo}
		设 $\bm A = \bm E + \bm \alpha \bm \beta^{\TT}$,其中 $\bm \alpha = [a_{1},a_{2},\cdots,a_{n}]^{\TT} \ne \bm 0$, $\bm \beta = [b_{1},b_{2},\cdots,b_{n}]^{\TT} \ne \bm 0$,且 $\bm \alpha^{\TT} \bm \beta = 2$.
		\begin{enumerate}
			\item 求 $\bm A$ 的特征值和特征向量;
			\item 求可逆矩阵 $\bm P$,使得 $\bm P^{-1} \bm A \bm P = \bm \varLambda$.
		\end{enumerate}
	\end{titwo}

	\begin{titwo}
		设向量 $\bm \alpha = [a_{1},a_{2},\cdots,a_{n}]^{\TT}$, $\bm \beta = [b_{1},\allowbreak b_{2},\allowbreak \cdots,\allowbreak b_{n}]^{\TT}$ 都是非零向量,且满足条件 $\bm \alpha^{\TT} \bm \beta = 0$,记 $n$ 阶矩阵 $\bm A = \bm \alpha \bm \beta^{\TT}$,求:
		\begin{enumerate}
			\item $\bm A^{2}$;
			\item $\bm A$ 的特征值和特征向量;
			\item $\bm A$ 能否相似于对角矩阵,说明理由.
		\end{enumerate}
	\end{titwo}

	\begin{titwo}
		设 $a_{0},a_{1},\cdots,a_{n-1}$ 是 $n$ 个实数,方阵
		\[
			\bm A = \begin{bsmallmatrix}
				0 & 1 & 0 & \cdots & 0 & 0 \\
				0 & 0 & 1 & \cdots & 0 & 0 \\
				\vdots & \vdots & \vdots & & \vdots & \vdots \\
				0 & 0 & 0 & \cdots & 0 & 1 \\
				-a_{0} & -a_{1} & -a_{2} & \cdots & -a_{n-2} & -a_{n-1}
			\end{bsmallmatrix}
		\]
		\begin{enumerate}
			\item 若 $\lambda$ 是 $\bm A$ 的特征值,证明 $\bm \xi = \bigl[1,\allowbreak \lambda,\allowbreak \lambda^{2},\allowbreak \cdots,\allowbreak \lambda^{n-1}\bigr]^{\TT}$ 是 $\bm A$ 的对应于特征值 $\lambda$ 的特征向量;
			\item 若 $\bm A$ 有 $n$ 个互异的特征值 $\lambda_{1},\lambda_{2},\cdots,\lambda_{n}$,求可逆矩阵 $\bm P$,使 $\bm P^{-1} \bm A \bm P = \bm \varLambda$.
		\end{enumerate}
	\end{titwo}

	\begin{titwo}
		设 $\bm A$ 是 $3$ 阶矩阵,$\bm \alpha_{1}$, $\bm \alpha_{2}$, $\bm \alpha_{3}$ 是 $3$ 维列向量,$\bm \alpha_{1} \ne \bm 0$,满足 $\bm A \bm \alpha_{1} = 2 \bm \alpha_{1}$, $\bm A \bm \alpha_{2} = \bm \alpha_{1} + 2 \bm \alpha_{2}$, $\bm A \bm \alpha_{3} = \bm \alpha_{2} + 2 \bm \alpha_{3}$.
		\begin{enumerate}
			\item 证明 $\bm \alpha_{1}$, $\bm \alpha_{2}$, $\bm \alpha_{3}$ 线性无关;
			\item $\bm A$ 能否相似于对角矩阵,说明理由.
		\end{enumerate}
	\end{titwo}

	\begin{titwo}
		设 $n$ 阶矩阵
		\[
			\bm A = \begin{bsmallmatrix}
				a_{1}b_{1} & a_{1}b_{2} & \cdots & a_{1}b_{n} \\
				a_{2}b_{1} & a_{2}b_{2} & \cdots & a_{2}b_{n} \\
				\vdots & \vdots &  & \vdots \\
				a_{n}b_{1} & a_{n}b_{2} & \cdots & a_{n}b_{n}
			\end{bsmallmatrix}.
		\]
		已知 $\tr(\bm A) = a \ne 0$. 证明:矩阵 $\bm A$ 相似于对角矩阵.
	\end{titwo}

	\begin{titwo}
		已知 $\bm A \sim \begin{bsmallmatrix}
			1 & 0 & 0 \\
			0 & 1 & 0 \\
			0 & 0 & -2
		\end{bsmallmatrix}$,则 $r(\bm A - \bm E) + r(2 \bm E + \bm A) = $ \htwo.
	\end{titwo}

	\begin{titwo}
		已知矩阵 $\bm A = \begin{bsmallmatrix}
			2 & 0 & 0 \\
			0 & 0 & 1 \\
			0 & 1 & x
		\end{bsmallmatrix}$ 与 $\bm B = \begin{bsmallmatrix}
			2 & 0 & 0 \\
			0 & y & 0 \\
			0 & 0 & -1
		\end{bsmallmatrix}$ 相似.
		\begin{enumerate}
			\item 求 $x$ 与 $y$;
			\item 求一个满足 $\bm P^{-1} \bm A \bm P = \bm B$ 的可逆矩阵 $\bm P$.
		\end{enumerate}
	\end{titwo}

	\begin{titwo}
		已知矩阵 $\bm A = \begin{bsmallmatrix}
			1 & 0 & -1 \\
			0 & 1 & 0 \\
			-2 & 1 & 0
		\end{bsmallmatrix}$ 与 $\bm B = \begin{bsmallmatrix}
			2 & 3 & 3 \\
			2 & 1 & 0 \\
			a & b & c
		\end{bsmallmatrix}$ 相似,求 $a$, $b$, $c$ 及可逆矩阵 $\bm P$,使 $\bm P^{-1} \bm A \bm P = \bm B$.
	\end{titwo}

	\begin{titwo}
		设 $\bm A = \begin{bsmallmatrix}
			8 & -2 & -2 \\
			-2 & 5 & -4 \\
			-2 & -4 & 5
		\end{bsmallmatrix}$,求实对称矩阵 $\bm B$,使 $\bm A = \bm B^{2}$.
	\end{titwo}

	\begin{titwo}
		证明:$\bm A \sim \bm B$,其中
		\[
			\bm A = \begin{bsmallmatrix}
				1 & & & & \\
				& 2 & & & \\
				& & \ddots & & \\
				& & & n-1 & \\
				& & & & n
			\end{bsmallmatrix},
			\bm B = \begin{bsmallmatrix}
				n & & & & \\
				& n-1 & & & \\
				& & \ddots & & \\
				& & & 2 & \\
				& & & & 1
			\end{bsmallmatrix}.
		\]
		并求可逆矩阵 $\bm P$,使得 $\bm P^{-1} \bm A \bm P = \bm B$.
	\end{titwo}

	\begin{titwo}
		设 $\bm \alpha = [a_{1},a_{2},\cdots,a_{n}]^{\TT} \ne \bm 0$, $\bm A = \bm \alpha \bm \alpha^{\TT}$,求可逆矩阵 $\bm P$,使 $\bm P^{-1} \bm A \bm P = \bm \varLambda$.
	\end{titwo}

	\begin{titwo}
		设
		\[
			\bm A = \begin{bsmallmatrix}
				2 & -2 & 0 \\
				-2 & 1 & -2 \\
				0 & -2 & 0
			\end{bsmallmatrix},
			\bm B = \begin{bsmallmatrix}
				1 & -2 & -2 \\
				-2 & 2 & 0 \\
				-2 & 0 & 0
			\end{bsmallmatrix},
		\]
		问 $\bm A$, $\bm B$ 是否相似,并说明理由.
	\end{titwo}

	\begin{titwo}
		设矩阵 $\bm A = \begin{bsmallmatrix}
			0 & 1 & 0 & 0 \\
			1 & 0 & 0 & 0 \\
			0 & 0 & k & 1 \\
			0 & 0 & 1 & 2
		\end{bsmallmatrix}$,已知 $\bm A$ 的一个特征值为 $3$.
		\begin{enumerate}
			\item 求 $k$;
			\item 求矩阵 $\bm P$,使 $(\bm A \bm P)^{\TT} (\bm A \bm P)$ 为对角矩阵.
		\end{enumerate}
	\end{titwo}