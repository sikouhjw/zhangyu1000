\documentclass[openany,twocolumn]{ctexbook}
\usepackage{amsmath,upgreek}
\usepackage{fourier}
\usepackage{esint}
\usepackage{physics}
\usepackage{siunitx}
\usepackage{lastpage}
\usepackage{graphicx}
\renewcommand\thefigure{\arabic{chapter}-\arabic{section}-\arabic{figure}}

\usepackage[a4paper,top=2.5cm,bottom=2.5cm,inner=1.5cm,outer=3cm]{geometry}
% \usepackage[toc]{multitoc}

\usepackage{tikz}
\usetikzlibrary{shapes.geometric,calc}

\newcommand\score[2]{
\pgfmathsetmacro\pgfxa{#1+1}
\tikzstyle{scorestars}=[star, star points=5, star point ratio=2.25, draw,inner sep=0.15em,anchor=outer point 3]
\begin{tikzpicture}[baseline]
  \foreach \i in {1,...,#2} {
	\pgfmathparse{(\i<=#1?"black":"white")}
	\edef\starcolor{\pgfmathresult}
	\draw (\i*1em,0) node[name=star\i,scorestars,fill=\starcolor]  {};
   }
   \pgfmathparse{(#1>int(#1)?int(#1+1):0}
   \let\partstar=\pgfmathresult
   \ifnum\partstar>0
	 \pgfmathsetmacro\starpart{#1-(int(#1))}
	 \path [clip] ($(star\partstar.outer point 3)!(star\partstar.outer point 2)!(star\partstar.outer point 4)$) rectangle 
	($(star\partstar.outer point 2 |- star\partstar.outer point 1)!\starpart!(star\partstar.outer point 1 -| star\partstar.outer point 5)$);
	 \fill (\partstar*1em,0) node[scorestars,fill=black]  {};
   \fi
,\end{tikzpicture}
}

\setlength{\headheight}{13pt}
\makeatletter
\usepackage{fancyhdr}
\pagestyle{fancy}
\fancyhf{}
\fancyhead[LO]{\slshape \rightmark}
\fancyhead[RE]{\slshape \leftmark}
% \fancyhead[LE,RO]{
% \@ifundefined{lastpage@lastpage}{%
% 	\score{0}{10}%
% }{%
% 	\score{10 * \thepage / \lastpage@lastpage}{10}%
% }%
% }
\fancyhead[C]{\slshape 仅供学习使用,严禁商业使用}
\fancyfoot[C]{\thepage}
% \fancyfoot[LE,RO]{\TeX}
\makeatother

\usepackage{caption}
\captionsetup{labelsep=space}


\usepackage{theorem}
\ctexset{
	chapter={
		name={},
		number=0\arabic{chapter},
	},
	section={
		format={\zihao{4}\heiti\centering},
		name={第,章},
		aftername={\hspace{1em}},
		number=\chinese{section},
	},
	subsection={
		format={\zihao{-4}\heiti\raggedright},
		name={,、},
		aftername={\hspace{0bp}},
		number=\chinese{subsection},
	},
	subsubsection={
		format={\zihao{-4}\heiti\raggedright},
		name={},
		aftername={\hspace{5bp}},
		number={\arabic{section}.\arabic{subsection}.\arabic{subsubsection}},
	},
}

{
	\theoremstyle{change}
	\theoremheaderfont{\bfseries}
	\theorembodyfont{\normalfont}
	\newtheorem{ti}{}[section]
}
\renewcommand{\theti}{\arabic{section}.\arabic{ti}}

\usepackage{ulem}
\newcommand{\hone}[1]{ \uline{\hspace{#1 pc}}}
\def\kuo{ (\hspace{1pc})}

\newcommand{\fourch}[4]{\noindent\begin{tabular}{*{4}{@{}p{1.97cm}}}(A)~#1 & (B)~#2 & (C)~#3 & (D)~#4\end{tabular}} % 一行
\newcommand{\twoch}[4]{\noindent\begin{tabular}{*{2}{@{}p{3.94cm}}}(A)~#1 & (B)~#2\end{tabular}\\\begin{tabular}{*{2}{@{}p{3.94cm}}}(C)~#3 & (D)~#4\end{tabular}}  %两行
\newcommand{\onech}[4]{\noindent(A)~#1 \\ (B)~#2 \\ (C)~#3 \\ (D)~#4}  % 四行

\def\leq{\leqslant}
\def\geq{\geqslant}
\def\ee{\mathrm{e}}
\def\CC{\mathrm{C}}
\edef\lim{\lim\limits}

\def\theenumi{\arabic{enumi}}
\def\labelenumi{(\theenumi)}

% \setCJKmainfont{SourceHanSerifCN}[
% UprightFont    = *-Regular,
% BoldFont       = *-Bold,
% ItalicFont     = *-Regular,
% BoldItalicFont = *-Bold
% ]

\usepackage[bookmarksopen=true,bookmarksnumbered=true,hidelinks]{hyperref}

\title{张宇考研数学题源探析经典 1000 题\\(习题分册·数学一)}
\author{张宇}


\begin{document}
	% \maketitle
	% \frontmatter
	% 	\onecolumn
	\chapter{前言}

	按照考研数学历年的命题规律和风格,结合最新的信息,考生在2020年考研复习备考中应做到以下五点:一是将考研基础知识和常规题目作为复习主体;二是要加强综合性试题的训练;三是加强计算能力的培养,使自己具备较强的处理数学计算过程的本领,要知道,绝大多数数学题都是要通过准确的计算才能得到正确答案的;四是要加强应用能力的培养,多做用数学基础知识解决实际问题的题目;五是要全面复习,将考研大纲中的所有知识作为复习范围,不要有所偏颇。以上五点也将是2020考研命题的趋势,请各位考生重视。

	本书是对题源的最新研究成果,它尽力搜集和命制了题源本身或与题源相关的重要考题,值得考生在复习全过程中认真做题、消化。我也将在各种场合对本书的题目进行详细讲解并予以重点提示,以期让考生把握住考试命题方向,准确复习备考。题源和题库研究是公共资源,从2019年考研命题的情况来看,它并不回避市面上已经公开的题源,甚至可考到原题,于是,我很高兴把我们所掌握的信息提供给全国考生,并乐于与大家分享这些资料。这对消除考研数学的神秘感,进一步促进考试的公正性与科学性都会起到重要作用。

	这本《张宇考研数学题源探析经典1000题》最初是按照数学一、数学二、数学三平均1000道题左右来命名的,多年来一直这样叫下来,成为了考研习题集的一个经典名称。事实上,数学一考试内容最多,题目不止1000道,数学二考试内容最少,题目少于1000道,数学三的考试内容居中,近于1000道。

	衷心感谢原命题专家们给予的指导与帮助。希望考生认真研读、操练本书中的每一道题目,提高解题能力,争取考研得到高分。

	\phantom{1}\hspace{\fill} {\LARGE 张宇}
	
	\phantom{1}\hspace{\fill} {2019 年 2 月\quad 于北京}
	\twocolumn
	% \tableofcontents
	% \mainmatter
	% \chapter{高等数学}
	高等数学是硕士研究生招生考试考查内容之一,主要考查考生对高等数学的基本概念、基本理论、基本方法的理解和掌握以及考生的抽象思维能力、逻辑推理能力、综合运用能力和解决实际问题的能力。在考研数学一试卷中分值为82分,约占56\%。
	\newpage
	\section{极限、连续}
	\subsection{函数极限}
	\begin{ti}
		求 $\lim_{x\to 0} \frac{\sqrt{1+x} - 1 - \frac{x}{2}}{\ee^{x^{2}}-1}$.
	\end{ti}

	\begin{ti}
		求 $\lim_{x \to 0} \frac{\ee^{x} + \ln(1 - x) - 1}{x - \arctan x}$.
	\end{ti}

	\begin{ti}
		求 $\lim_{x \to 0} \frac{(1+x)^{\frac{2}{x}} - \ee^{2}[1 - \ln(1+x)]}{x}$.
	\end{ti}

	\begin{ti}
		求 $\lim_{x \to 0} \frac{\left(1 + x^{2}\right)(1 - \cos 2x) - 2x^{2}}{x^{4}}$.
	\end{ti}

	\begin{ti}
		求 $\lim_{x \to 0} \frac{\sqrt{1-x^{2}} \sin^{2}x - \tan^{2}x }{x^{2}[\ln(1+x)]^{2}}$.
	\end{ti}

	\begin{ti}
		求 $\lim_{x \to 0} \frac{(3 + 2 \tan x)^{x} - 3^{x}}{3 \sin^{2}x + x^{3} \cos\frac{1}{x}}$.
	\end{ti}

	\begin{ti}
		求 $\lim_{x \to 2}\frac{\sqrt{5x - 1} - \sqrt{2x + 5}}{x^{2} - 4}$.
	\end{ti}

	\begin{ti}
		求 $\lim_{x \to 0}\int_{0}^{x} \frac{\sin 2t}{\sqrt{4+t^{2}}\int_{0}^{x} \left(\sqrt{t+1} - 1\right)\dd{t}} \dd{t}$.
	\end{ti}

	\begin{ti}
		求 $\lim_{x \to \infty} \ee^{-x} \left( 1 + \frac{1}{x} \right)^{x^{2}}$.
	\end{ti}

	\begin{ti}
		求 $\lim_{x \to 3^{+}} \frac{\cos x \ln (x - 3)}{\ln\left( \ee^{x} - \ee^{3} \right)}$.
	\end{ti}

	\begin{ti}
		求 $\lim_{x \to \infty} x^{2} \left( a^{\frac{1}{x}} + a^{-\frac{1}{x}} - 2 \right)$,其中常数 $a > 0$.
	\end{ti}

	\begin{ti}
		求 $\lim_{x \to 0}\frac{1}{x}\left( \cot x - \frac{1}{x} \right)$.
	\end{ti}

	\begin{ti}
		求 $\lim_{x \to +\infty}\left( \sqrt[3]{x^{3} + 2x^{2} + 1} - x\ee^{\frac{1}{x}} \right)$.
	\end{ti}

	\begin{ti}
		求 $\lim_{x \to 0}\left( \frac{1+x}{1-e^{-x}} - \frac{1}{x} \right)$.
	\end{ti}

	\begin{ti}
		求 $\lim_{x \to 0^{+}} x^{\ln\left( \frac{\ln x - 1}{\ln x + 1} \right)}$.
	\end{ti}

	\begin{ti}
		求 $\lim_{x \to \infty} \left( \tan\frac{\uppi x}{1 + 2x} \right)^{\frac{1}{x}}$.
	\end{ti}

	\begin{ti}
		求 $\lim_{x \to 0^{+}} \left( \frac{\sin x}{x} \right)^{\frac{1}{1 - \cos x}}$.
	\end{ti}

	\begin{ti}
		求 $\lim_{x \to 0}\left( \frac{\cos x}{\cos 2x} \right)^{\frac{1}{x^{2}}}$.
	\end{ti}

	\begin{ti}
		求 $\lim_{x \to 0} \frac{\sin x - x\cos x}{x - \sin x}$.
	\end{ti}

	\begin{ti}
		求 $\lim_{x \to 0}\frac{1 + \frac{1}{2}x^{2} - \sqrt{1 + x^{2}}}{\left( \cos x - \ee^{\frac{x^{2}}{2}} \right) \sin \frac{x^{2}}{2}}$.
	\end{ti}

	\begin{ti}
		求 $\lim_{x \to \infty} \left( \sqrt[6]{x^{6} + x^{5}} - \sqrt[6]{x^{6} - x^{5}} \right)$.
	\end{ti}

	\begin{ti}
		求 $\lim_{x \to +\infty}\left[ \left( x^{3} + \frac{x}{2} - \tan \frac{1}{x} \right) \ee^{\frac{1}{x}} - \sqrt{1 + x^{6}} \right]$.
	\end{ti}

	\begin{ti}
		求 $\lim_{x \to 0}\frac{\ee^{\tan x} - \ee^{\sin x}}{x \sin^{2} x}$.
	\end{ti}

	\begin{ti}
		求 $\lim_{x \to 0} \frac{\sin x + x^{2} \sin\frac{1}{x}}{(1 + \cos x)\ln(1 + x)}$.
	\end{ti}

	\begin{ti}
		求 $\lim_{x \to 0}\left[ \frac{a}{x} - \left( \frac{1}{x^{2}} - a^{2} \right) \ln(1 + ax) \right]$,其中 $a \ne 0$.
	\end{ti}

	\begin{ti}
		求 $\lim_{x \to 0} \frac{(1 + x)^{\frac{1}{x}} - (1 + 2x)^{\frac{1}{2x}}}{\sin x}$.
	\end{ti}

	\begin{ti}
		求 $\lim_{x \to 0}\frac{\int_{0}^{\sin^{2}x} \ln(1 + t)\dd{t}}{\left( \sqrt[3]{1 + x^{3}} - 1 \right)\sin x}$.
	\end{ti}

	\begin{ti}
		求 $\lim_{x \to 0} \frac{ \int_{0}^{x} \left[ \int_{0}^{u^{2}} \arctan(1 + t) \dd{t} \right] \dd{u} }{x(1 - \cos x)}$.
	\end{ti}

	\begin{ti}
		求 $\lim_{x \to 0^{+}} \frac{x^{x} - ( \sin x )^{x}}{x^{2}\ln(1 + x)}$.
	\end{ti}

	\begin{ti}
		求 $\lim_{x \to 0} \frac{ \cos x - \ee^{-\frac{x^{2}}{2}} }{x^{2} [ x + \ln(1 - x) ]}$.
	\end{ti}

	\begin{ti}
		求 $\lim_{x \to 0} \frac{1}{x^{3}} \left[ \left( \frac{2 + \cos x}{3} \right)^{x} - 1 \right]$.
	\end{ti}

	\begin{ti}
		求 $\lim_{x \to 0} \frac{\ln\left( \sin^{2}x + \ee^{x} \right) - x}{\ln\left( x^{2} + \ee^{2x} \right) - 2x}$.
	\end{ti}

	\begin{ti}
		求 $\lim_{x \to 1} \frac{x - x^{x}}{1 - x + \ln x}$.
	\end{ti}

	\begin{ti}
		求 $\lim_{x \to 0} \left( \frac{a_{1}^{x} + a_{2}^{x} + \cdots + a_{n}^{x}}{n} \right)^{\frac{1}{x}}$,$a_{i} > 0$,且 $a_{i} \ne 1, i = 1,2,\cdots,n,n \geq 2$.
	\end{ti}

	\begin{ti}
		设 $\lim_{x \to 0} \frac{\ln\left[ 1 + \frac{f(x)}{\sin x} \right]}{a^{x} - 1} = A (a > 0, a \ne 1)$,求 $\lim_{x \to 0}\frac{f(x)}{x^{2}}$.
	\end{ti}

	\begin{ti}
		已知 $\lim_{x \to 1} f(x)$ 存在,且 $f(x) = \frac{x - \arctan(x - 1) - 1}{(x - 1)^{3}} + 2x^{2} \ee^{x-1} \cdot \lim_{x \to 1} f(x)$,求 $f(x)$.
	\end{ti}

	\begin{ti}
		设函数 $f(x) = (1 + x)^{\frac{1}{x}}(x > 0)$,证明:存在常数 $A,B$,使得当 $x \to 0^{+}$ 时,恒有
		\begin{equation*}
			f(x) = \ee + Ax +Bx^{2} + o\left( x^{2} \right),
		\end{equation*}
		并求常数 $A,B$.
	\end{ti}

	\begin{ti}
		已知 $\lim_{x \to 0} \frac{(1+x)^{\frac{1}{x}} - \left( A + Bx + Cx^{2} \right)}{x^{3}} = D \ne 0$. 求常数 $A,B,C,D$.
	\end{ti}

	\begin{ti}
		设函数 $f(x) = \begin{cases}
			\frac{\ln\left( 1 + x^{3} \right)}{\arcsin x - x}, & x < 0,\\
			\frac{\ee^{-x} + \frac{1}{2}x^{2} + x - 1}{x \sin \frac{x}{6}}, & x > 0,
		\end{cases}$,$g(x) = \frac{\ee^{\frac{1}{x}}\arctan\frac{1}{x}}{1 + \ee^{\frac{2}{x}}}$,求 $\lim_{x \to 0} f[g(x)]$.
	\end{ti}

	\begin{ti}
		设 $\alpha \geq 5$ 且为常数,则 $k$ 为何值时极限
		\begin{equation*}
			I = \lim_{x \to +\infty} \left[ \left( x^{\alpha} + 8x^{4} + 2 \right)^{k} - x \right]
		\end{equation*}
		存在,并求此极限值.
	\end{ti}
	
	\begin{ti}
		已知极限
		\[
			I = \lim_{x \to 0} \left( \frac{a}{x^{2}} + \frac{b}{x^{4}} + \frac{c}{x^{5}} \int_{0}^{x} \ee^{-t^{2}} \dd{t} \right) = 1,
		\]
		求常数 $a,b,c$.
	\end{ti}

	\begin{ti}
		求 $\lim_{x \to 0} \frac{ \sqrt{\cos x} - \sqrt[3]{\cos x} }{\sin^{2}x}$.
	\end{ti}

	\begin{ti}
		求 $\lim_{x \to 1} \frac{\left( 1 - \sqrt[3]{x} \right) \left( 1 - \sqrt[4]{x} \right) \cdots \left( 1 - \sqrt[n]{x} \right) }{(1 - x)^{n-2}}$.
	\end{ti}
	
	\begin{ti}
		求 $\lim_{x \to 0} \frac{1 - \cos x \cdot \sqrt{\cos 2x} \cdot \sqrt[3]{\cos 3x}}{x^{2}}$.
	\end{ti}

	\begin{ti}
		设函数 $f(x)$ 满足 $f(1) = 1$,且有 $f'(x) = \frac{1}{x^{2} + f^{2}(x)}$,证明:极限 $\lim_{x \to \infty} f(x)$ 存在,且极限值小于 $1 + \frac{\uppi}{4}$.
	\end{ti}

	\begin{ti}
		设 $x \geq 0$ 时,$f(x)$ 满足 $f'(x) = \frac{1}{x^{2} + f^{2}(x)}$,且 $f(0) = 1$,证明:$\lim_{x \to +\infty} f(x)$ 存在且极限值小于 $1 + \frac{\uppi}{2}$.
	\end{ti}
	\subsection{无穷小比阶}

	\begin{ti}
		当 $x \to 0$ ,$(1 - \cos x)\ln\left( 1 + 2x^{3} \right)$ 是比 $x \sin x^{n}$ 高阶的无穷小,而 $x \sin x^{n}$ 是比 $\ee^{x\tan^{2} x} - 1$ 高阶的无穷小,则正整数 $n = $ \htwo.
	\end{ti}

	\begin{ti}
		当 $x \to 0^{+}$ 时,$\sqrt{1 + \tan \sqrt{x}} - \sqrt{1 + \sin\sqrt{x}}$ 是 $x$ 的 $k$ 阶无穷小,则 $k =$ \htwo.
	\end{ti}

	\begin{ti}
		当 $x \to 0$ 时,$f(x) = \ln\left( 1+x^{2} \right) - 2\sqrt[3]{\left( \ee^{x} - 1 \right)^{2}}$ 是无穷小量 $x^{k}$ 的同阶无穷小,则 $k = $ \kuo.

		\fourch{$1$}{$2$}{$\frac{2}{3}$}{$\frac{3}{2}$}
	\end{ti}

	\begin{ti}
		当 $x \to 0$ 时,下列无穷小量中,最高阶的无穷小是\kuo.

		\twoch{$\ln\left( x + \sqrt{1 + x^{2}} \right)$}{$1 - \cos x$}{$\tan x - \sin x$}{$\ee^{x} + \ee^{-x} - 2$}
	\end{ti}

	\begin{ti}
		当 $x \to 0^{+}$ 时,下列无穷小量中,与 $x$ 同阶的无穷小是\kuo.

		\twoch{$\sqrt{1 + x} - 1$}{$\ln(1 + x) - x$}{$\cos(\sin x) - 1$}{$x^{x} - 1$}
	\end{ti}

	\begin{ti}
		当 $x \to 0$ 时,$f(x) = x - \sin x + \int_{0}^{x} t^{2} \ee^{t^{2}} \dd{t}$ 是 $x$ 的 $k$ 阶无穷小,则 $k=$ \kuo.

		\fourch{$3$}{$4$}{$5$}{$6$}
	\end{ti}

	\begin{ti}
		当 $x \to 0^{+}$ 时,试比较无穷小量 $\alpha$,$\beta$ 和 $\gamma$ 三者之间的阶,其中
		\[
			\alpha = \int_{0}^{x} \cos t^{2} \dd{t},\beta = \int_{0}^{x^{2}} \tan \sqrt{t} \dd{t},\gamma = \int_{0}^{\sqrt{x}} \sin t^{3} \dd{t}.
		\]
	\end{ti}

	\begin{ti}
		当 $x \to 0$ 时,$\sin x \left( \cos x - 4 \right) + 3x$ 为 $x$ 的几阶无穷小?
	\end{ti}

	\begin{ti}
		当 $x \to 0$ 时,确定下列无穷小量的阶数:
		\begin{enumerate}
			\item $\tan\left( \sqrt{x+2} - \sqrt{2} \right)$;
			\item $\sqrt[3]{1 + \sqrt[3]{x}} - 1$;
			\item $3^{\sqrt{x}} - 1$.  
		\end{enumerate}
	\end{ti}

	\begin{ti}
		当 $x \to 0$ 时,$x - \sin x \cos x \cos 2x$ 与 $cx^{k}$ 为等价无穷小,则 $c=$ \htwo,$k=$ \htwo.
	\end{ti}

	\begin{ti}
		当 $x \to 0$ 时,$1 - \cos x \cos 2x \cos 3x$ 对于无穷小 $x$ 的阶数等于 \htwo.
	\end{ti}

	\begin{ti}
		极限 $\lim_{x \to \infty} \frac{\ee^{\sin\frac{1}{x}}-1}{\left( 1 + \frac{1}{x} \right)^{\alpha} - \left( 1 + \frac{1}{x} \right)} = A \ne 0$ 的充要条件是
		
		\noindent\kuo.

		\twoch{$\alpha > 1$}{$\alpha \ne 1$}{$\alpha > 0$}{与 $\alpha$ 无关}
	\end{ti}

	\begin{ti}
		设当 $x \to 0$ 时,$\ee^{\tan x} - \ee^{x}$ 与 $x^{n}$ 是同阶无穷小,则 $n$ 为 \kuo.

		\fourch{$1$}{$2$}{$3$}{$4$}
	\end{ti}

	\begin{ti}
		设当 $x \to 0$ 时,$f(x) = ax^{3} + bx$ 与 $g(x) =$ $\int_{0}^{\sin x} \left( \ee^{t^{2}} -1 \right) \dd{t}$ 是等价无穷小,则\kuo.

		\twoch{$a = \frac{1}{3},b=1$}{$a = 3,b=0$}{$a = \frac{1}{3},b=0$}{$a = 1,b=0$}
	\end{ti}

	\begin{ti}
		设当 $x \to 0$ 时,$f(x) = \ln\left( 1+x^{2} \right) - \ln\left( 1 + \sin^{2}x \right)$ 是 $x$ 的 $n$ 阶无穷小,则正整数 $n$ 为\kuo.
		
		\fourch{$1$}{$2$}{$3$}{$4$}
	\end{ti}

	\begin{ti}
		当 $x \to \uppi$ 时,若有 $\sqrt[4]{\sin\frac{x}{2}} - 1 \sim A(x - \uppi)^{k}$,则 $A=$\htwo,$k=$\htwo.
	\end{ti}

	\begin{ti}
		半径分别为 $R,r(R>r>0)$ 的两个圆相切于坐标轴原点. 如图~\ref{fig:1.1.1} 所示.
		\begin{enumerate}
			\item 当 $x \to 0^{+}$ 时,若线段长 $MM_{1}$ 与 $x^{k}$ 同阶,求 $k$;
			\item 当 $x \to 0^{+}$ 时,若 $\angle MOM_{1}$ 与 $x^{c}$ 同阶,求 $c$.
		\end{enumerate}
		\begin{figure}[htbp]
			\centering
			\includegraphics[scale=1]{figure/fig1-1-1.pdf}
			\caption{}\label{fig:1.1.1}
		\end{figure}
	\end{ti}
	\subsection{数列极限}

	\begin{ti}
		求 $\lim_{n \to \infty} n^{3} \left( \sin\frac{1}{n} - \frac{1}{2} \sin\frac{2}{n} \right)$.
	\end{ti}

	\begin{ti}
		求 $\lim_{n \to \infty} \left( \sqrt{n + 3\sqrt{n}} - \sqrt{n - \sqrt{n}} \right)$.
	\end{ti}

	\begin{ti}
		求 $\lim_{n \to \infty} \left[ \sqrt{n}\left( \sqrt{n+1} - \sqrt{n} \right) + \frac{1}{2} \right]^{\frac{\sqrt{n+1} + \sqrt{n}}{\sqrt{n+1} - \sqrt{n}}}$.
	\end{ti}

	\begin{ti}
		求 $\lim_{n \to \infty} n^{2} \left( a^{\frac{1}{n}} - a^{\frac{1}{n+1}} \right)$,其中 $a > 0$.
	\end{ti}

	\begin{ti}
		求 $\lim_{n \to \infty} \left( 1 + 2^{n} + 3^{n} \right)^{\frac{1}{n}}$.
	\end{ti}

	\begin{ti}
		求 $\lim_{n \to \infty} \cos\frac{x}{2}\cos\frac{x}{4}\cdots \cos\frac{x}{2^{n}}$.
	\end{ti}

	\begin{ti}
		求 $\lim_{n \to \infty} n^{2}\left( \arctan\frac{a}{n} - \arctan \frac{a}{n+1} \right)$,$a > 0$.
	\end{ti}

	\begin{ti}
		设
		\[
			\lim_{n \to \infty} \frac{n^{99}}{n^{k} - (n-1)^{k}}
		\]
		存在且不为零,则常数 $k =$\htwo.
	\end{ti}

	\begin{ti}
		设数列 $\{ a_{n} \}$ 满足 $\lim_{n \to \infty}\frac{a_{n+1}}{a_{n}} = 1$,则\kuo.

		\twoch{$\{ a_{n} \}$ 有界}{$\{ a_{n} \}$ 不存在极限}{$\{ a_{n} \}$ 自某项起同号}{$\{ a_{n} \}$ 自某项起单调}
	\end{ti}

	\begin{ti}
		设数列 $\{ x_{n} \}$ 满足 $x_{n} > 0$,且 $\lim_{n \to \infty}\frac{x_{n+1}}{x_{n}} = \frac{1}{2}$,则\kuo.

		\onech{$\lim_{n\to\infty}x_{n} = 0$}{$\lim_{n\to\infty}x_{n}$ 存在,但不为零}{$\lim_{n\to\infty}x_{n}$ 不存在}{$\lim_{n\to\infty}x_{n}$ 可能存在,也可能不存在}
	\end{ti}

	\begin{ti}
		已知数列 $\{ a_{n} \}$ 单调,下列结论正确的是\kuo.
		
		\twoch{$\lim_{n \to \infty}\left( \ee^{a_{n}} - 1 \right)$ 存在}{$\lim_{n \to \infty} \frac{1}{1 + a_{n}^{2}}$ 存在}{$\lim_{n \to \infty} \sin a_{n}$ 存在}{$\lim_{n \to \infty} \frac{1}{1 - a_{n}^{2}}$ 存在}
	\end{ti}

	\begin{ti}
		设 $a_{1} = 1$,$a_{2} = 2$,$a_{n+2} = \frac{2a_{n}a_{n+1}}{a_{n} + a_{n+1}} (n=1,2,\cdots)$.
		\begin{enumerate}
			\item 求 $b_{n} = \frac{1}{a_{n+1}} - \frac{1}{a_{n}}$ 的表达式;
			\item 求 $\sum_{k=1}^{n} b_{k}$ 和 $\lim_{n \to \infty} a_{n}$.
		\end{enumerate}
	\end{ti}

	\begin{ti}
		设 $a_{1} = 3$,$a_{n+1} = a_{n}^{2} + a_{n}(n = 1,2,\cdots)$,求极限
		\[
			\lim_{n \to \infty} \left( \frac{1}{1 + a_{1}} + \frac{1}{1 + a_{2}} + \cdots + \frac{1}{1 + a_{n}} \right).
		\]
	\end{ti}
	
	\begin{ti}
		已知 $x_{1} = \frac{1}{2}$,$2 x_{n+1} + x_{n}^{2} = 1$,求 $\lim_{n \to \infty} x_{n}$.
	\end{ti}

	\begin{ti}
		设 $x_{1} = 1$,$x_{n} = 1 + \frac{1}{1 + x_{n-1}}(n = 2,3,\cdots)$. 证明 $\lim_{n \to \infty} x_{n}$ 存在,并求该极限.
	\end{ti}

	\begin{ti}
		设 $x_{1} = 1$,$x_{n+1} = \frac{x_{n} + 3}{x_{n} + 1}$,求 $\lim_{n \to \infty} x_{n}$.
	\end{ti}

	\begin{ti}
		设当 $a \leq x \leq b$ 时,$a \leq f(x) \leq b$,并设存在常数 $k$,$0 \leq k < 1$,对于 $[a,b]$ 上的任意两点 $x_{1}$ 与 $x_{2}$,都有 $|f(x_{1}) - f(x_{2})| \leq k |x_{1} - x_{2}|$. 证明:
		\begin{enumerate}
			\item 存在唯一的 $\xi \in [a,b]$ 使 $f(\xi) = \xi$;
			\item 对于任意给定的 $x_{1} \in [a,b]$,定义 $x_{n+1} = f(x_{n})$,$n = 1,2,\cdots$,则 $\lim_{n \to \infty} x_{n}$ 存在,且 $\lim_{n \to \infty} x_{n} = \xi$.
		\end{enumerate}
	\end{ti}

	\begin{ti}
		已知 $\left( 2 + \sqrt{2} \right)^{n} = A_{n} + B_{n}\sqrt{2}$,$A_{n},B_{n}$ 为整数,$n = 1,2,3,\cdots$,求 $\lim_{n\to \infty} \frac{A_{n}}{B_{n}}$.
	\end{ti}

	\begin{ti}
		设 $f(x)$ 在 $[0,+\infty)$ 上连续,满足 $0 \leq f(x) \leq x, x \in [0,+\infty)$,设 $a_{1} \geq 0$,$a_{n+1} = f(a_{n})(n = 1,2,\cdots)$,证明:
		\begin{enumerate}
			\item $\{ a_{n} \}$ 为收敛数列;
			\item 设 $\lim_{n \to \infty} a_{n} = t$,则有 $f(t) = t$;
			\item 若条件改为 $0 \leq f(x) < x,x \in (0,+\infty)$,则 $t = 0$.
		\end{enumerate}
	\end{ti}

	\begin{ti}
		\begin{enumerate}
			\item 设 $f(x) = x + \ln(2 - x)$,求 $f(x)$ 的最大值;
			\item 设 $x_{1} = \ln 2$,$x_{n} = \sum_{i=1}^{n-1} \ln(2 - x_{i}), n = 2,3,\cdots$,证明 $\lim_{n \to \infty} x_{n}$ 存在并求其极限值.
		\end{enumerate}
	\end{ti}

	\begin{ti}
		设 $x_{1} = 1$,$x_{n} = \int_{0}^{1} \min\{x,x_{n-1}\} \dd{x}, n = 2,3,\cdots$,证明 $\lim_{n \to \infty} x_{n}$ 存在并求其极限值.
	\end{ti}

	\begin{ti}
		设数列 $\{ x_{n} \}$ 满足 $0 < x_{1} < 1$,$\ln(1 + x_{n}) = \ee^{x_{n+1}} - 1(n = 1,2,\cdots)$,证明
		\begin{enumerate}
			\item 当 $0 < x < 1$ 时,$\ln(1 + x) < x < \ee^{x} - 1$;
			\item $\lim_{n \to \infty} x_{n}$ 存在,并求该极限.
		\end{enumerate}
	\end{ti}

	\begin{ti}
		\begin{enumerate}
			\item 证明方程 $x = 2\ln(1 + x)$ 在 $(0,+\infty)$ 内有唯一实根 $\xi$;
			\item 任取 $x_{1} > \xi$,定义 $x_{n+1} = 2\ln(1 + x_{n}), n = 1,2,\cdots$,证明 $\lim_{n \to \infty} x_{n} = \xi$.
		\end{enumerate}
	\end{ti}

	\begin{ti}
		\begin{enumerate}
			\item 证明方程 $\ee^{x} + x^{2n+1} = 0$ 在 $(-1,0)$ 内有唯一实根 $x_{n}, n = 0,1,2,\cdots$;
			\item 证明 $\lim_{n \to \infty} x_{n}$ 存在并求其值 $a$;
			\item 求 $\lim_{n \to \infty} n(x_{n} - a)$.
		\end{enumerate}
	\end{ti}

	\begin{ti}
		设 $F(x,y) = \frac{f(y - x)}{2x}$,$F(1,y) = \frac{y^{2}}{2} - y + 5$,$x_{0} > 0$,$x_{1} = F(x_{0},2x_{0})$,$\cdots$,$x_{n+1} = F(x_{n},2x_{n}), n = 1,2,\cdots$. 证明 $\lim_{n \to \infty} x_{n}$ 存在,并求该极限.
	\end{ti}

	\begin{ti}
		已知
		\[
			f_{n}(x) = \CC_{n}^{1} \cos x - \CC_{n}^{2} \cos^{2}x + \cdots + (-1)^{n-1} \CC_{n}^{n} \cos^{n}x.
		\]
		\begin{enumerate}
			\item 证明方程 $f_{n}(x) = \frac{1}{2}$ 在区间 $\left( 0,\frac{\uppi}{2} \right)$ 中仅有一根 $x_{n}, n = 1,2,3,\cdots$;
			\item 求 $\lim_{n \to \infty} f_{n}\left( \arccos\frac{1}{n} \right)$;
			\item 设 $x_{n} \in \left( 0,\frac{\uppi}{2} \right)$ 满足 $f_{n}(x_{n}) = \frac{1}{2}$,证明 $\lim_{n \to \infty} x_{n} = \frac{\uppi}{2}$.
		\end{enumerate}
	\end{ti}

	\begin{ti}
		\begin{enumerate}
			\item 证明:当 $x \to 0^{+}$ 时,不等式 $0 < \tan^{2}x - x^{2} < x^{4}$ 成立;
			\item 设 $x_{n} = \sum_{k=1}^{n} \tan^{2}\frac{1}{\sqrt{n+k}}$,求 $\lim_{n \to \infty}x_{n}$.
		\end{enumerate}
	\end{ti}

	\begin{ti}
		\begin{enumerate}
			\item 设 $f(x)$ 在 $(0,+\infty)$ 内可导,$f'(x) > 0, x \in (0,+\infty)$,证明 $f(x)$ 在 $(0,+\infty)$ 内单调增加;
			\item 证明 $f(x) = \left( n^{x} + 1 \right)^{-\frac{1}{x}}$ 在 $(0,+\infty)$ 内单调增加,其中 $n$ 为正整数;
			\item 设数列 $x_{n} = \sum_{k=1}^{n} \left( n^{k} + 1 \right)^{-\frac{1}{k}}$,求 $\lim_{n \to \infty} x_{n}$.
		\end{enumerate}
	\end{ti}
	\subsection{连续与间断}

	\begin{ti}
		当 $x \in \left( -\frac{1}{2},1 \right]$ 时,确定函数 $f(x) = \frac{\tan \uppi x}{|x|\left( x^{2} - 1 \right)}$ 的间断点,并判定其类型.
	\end{ti}

	\begin{ti}
		确定函数 $f(x) = \frac{x(x - 1)}{|x| x^{2} - |x|}$ 的间断点,并判定其类型.
	\end{ti}

	\begin{ti}
		设 $a > 0$,$b > 0$,$c > 0$,
		\[
			A(x) = \begin{cases}
				\left( \frac{a^{x} + b^{x}}{2} \right)^{\frac{1}{x}}, & x \ne 0,\\
				c, & x = 0.
			\end{cases}
		\]
		\begin{enumerate}
			\item 讨论 $A(x)$ 在 $x = 0$ 处的连续性;
			\item 讨论 $\lim_{x \to +\infty} A(x)$,$\lim_{x \to -\infty} A(x)$,$\lim_{x \to 0} A(x)$,$A(-1)$,$A(1)$ 五者之间的大小关系.
		\end{enumerate}
	\end{ti}

	\begin{ti}
		求 $f(x) = \frac{1}{1 - \ee^{\frac{x}{1 - x}}}$ 的连续区间、间断点,并判别间断点的类型.
	\end{ti}

	\begin{ti}
		求函数 $f(x) = \lim_{n \to \infty} \frac{x^{n+2} - x^{-n}}{x^{n} + x^{-n}}$ 的间断点并指出其类型.
	\end{ti}

	\begin{ti}
		若
		\[
			f(x) = \frac{\sqrt[3]{x}}{\lambda - \ee^{-kx}}
		\]
		在 $(-\infty,+\infty)$ 内连续,且 $\lim_{x \to -\infty} f(x) = 0$,则\kuo.
		
		\twoch{$\lambda < 0, k < 0$}{$\lambda < 0, k > 0$}{$\lambda \geq 0, k < 0$}{$\lambda \leq 0, k > 0$}
	\end{ti}

	\begin{ti}
		若
		\[
			f(x) = \begin{cases}
				\ee^{x} (\sin x + \cos x), & x > 0,\\
				2x + a, & x \leq 0
			\end{cases}
		\]
		 是 $(-\infty,+\infty)$ 内的连续函数,则 $a =$\htwo.
	\end{ti}

	\begin{ti}
		试讨论函数 $g(x) = \begin{cases}
			x^{\alpha} \sin\frac{1}{x}, & x > 0,\\
			\ee^{x} + \beta, & x \leq 0
		\end{cases}$ 在点 $x = 0$ 处的连续性.
	\end{ti}

	\begin{ti}
		求函数 $F(x) = \begin{cases}
			\frac{x(\uppi + 2x)}{2 \cos x}, & x \leq 0,\\
			\sin\frac{1}{x^{2} - 1}, & x > 0
		\end{cases}$ 的间断点,并判断它们的类型.
	\end{ti}

	\begin{ti}
		设 $f(x) = \lim_{n \to \infty}\frac{\ee^{\frac{1}{x}} \arctan\frac{1}{1 + x}}{x^{2} + \ee^{nx}}$,求 $f(x)$ 的间断点并判定其类型.
	\end{ti}

	\begin{ti}
		设 $f(x) = \begin{cases}
			\ee^{\frac{1}{x - 1}}, & x > 0,\\
			\ln(1 + x), & -1 < x < 0,
		\end{cases}$ 求 $f(x)$ 的间断点,并说明间断点的类型.
	\end{ti}

	\begin{ti}
		设 $f(x;t) = \left( \frac{x - 1}{t - 1} \right)^{\frac{t}{x - t}}((x - 1)(t - 1)>0, x \ne t)$,函数 $f(x)$ 由表达式
		\[
			f(x) = \lim_{t \to x}f(x;t)
		\]
		确定,求 $f(x)$ 的连续区间和间断点,并判定间断点的类型.
	\end{ti}

	\begin{ti}
		设函数 $f(x)$ 在 $[a,b]$ 上连续,$x_{1},x_{2},\cdots,x_{n},\cdots$ 是 $[a,b]$ 上的一个点列,求 $\lim_{n \to \infty} \sqrt[n]{\frac{1}{n}\sum_{k=1}^{n}\ee^{f(x_{k})}}$.
	\end{ti}

	\begin{ti}
		\begin{enumerate}
			\item 求函数 $f(x) = \lim_{n \to \infty} \sqrt[n]{1 + (2x)^{n} + x^{2n}}(x \geq 0)$ 的表达式;
			\item 讨论函数 $f(x)$ 的连续性.
		\end{enumerate}
	\end{ti}

	\begin{ti}
		已知 $f(x) = \lim_{n \to \infty} \frac{x^{2n-1} + ax^{2} + bx}{x^{2n} + 1}$ 是连续函数,求 $a,b$ 的值.
	\end{ti}

	\begin{ti}
		求函数 $f(x) = \frac{x^{3} + 1}{|x + 1|\left( x^{2} - x \right)} \sin\left( \frac{|x - 1|}{x + 2}\uppi \right)$ 的所有间断点,并判断它们的类型.
	\end{ti}
	
	\subsection{变限积分}
	\paragraph{1. 直接求导}

	\begin{ti}
		设 $f(x)$ 连续,$f(0) = 1$,则曲线 $y = \int_{0}^{x} f(t) \dd{t}$ 在 $(0,0)$ 处的切线方程是\hone{6}.
	\end{ti}

	\begin{ti}
		函数 $F(x) = \int_{1}^{x} \bigl( 1 - \ln \sqrt{t} \bigr) \dd{t} (x > 0)$ 的递减区间为\hone{6}.
	\end{ti}

	\begin{ti}
		设 $f(x)$ 是连续函数,且 $\int_{0}^{x^{3} - 1} f(t) \dd{t} = x$,则 $f(7) = $\hone{4}.
	\end{ti}

	\begin{ti}
		设 $f(x)$ 为连续函数,且 $F(x) = \int_{\frac{1}{x}}^{\ln x} f(t) \dd{t}$,则 $F'(x) = $\hone{6}.
	\end{ti}

	\begin{ti}
		$\frac{\dd}{\dd{x}} \bigl[ \int_{0}^{x} \sin (x - t)^{2} \dd{t} \bigr] = $\hone{6}.
	\end{ti}
\end{document}