\documentclass[openany,twocolumn]{ctexbook}
\usepackage{amsmath,amssymb,mathtools,upgreek}
\usepackage{fourier}
\usepackage{esint}
\usepackage{bm}
\makeatletter
\def\vdots@i#1#2#3{\vbox{
  #1\baselineskip#2\p@ \lineskiplimit\z@
  \kern#3\p@\hbox{.}\hbox{.}\hbox{.}}}
\DeclareRobustCommand\vdots{
  \mathchoice
    {\vdots@i{}{4}{6}}
    {\vdots@i{}{4}{6}}
    {\vdots@i{\scriptsize}{2}{1}}
    {\vdots@i{\tiny}{2}{1}}
}
\makeatother
\usepackage{physics}
\usepackage{siunitx}
\usepackage{lastpage}
\usepackage{graphicx}
\numberwithin{figure}{section}
\renewcommand\thefigure{\arabic{chapter}-\arabic{section}-\arabic{figure}}
\usepackage{floatrow}
\usepackage{subcaption}
% \renewcommand\thesubfigure{(\alph{subfigure})}
% \captionsetup[sub]{labelformat=simple}
\xeCJKDeclareCharClass{FullRight}{"2236}
\newcommand\ratio[2]{#1^^^^2236#2}

\usepackage[a4paper,top=2.5cm,bottom=2.5cm,inner=1.5cm,outer=3cm]{geometry}
% \usepackage[toc]{multitoc}

\usepackage{tikz}
\usetikzlibrary{shapes.geometric,calc}
\newcommand*{\circled}[1]{\lower.7ex\hbox{\tikz\draw (0pt, 0pt)%
	circle (.5em) node {\makebox[1em][c]{\small #1}};}}
\let\libcirc\circled

\makeatletter
\newcommand{\rmnum}[1]{\romannumeral #1}
\newcommand{\Rmnum}[1]{\expandafter\@slowromancap\romannumeral #1@}
\makeatother

\newcommand\score[2]{
\pgfmathsetmacro\pgfxa{#1+1}
\tikzstyle{scorestars}=[star, star points=5, star point ratio=2.25, draw,inner sep=0.15em,anchor=outer point 3]
\begin{tikzpicture}[baseline]
  \foreach \i in {1,...,#2} {
	\pgfmathparse{(\i<=#1?"black":"white")}
	\edef\starcolor{\pgfmathresult}
	\draw (\i*1em,0) node[name=star\i,scorestars,fill=\starcolor]  {};
   }
   \pgfmathparse{(#1>int(#1)?int(#1+1):0}
   \let\partstar=\pgfmathresult
   \ifnum\partstar>0
	 \pgfmathsetmacro\starpart{#1-(int(#1))}
	 \path [clip] ($(star\partstar.outer point 3)!(star\partstar.outer point 2)!(star\partstar.outer point 4)$) rectangle 
	($(star\partstar.outer point 2 |- star\partstar.outer point 1)!\starpart!(star\partstar.outer point 1 -| star\partstar.outer point 5)$);
	 \fill (\partstar*1em,0) node[scorestars,fill=black]  {};
   \fi
,\end{tikzpicture}
}

\usepackage{wallpaper}
\renewcommand{\CenterWallPaper}[2]{%
\AddToShipoutPicture{\put(\LenToUnit{\wpXoffset},\LenToUnit{\wpYoffset}){%
	 \parbox[b][\paperheight]{\paperwidth}{%
		\vfill
		\centering
		\tikz[opacity=0.075] \node[inner sep=0pt] {\includegraphics[angle=90,width=#1\paperwidth,height=#1\paperheight,keepaspectratio]{#2}};%
		\vfill
	 }}
  }
}
% \CenterWallPaper{1}{figure/ctanlion.pdf}
% \CenterWallPaper{1}{figure/wallpaper.pdf}

\usepackage{tabularx,diagbox}

\setlength{\headheight}{13pt}
\makeatletter
\usepackage{fancyhdr}
\pagestyle{fancy}
\fancyhf{}
\fancyhead[LO]{\bfseries \rightmark}
\fancyhead[RE]{\bfseries \leftmark}
% \fancyhead[LE,RO]{
% \@ifundefined{lastpage@lastpage}{%
% 	\score{0}{10}%
% }{%
% 	\score{10 * \thepage / \lastpage@lastpage}{10}%
% }%
% }
\fancyhead[C]{\bfseries \href{https://github.com/sikouhjw/zhangyu1000}{仅供学习使用,严禁商业使用}}
\fancyfoot[C]{\zihao{-5} {\kaishu 不论一个人的数学水平有多高,只要对数学拥有一颗真诚的心,他就在自己的心灵上得到了升华。}---{\itshape SCIbird}}
\fancyhead[LE,RO]{\bfseries --\thepage/\pageref{LastPage}--}
\makeatother

\usepackage{caption}
\captionsetup{labelsep=space}


\usepackage{theorem}
\ctexset{
	chapter={
		name={},
		number=0\arabic{chapter},
	},
	section={
		format={\zihao{4}\bfseries\centering},
		name={第,章},
		aftername={\hspace{1em}},
		number=\chinese{section},
	},
	subsection={
		format={\zihao{-4}\bfseries\raggedright},
		name={,、},
		aftername={\hspace{0bp}},
		number=\chinese{subsection},
	},
	subsubsection={
		format={\zihao{-4}\bfseries\raggedright},
		name={},
		aftername={\hspace{5bp}},
		number={\arabic{section}.\arabic{subsection}.\arabic{subsubsection}},
	},
}

{
	\theoremstyle{change}
	\theoremheaderfont{\bfseries}
	\theorembodyfont{\normalfont}
	\newtheorem{ti}{}[section]
}
\renewcommand{\theti}{\arabic{section}.\arabic{ti}}

{
	\theoremstyle{change}
	\theoremheaderfont{\bfseries}
	\theorembodyfont{\normalfont}
	\newtheorem{titwo}{}[chapter]
}
\renewcommand{\thetitwo}{\arabic{titwo}.}

\usepackage{ulem}
\newcommand{\hone}[1]{ \uline{\hspace{#1 pc}}}
\def\htwo{\CJKunderline*[hidden = true]{瞻彼阕者虚室生白}}
\def\kuo{ \mbox{(\hspace{1pc})}}

\newcommand{\fourch}[4]{\noindent\begin{tabular}{*{4}{@{}p{1.97cm}}}(\texttt{A})~#1 & (\texttt{B})~#2 & (\texttt{C})~#3 & (\texttt{D})~#4\end{tabular}} % 一行
\newcommand{\twoch}[4]{\noindent\begin{tabular}{*{2}{@{}p{3.94cm}}}(\texttt{A})~#1 & (\texttt{B})~#2\end{tabular}\\\begin{tabular}{*{2}{@{}p{3.94cm}}}(\texttt{C})~#3 & (\texttt{D})~#4\end{tabular}}  %两行
\newcommand{\onech}[4]{\noindent(\texttt{A})~#1 \\ (\texttt{B})~#2 \\ (\texttt{C})~#3 \\ (\texttt{D})~#4}  % 四行

\def\leq{\leqslant}
\def\geq{\geqslant}
\def\ee{\mathrm{e}}
\def\CC{\mathrm{C}}
\def\TT{\mathrm{T}}
\def\AA{\mathrm{A}}
\def\astt{*}
\edef\lim{\lim\limits}
\edef\sum{\sum\limits}
\let\div\relax
\DeclareMathOperator{\div}{div}
\let\grad\relax
\DeclareMathOperator{\grad}{grad}
\DeclareMathOperator{\rot}{rot}
\DeclareMathOperator{\Cov}{Cov}
\long\def\guanggao{
	\vspace*{\fill}
	\begin{center}
		\bfseries 广告位招租
	\end{center}
	\vspace*{\fill}
}
\def\theenumi{\arabic{enumi}}
\def\labelenumi{(\theenumi)}

% \setCJKmainfont{SourceHanSerifCN}[
% UprightFont    = *-Regular,
% BoldFont       = *-Bold,
% ItalicFont     = *-Regular,
% BoldItalicFont = *-Bold
% ]
% \setCJKfamilyfont{kaishu}{FZXKTJW.ttf}
% \def\kaishu{\CJKfamily{kaishu}}

\usepackage[bookmarksopen=true,bookmarksnumbered=true,hidelinks]{hyperref}

\title{\href{https://github.com/sikouhjw/zhangyu1000}{张宇考研数学题源探析经典 1000 题\\(习题分册·数学一)}}
\author{张宇}
\date{2019 年 3 月}


\begin{document}
	% \frontmatter
	% \maketitle
	% 	\onecolumn
	\chapter{前言}

	按照考研数学历年的命题规律和风格,结合最新的信息,考生在2020年考研复习备考中应做到以下五点:一是将考研基础知识和常规题目作为复习主体;二是要加强综合性试题的训练;三是加强计算能力的培养,使自己具备较强的处理数学计算过程的本领,要知道,绝大多数数学题都是要通过准确的计算才能得到正确答案的;四是要加强应用能力的培养,多做用数学基础知识解决实际问题的题目;五是要全面复习,将考研大纲中的所有知识作为复习范围,不要有所偏颇。以上五点也将是2020考研命题的趋势,请各位考生重视。

	本书是对题源的最新研究成果,它尽力搜集和命制了题源本身或与题源相关的重要考题,值得考生在复习全过程中认真做题、消化。我也将在各种场合对本书的题目进行详细讲解并予以重点提示,以期让考生把握住考试命题方向,准确复习备考。题源和题库研究是公共资源,从2019年考研命题的情况来看,它并不回避市面上已经公开的题源,甚至可考到原题,于是,我很高兴把我们所掌握的信息提供给全国考生,并乐于与大家分享这些资料。这对消除考研数学的神秘感,进一步促进考试的公正性与科学性都会起到重要作用。

	这本《张宇考研数学题源探析经典1000题》最初是按照数学一、数学二、数学三平均1000道题左右来命名的,多年来一直这样叫下来,成为了考研习题集的一个经典名称。事实上,数学一考试内容最多,题目不止1000道,数学二考试内容最少,题目少于1000道,数学三的考试内容居中,近于1000道。

	衷心感谢原命题专家们给予的指导与帮助。希望考生认真研读、操练本书中的每一道题目,提高解题能力,争取考研得到高分。

	\phantom{1}\hspace{\fill} {\LARGE 张宇}
	
	\phantom{1}\hspace{\fill} {2019 年 2 月\quad 于北京}
	\twocolumn
	% \tableofcontents
	% \mainmatter
	% \chapter{高等数学}
	高等数学是硕士研究生招生考试考查内容之一,主要考查考生对高等数学的基本概念、基本理论、基本方法的理解和掌握以及考生的抽象思维能力、逻辑推理能力、综合运用能力和解决实际问题的能力。在考研数学一试卷中分值为82分,约占56\%。
	\newpage
	\section{极限、连续}
	\subsection{函数极限}
	\begin{ti}
		求 $\lim_{x\to 0} \frac{\sqrt{1+x} - 1 - \frac{x}{2}}{\ee^{x^{2}}-1}$.
	\end{ti}

	\begin{ti}
		求 $\lim_{x \to 0} \frac{\ee^{x} + \ln(1 - x) - 1}{x - \arctan x}$.
	\end{ti}

	\begin{ti}
		求 $\lim_{x \to 0} \frac{(1+x)^{\frac{2}{x}} - \ee^{2}[1 - \ln(1+x)]}{x}$.
	\end{ti}

	\begin{ti}
		求 $\lim_{x \to 0} \frac{\left(1 + x^{2}\right)(1 - \cos 2x) - 2x^{2}}{x^{4}}$.
	\end{ti}

	\begin{ti}
		求 $\lim_{x \to 0} \frac{\sqrt{1-x^{2}} \sin^{2}x - \tan^{2}x }{x^{2}[\ln(1+x)]^{2}}$.
	\end{ti}

	\begin{ti}
		求 $\lim_{x \to 0} \frac{(3 + 2 \tan x)^{x} - 3^{x}}{3 \sin^{2}x + x^{3} \cos\frac{1}{x}}$.
	\end{ti}

	\begin{ti}
		求 $\lim_{x \to 2}\frac{\sqrt{5x - 1} - \sqrt{2x + 5}}{x^{2} - 4}$.
	\end{ti}

	\begin{ti}
		求 $\lim_{x \to 0}\int_{0}^{x} \frac{\sin 2t}{\sqrt{4+t^{2}}\int_{0}^{x} \left(\sqrt{t+1} - 1\right)\dd{t}} \dd{t}$.
	\end{ti}

	\begin{ti}
		求 $\lim_{x \to \infty} \ee^{-x} \left( 1 + \frac{1}{x} \right)^{x^{2}}$.
	\end{ti}

	\begin{ti}
		求 $\lim_{x \to 3^{+}} \frac{\cos x \ln (x - 3)}{\ln\left( \ee^{x} - \ee^{3} \right)}$.
	\end{ti}

	\begin{ti}
		求 $\lim_{x \to \infty} x^{2} \left( a^{\frac{1}{x}} + a^{-\frac{1}{x}} - 2 \right)$,其中常数 $a > 0$.
	\end{ti}

	\begin{ti}
		求 $\lim_{x \to 0}\frac{1}{x}\left( \cot x - \frac{1}{x} \right)$.
	\end{ti}

	\begin{ti}
		求 $\lim_{x \to +\infty}\left( \sqrt[3]{x^{3} + 2x^{2} + 1} - x\ee^{\frac{1}{x}} \right)$.
	\end{ti}

	\begin{ti}
		求 $\lim_{x \to 0}\left( \frac{1+x}{1-e^{-x}} - \frac{1}{x} \right)$.
	\end{ti}

	\begin{ti}
		求 $\lim_{x \to 0^{+}} x^{\ln\left( \frac{\ln x - 1}{\ln x + 1} \right)}$.
	\end{ti}

	\begin{ti}
		求 $\lim_{x \to \infty} \left( \tan\frac{\uppi x}{1 + 2x} \right)^{\frac{1}{x}}$.
	\end{ti}

	\begin{ti}
		求 $\lim_{x \to 0^{+}} \left( \frac{\sin x}{x} \right)^{\frac{1}{1 - \cos x}}$.
	\end{ti}

	\begin{ti}
		求 $\lim_{x \to 0}\left( \frac{\cos x}{\cos 2x} \right)^{\frac{1}{x^{2}}}$.
	\end{ti}

	\begin{ti}
		求 $\lim_{x \to 0} \frac{\sin x - x\cos x}{x - \sin x}$.
	\end{ti}

	\begin{ti}
		求 $\lim_{x \to 0}\frac{1 + \frac{1}{2}x^{2} - \sqrt{1 + x^{2}}}{\left( \cos x - \ee^{\frac{x^{2}}{2}} \right) \sin \frac{x^{2}}{2}}$.
	\end{ti}

	\begin{ti}
		求 $\lim_{x \to \infty} \left( \sqrt[6]{x^{6} + x^{5}} - \sqrt[6]{x^{6} - x^{5}} \right)$.
	\end{ti}

	\begin{ti}
		求 $\lim_{x \to +\infty}\left[ \left( x^{3} + \frac{x}{2} - \tan \frac{1}{x} \right) \ee^{\frac{1}{x}} - \sqrt{1 + x^{6}} \right]$.
	\end{ti}

	\begin{ti}
		求 $\lim_{x \to 0}\frac{\ee^{\tan x} - \ee^{\sin x}}{x \sin^{2} x}$.
	\end{ti}

	\begin{ti}
		求 $\lim_{x \to 0} \frac{\sin x + x^{2} \sin\frac{1}{x}}{(1 + \cos x)\ln(1 + x)}$.
	\end{ti}

	\begin{ti}
		求 $\lim_{x \to 0}\left[ \frac{a}{x} - \left( \frac{1}{x^{2}} - a^{2} \right) \ln(1 + ax) \right]$,其中 $a \ne 0$.
	\end{ti}

	\begin{ti}
		求 $\lim_{x \to 0} \frac{(1 + x)^{\frac{1}{x}} - (1 + 2x)^{\frac{1}{2x}}}{\sin x}$.
	\end{ti}

	\begin{ti}
		求 $\lim_{x \to 0}\frac{\int_{0}^{\sin^{2}x} \ln(1 + t)\dd{t}}{\left( \sqrt[3]{1 + x^{3}} - 1 \right)\sin x}$.
	\end{ti}

	\begin{ti}
		求 $\lim_{x \to 0} \frac{ \int_{0}^{x} \left[ \int_{0}^{u^{2}} \arctan(1 + t) \dd{t} \right] \dd{u} }{x(1 - \cos x)}$.
	\end{ti}

	\begin{ti}
		求 $\lim_{x \to 0^{+}} \frac{x^{x} - ( \sin x )^{x}}{x^{2}\ln(1 + x)}$.
	\end{ti}

	\begin{ti}
		求 $\lim_{x \to 0} \frac{ \cos x - \ee^{-\frac{x^{2}}{2}} }{x^{2} [ x + \ln(1 - x) ]}$.
	\end{ti}

	\begin{ti}
		求 $\lim_{x \to 0} \frac{1}{x^{3}} \left[ \left( \frac{2 + \cos x}{3} \right)^{x} - 1 \right]$.
	\end{ti}

	\begin{ti}
		求 $\lim_{x \to 0} \frac{\ln\left( \sin^{2}x + \ee^{x} \right) - x}{\ln\left( x^{2} + \ee^{2x} \right) - 2x}$.
	\end{ti}

	\begin{ti}
		求 $\lim_{x \to 1} \frac{x - x^{x}}{1 - x + \ln x}$.
	\end{ti}

	\begin{ti}
		求 $\lim_{x \to 0} \left( \frac{a_{1}^{x} + a_{2}^{x} + \cdots + a_{n}^{x}}{n} \right)^{\frac{1}{x}}$,$a_{i} > 0$,且 $a_{i} \ne 1, i = 1,2,\cdots,n,n \geq 2$.
	\end{ti}

	\begin{ti}
		设 $\lim_{x \to 0} \frac{\ln\left[ 1 + \frac{f(x)}{\sin x} \right]}{a^{x} - 1} = A (a > 0, a \ne 1)$,求 $\lim_{x \to 0}\frac{f(x)}{x^{2}}$.
	\end{ti}

	\begin{ti}
		已知 $\lim_{x \to 1} f(x)$ 存在,且 $f(x) = \frac{x - \arctan(x - 1) - 1}{(x - 1)^{3}} + 2x^{2} \ee^{x-1} \cdot \lim_{x \to 1} f(x)$,求 $f(x)$.
	\end{ti}

	\begin{ti}
		设函数 $f(x) = (1 + x)^{\frac{1}{x}}(x > 0)$,证明:存在常数 $A,B$,使得当 $x \to 0^{+}$ 时,恒有
		\begin{equation*}
			f(x) = \ee + Ax +Bx^{2} + o\left( x^{2} \right),
		\end{equation*}
		并求常数 $A,B$.
	\end{ti}

	\begin{ti}
		已知 $\lim_{x \to 0} \frac{(1+x)^{\frac{1}{x}} - \left( A + Bx + Cx^{2} \right)}{x^{3}} = D \ne 0$. 求常数 $A,B,C,D$.
	\end{ti}

	\begin{ti}
		设函数 $f(x) = \begin{cases}
			\frac{\ln\left( 1 + x^{3} \right)}{\arcsin x - x}, & x < 0,\\
			\frac{\ee^{-x} + \frac{1}{2}x^{2} + x - 1}{x \sin \frac{x}{6}}, & x > 0,
		\end{cases}$,$g(x) = \frac{\ee^{\frac{1}{x}}\arctan\frac{1}{x}}{1 + \ee^{\frac{2}{x}}}$,求 $\lim_{x \to 0} f[g(x)]$.
	\end{ti}

	\begin{ti}
		设 $\alpha \geq 5$ 且为常数,则 $k$ 为何值时极限
		\begin{equation*}
			I = \lim_{x \to +\infty} \left[ \left( x^{\alpha} + 8x^{4} + 2 \right)^{k} - x \right]
		\end{equation*}
		存在,并求此极限值.
	\end{ti}
	
	\begin{ti}
		已知极限
		\[
			I = \lim_{x \to 0} \left( \frac{a}{x^{2}} + \frac{b}{x^{4}} + \frac{c}{x^{5}} \int_{0}^{x} \ee^{-t^{2}} \dd{t} \right) = 1,
		\]
		求常数 $a,b,c$.
	\end{ti}

	\begin{ti}
		求 $\lim_{x \to 0} \frac{ \sqrt{\cos x} - \sqrt[3]{\cos x} }{\sin^{2}x}$.
	\end{ti}

	\begin{ti}
		求 $\lim_{x \to 1} \frac{\left( 1 - \sqrt[3]{x} \right) \left( 1 - \sqrt[4]{x} \right) \cdots \left( 1 - \sqrt[n]{x} \right) }{(1 - x)^{n-2}}$.
	\end{ti}
	
	\begin{ti}
		求 $\lim_{x \to 0} \frac{1 - \cos x \cdot \sqrt{\cos 2x} \cdot \sqrt[3]{\cos 3x}}{x^{2}}$.
	\end{ti}

	\begin{ti}
		设函数 $f(x)$ 满足 $f(1) = 1$,且有 $f'(x) = \frac{1}{x^{2} + f^{2}(x)}$,证明:极限 $\lim_{x \to \infty} f(x)$ 存在,且极限值小于 $1 + \frac{\uppi}{4}$.
	\end{ti}

	\begin{ti}
		设 $x \geq 0$ 时,$f(x)$ 满足 $f'(x) = \frac{1}{x^{2} + f^{2}(x)}$,且 $f(0) = 1$,证明:$\lim_{x \to +\infty} f(x)$ 存在且极限值小于 $1 + \frac{\uppi}{2}$.
	\end{ti}
	\subsection{无穷小比阶}

	\begin{ti}
		当 $x \to 0$ ,$(1 - \cos x)\ln\left( 1 + 2x^{3} \right)$ 是比 $x \sin x^{n}$ 高阶的无穷小,而 $x \sin x^{n}$ 是比 $\ee^{x\tan^{2} x} - 1$ 高阶的无穷小,则正整数 $n = $ \htwo.
	\end{ti}

	\begin{ti}
		当 $x \to 0^{+}$ 时,$\sqrt{1 + \tan \sqrt{x}} - \sqrt{1 + \sin\sqrt{x}}$ 是 $x$ 的 $k$ 阶无穷小,则 $k =$ \htwo.
	\end{ti}

	\begin{ti}
		当 $x \to 0$ 时,$f(x) = \ln\left( 1+x^{2} \right) - 2\sqrt[3]{\left( \ee^{x} - 1 \right)^{2}}$ 是无穷小量 $x^{k}$ 的同阶无穷小,则 $k = $ \kuo.

		\fourch{$1$}{$2$}{$\frac{2}{3}$}{$\frac{3}{2}$}
	\end{ti}

	\begin{ti}
		当 $x \to 0$ 时,下列无穷小量中,最高阶的无穷小是\kuo.

		\twoch{$\ln\left( x + \sqrt{1 + x^{2}} \right)$}{$1 - \cos x$}{$\tan x - \sin x$}{$\ee^{x} + \ee^{-x} - 2$}
	\end{ti}

	\begin{ti}
		当 $x \to 0^{+}$ 时,下列无穷小量中,与 $x$ 同阶的无穷小是\kuo.

		\twoch{$\sqrt{1 + x} - 1$}{$\ln(1 + x) - x$}{$\cos(\sin x) - 1$}{$x^{x} - 1$}
	\end{ti}

	\begin{ti}
		当 $x \to 0$ 时,$f(x) = x - \sin x + \int_{0}^{x} t^{2} \ee^{t^{2}} \dd{t}$ 是 $x$ 的 $k$ 阶无穷小,则 $k=$ \kuo.

		\fourch{$3$}{$4$}{$5$}{$6$}
	\end{ti}

	\begin{ti}
		当 $x \to 0^{+}$ 时,试比较无穷小量 $\alpha$,$\beta$ 和 $\gamma$ 三者之间的阶,其中
		\[
			\alpha = \int_{0}^{x} \cos t^{2} \dd{t},\beta = \int_{0}^{x^{2}} \tan \sqrt{t} \dd{t},\gamma = \int_{0}^{\sqrt{x}} \sin t^{3} \dd{t}.
		\]
	\end{ti}

	\begin{ti}
		当 $x \to 0$ 时,$\sin x \left( \cos x - 4 \right) + 3x$ 为 $x$ 的几阶无穷小?
	\end{ti}

	\begin{ti}
		当 $x \to 0$ 时,确定下列无穷小量的阶数:
		\begin{enumerate}
			\item $\tan\left( \sqrt{x+2} - \sqrt{2} \right)$;
			\item $\sqrt[3]{1 + \sqrt[3]{x}} - 1$;
			\item $3^{\sqrt{x}} - 1$.  
		\end{enumerate}
	\end{ti}

	\begin{ti}
		当 $x \to 0$ 时,$x - \sin x \cos x \cos 2x$ 与 $cx^{k}$ 为等价无穷小,则 $c=$ \htwo,$k=$ \htwo.
	\end{ti}

	\begin{ti}
		当 $x \to 0$ 时,$1 - \cos x \cos 2x \cos 3x$ 对于无穷小 $x$ 的阶数等于 \htwo.
	\end{ti}

	\begin{ti}
		极限 $\lim_{x \to \infty} \frac{\ee^{\sin\frac{1}{x}}-1}{\left( 1 + \frac{1}{x} \right)^{\alpha} - \left( 1 + \frac{1}{x} \right)} = A \ne 0$ 的充要条件是
		
		\noindent\kuo.

		\twoch{$\alpha > 1$}{$\alpha \ne 1$}{$\alpha > 0$}{与 $\alpha$ 无关}
	\end{ti}

	\begin{ti}
		设当 $x \to 0$ 时,$\ee^{\tan x} - \ee^{x}$ 与 $x^{n}$ 是同阶无穷小,则 $n$ 为 \kuo.

		\fourch{$1$}{$2$}{$3$}{$4$}
	\end{ti}

	\begin{ti}
		设当 $x \to 0$ 时,$f(x) = ax^{3} + bx$ 与 $g(x) =$ $\int_{0}^{\sin x} \left( \ee^{t^{2}} -1 \right) \dd{t}$ 是等价无穷小,则\kuo.

		\twoch{$a = \frac{1}{3},b=1$}{$a = 3,b=0$}{$a = \frac{1}{3},b=0$}{$a = 1,b=0$}
	\end{ti}

	\begin{ti}
		设当 $x \to 0$ 时,$f(x) = \ln\left( 1+x^{2} \right) - \ln\left( 1 + \sin^{2}x \right)$ 是 $x$ 的 $n$ 阶无穷小,则正整数 $n$ 为\kuo.
		
		\fourch{$1$}{$2$}{$3$}{$4$}
	\end{ti}

	\begin{ti}
		当 $x \to \uppi$ 时,若有 $\sqrt[4]{\sin\frac{x}{2}} - 1 \sim A(x - \uppi)^{k}$,则 $A=$\htwo,$k=$\htwo.
	\end{ti}

	\begin{ti}
		半径分别为 $R,r(R>r>0)$ 的两个圆相切于坐标轴原点. 如图~\ref{fig:1.1.1} 所示.
		\begin{enumerate}
			\item 当 $x \to 0^{+}$ 时,若线段长 $MM_{1}$ 与 $x^{k}$ 同阶,求 $k$;
			\item 当 $x \to 0^{+}$ 时,若 $\angle MOM_{1}$ 与 $x^{c}$ 同阶,求 $c$.
		\end{enumerate}
		\begin{figure}[htbp]
			\centering
			\includegraphics[scale=1]{figure/fig1-1-1.pdf}
			\caption{}\label{fig:1.1.1}
		\end{figure}
	\end{ti}
	\subsection{数列极限}

	\begin{ti}
		求 $\lim_{n \to \infty} n^{3} \left( \sin\frac{1}{n} - \frac{1}{2} \sin\frac{2}{n} \right)$.
	\end{ti}

	\begin{ti}
		求 $\lim_{n \to \infty} \left( \sqrt{n + 3\sqrt{n}} - \sqrt{n - \sqrt{n}} \right)$.
	\end{ti}

	\begin{ti}
		求 $\lim_{n \to \infty} \left[ \sqrt{n}\left( \sqrt{n+1} - \sqrt{n} \right) + \frac{1}{2} \right]^{\frac{\sqrt{n+1} + \sqrt{n}}{\sqrt{n+1} - \sqrt{n}}}$.
	\end{ti}

	\begin{ti}
		求 $\lim_{n \to \infty} n^{2} \left( a^{\frac{1}{n}} - a^{\frac{1}{n+1}} \right)$,其中 $a > 0$.
	\end{ti}

	\begin{ti}
		求 $\lim_{n \to \infty} \left( 1 + 2^{n} + 3^{n} \right)^{\frac{1}{n}}$.
	\end{ti}

	\begin{ti}
		求 $\lim_{n \to \infty} \cos\frac{x}{2}\cos\frac{x}{4}\cdots \cos\frac{x}{2^{n}}$.
	\end{ti}

	\begin{ti}
		求 $\lim_{n \to \infty} n^{2}\left( \arctan\frac{a}{n} - \arctan \frac{a}{n+1} \right)$,$a > 0$.
	\end{ti}

	\begin{ti}
		设
		\[
			\lim_{n \to \infty} \frac{n^{99}}{n^{k} - (n-1)^{k}}
		\]
		存在且不为零,则常数 $k =$\htwo.
	\end{ti}

	\begin{ti}
		设数列 $\{ a_{n} \}$ 满足 $\lim_{n \to \infty}\frac{a_{n+1}}{a_{n}} = 1$,则\kuo.

		\twoch{$\{ a_{n} \}$ 有界}{$\{ a_{n} \}$ 不存在极限}{$\{ a_{n} \}$ 自某项起同号}{$\{ a_{n} \}$ 自某项起单调}
	\end{ti}

	\begin{ti}
		设数列 $\{ x_{n} \}$ 满足 $x_{n} > 0$,且 $\lim_{n \to \infty}\frac{x_{n+1}}{x_{n}} = \frac{1}{2}$,则\kuo.

		\onech{$\lim_{n\to\infty}x_{n} = 0$}{$\lim_{n\to\infty}x_{n}$ 存在,但不为零}{$\lim_{n\to\infty}x_{n}$ 不存在}{$\lim_{n\to\infty}x_{n}$ 可能存在,也可能不存在}
	\end{ti}

	\begin{ti}
		已知数列 $\{ a_{n} \}$ 单调,下列结论正确的是\kuo.
		
		\twoch{$\lim_{n \to \infty}\left( \ee^{a_{n}} - 1 \right)$ 存在}{$\lim_{n \to \infty} \frac{1}{1 + a_{n}^{2}}$ 存在}{$\lim_{n \to \infty} \sin a_{n}$ 存在}{$\lim_{n \to \infty} \frac{1}{1 - a_{n}^{2}}$ 存在}
	\end{ti}

	\begin{ti}
		设 $a_{1} = 1$,$a_{2} = 2$,$a_{n+2} = \frac{2a_{n}a_{n+1}}{a_{n} + a_{n+1}} (n=1,2,\cdots)$.
		\begin{enumerate}
			\item 求 $b_{n} = \frac{1}{a_{n+1}} - \frac{1}{a_{n}}$ 的表达式;
			\item 求 $\sum_{k=1}^{n} b_{k}$ 和 $\lim_{n \to \infty} a_{n}$.
		\end{enumerate}
	\end{ti}

	\begin{ti}
		设 $a_{1} = 3$,$a_{n+1} = a_{n}^{2} + a_{n}(n = 1,2,\cdots)$,求极限
		\[
			\lim_{n \to \infty} \left( \frac{1}{1 + a_{1}} + \frac{1}{1 + a_{2}} + \cdots + \frac{1}{1 + a_{n}} \right).
		\]
	\end{ti}
	
	\begin{ti}
		已知 $x_{1} = \frac{1}{2}$,$2 x_{n+1} + x_{n}^{2} = 1$,求 $\lim_{n \to \infty} x_{n}$.
	\end{ti}

	\begin{ti}
		设 $x_{1} = 1$,$x_{n} = 1 + \frac{1}{1 + x_{n-1}}(n = 2,3,\cdots)$. 证明 $\lim_{n \to \infty} x_{n}$ 存在,并求该极限.
	\end{ti}

	\begin{ti}
		设 $x_{1} = 1$,$x_{n+1} = \frac{x_{n} + 3}{x_{n} + 1}$,求 $\lim_{n \to \infty} x_{n}$.
	\end{ti}

	\begin{ti}
		设当 $a \leq x \leq b$ 时,$a \leq f(x) \leq b$,并设存在常数 $k$,$0 \leq k < 1$,对于 $[a,b]$ 上的任意两点 $x_{1}$ 与 $x_{2}$,都有 $|f(x_{1}) - f(x_{2})| \leq k |x_{1} - x_{2}|$. 证明:
		\begin{enumerate}
			\item 存在唯一的 $\xi \in [a,b]$ 使 $f(\xi) = \xi$;
			\item 对于任意给定的 $x_{1} \in [a,b]$,定义 $x_{n+1} = f(x_{n})$,$n = 1,2,\cdots$,则 $\lim_{n \to \infty} x_{n}$ 存在,且 $\lim_{n \to \infty} x_{n} = \xi$.
		\end{enumerate}
	\end{ti}

	\begin{ti}
		已知 $\left( 2 + \sqrt{2} \right)^{n} = A_{n} + B_{n}\sqrt{2}$,$A_{n},B_{n}$ 为整数,$n = 1,2,3,\cdots$,求 $\lim_{n\to \infty} \frac{A_{n}}{B_{n}}$.
	\end{ti}

	\begin{ti}
		设 $f(x)$ 在 $[0,+\infty)$ 上连续,满足 $0 \leq f(x) \leq x, x \in [0,+\infty)$,设 $a_{1} \geq 0$,$a_{n+1} = f(a_{n})(n = 1,2,\cdots)$,证明:
		\begin{enumerate}
			\item $\{ a_{n} \}$ 为收敛数列;
			\item 设 $\lim_{n \to \infty} a_{n} = t$,则有 $f(t) = t$;
			\item 若条件改为 $0 \leq f(x) < x,x \in (0,+\infty)$,则 $t = 0$.
		\end{enumerate}
	\end{ti}

	\begin{ti}
		\begin{enumerate}
			\item 设 $f(x) = x + \ln(2 - x)$,求 $f(x)$ 的最大值;
			\item 设 $x_{1} = \ln 2$,$x_{n} = \sum_{i=1}^{n-1} \ln(2 - x_{i}), n = 2,3,\cdots$,证明 $\lim_{n \to \infty} x_{n}$ 存在并求其极限值.
		\end{enumerate}
	\end{ti}

	\begin{ti}
		设 $x_{1} = 1$,$x_{n} = \int_{0}^{1} \min\{x,x_{n-1}\} \dd{x}, n = 2,3,\cdots$,证明 $\lim_{n \to \infty} x_{n}$ 存在并求其极限值.
	\end{ti}

	\begin{ti}
		设数列 $\{ x_{n} \}$ 满足 $0 < x_{1} < 1$,$\ln(1 + x_{n}) = \ee^{x_{n+1}} - 1(n = 1,2,\cdots)$,证明
		\begin{enumerate}
			\item 当 $0 < x < 1$ 时,$\ln(1 + x) < x < \ee^{x} - 1$;
			\item $\lim_{n \to \infty} x_{n}$ 存在,并求该极限.
		\end{enumerate}
	\end{ti}

	\begin{ti}
		\begin{enumerate}
			\item 证明方程 $x = 2\ln(1 + x)$ 在 $(0,+\infty)$ 内有唯一实根 $\xi$;
			\item 任取 $x_{1} > \xi$,定义 $x_{n+1} = 2\ln(1 + x_{n}), n = 1,2,\cdots$,证明 $\lim_{n \to \infty} x_{n} = \xi$.
		\end{enumerate}
	\end{ti}

	\begin{ti}
		\begin{enumerate}
			\item 证明方程 $\ee^{x} + x^{2n+1} = 0$ 在 $(-1,0)$ 内有唯一实根 $x_{n}, n = 0,1,2,\cdots$;
			\item 证明 $\lim_{n \to \infty} x_{n}$ 存在并求其值 $a$;
			\item 求 $\lim_{n \to \infty} n(x_{n} - a)$.
		\end{enumerate}
	\end{ti}

	\begin{ti}
		设 $F(x,y) = \frac{f(y - x)}{2x}$,$F(1,y) = \frac{y^{2}}{2} - y + 5$,$x_{0} > 0$,$x_{1} = F(x_{0},2x_{0})$,$\cdots$,$x_{n+1} = F(x_{n},2x_{n}), n = 1,2,\cdots$. 证明 $\lim_{n \to \infty} x_{n}$ 存在,并求该极限.
	\end{ti}

	\begin{ti}
		已知
		\[
			f_{n}(x) = \CC_{n}^{1} \cos x - \CC_{n}^{2} \cos^{2}x + \cdots + (-1)^{n-1} \CC_{n}^{n} \cos^{n}x.
		\]
		\begin{enumerate}
			\item 证明方程 $f_{n}(x) = \frac{1}{2}$ 在区间 $\left( 0,\frac{\uppi}{2} \right)$ 中仅有一根 $x_{n}, n = 1,2,3,\cdots$;
			\item 求 $\lim_{n \to \infty} f_{n}\left( \arccos\frac{1}{n} \right)$;
			\item 设 $x_{n} \in \left( 0,\frac{\uppi}{2} \right)$ 满足 $f_{n}(x_{n}) = \frac{1}{2}$,证明 $\lim_{n \to \infty} x_{n} = \frac{\uppi}{2}$.
		\end{enumerate}
	\end{ti}

	\begin{ti}
		\begin{enumerate}
			\item 证明:当 $x \to 0^{+}$ 时,不等式 $0 < \tan^{2}x - x^{2} < x^{4}$ 成立;
			\item 设 $x_{n} = \sum_{k=1}^{n} \tan^{2}\frac{1}{\sqrt{n+k}}$,求 $\lim_{n \to \infty}x_{n}$.
		\end{enumerate}
	\end{ti}

	\begin{ti}
		\begin{enumerate}
			\item 设 $f(x)$ 在 $(0,+\infty)$ 内可导,$f'(x) > 0, x \in (0,+\infty)$,证明 $f(x)$ 在 $(0,+\infty)$ 内单调增加;
			\item 证明 $f(x) = \left( n^{x} + 1 \right)^{-\frac{1}{x}}$ 在 $(0,+\infty)$ 内单调增加,其中 $n$ 为正整数;
			\item 设数列 $x_{n} = \sum_{k=1}^{n} \left( n^{k} + 1 \right)^{-\frac{1}{k}}$,求 $\lim_{n \to \infty} x_{n}$.
		\end{enumerate}
	\end{ti}
	\subsection{连续与间断}

	\begin{ti}
		当 $x \in \left( -\frac{1}{2},1 \right]$ 时,确定函数 $f(x) = \frac{\tan \uppi x}{|x|\left( x^{2} - 1 \right)}$ 的间断点,并判定其类型.
	\end{ti}

	\begin{ti}
		确定函数 $f(x) = \frac{x(x - 1)}{|x| x^{2} - |x|}$ 的间断点,并判定其类型.
	\end{ti}

	\begin{ti}
		设 $a > 0$,$b > 0$,$c > 0$,
		\[
			A(x) = \begin{cases}
				\left( \frac{a^{x} + b^{x}}{2} \right)^{\frac{1}{x}}, & x \ne 0,\\
				c, & x = 0.
			\end{cases}
		\]
		\begin{enumerate}
			\item 讨论 $A(x)$ 在 $x = 0$ 处的连续性;
			\item 讨论 $\lim_{x \to +\infty} A(x)$,$\lim_{x \to -\infty} A(x)$,$\lim_{x \to 0} A(x)$,$A(-1)$,$A(1)$ 五者之间的大小关系.
		\end{enumerate}
	\end{ti}

	\begin{ti}
		求 $f(x) = \frac{1}{1 - \ee^{\frac{x}{1 - x}}}$ 的连续区间、间断点,并判别间断点的类型.
	\end{ti}

	\begin{ti}
		求函数 $f(x) = \lim_{n \to \infty} \frac{x^{n+2} - x^{-n}}{x^{n} + x^{-n}}$ 的间断点并指出其类型.
	\end{ti}

	\begin{ti}
		若
		\[
			f(x) = \frac{\sqrt[3]{x}}{\lambda - \ee^{-kx}}
		\]
		在 $(-\infty,+\infty)$ 内连续,且 $\lim_{x \to -\infty} f(x) = 0$,则\kuo.
		
		\twoch{$\lambda < 0, k < 0$}{$\lambda < 0, k > 0$}{$\lambda \geq 0, k < 0$}{$\lambda \leq 0, k > 0$}
	\end{ti}

	\begin{ti}
		若
		\[
			f(x) = \begin{cases}
				\ee^{x} (\sin x + \cos x), & x > 0,\\
				2x + a, & x \leq 0
			\end{cases}
		\]
		 是 $(-\infty,+\infty)$ 内的连续函数,则 $a =$\htwo.
	\end{ti}

	\begin{ti}
		试讨论函数 $g(x) = \begin{cases}
			x^{\alpha} \sin\frac{1}{x}, & x > 0,\\
			\ee^{x} + \beta, & x \leq 0
		\end{cases}$ 在点 $x = 0$ 处的连续性.
	\end{ti}

	\begin{ti}
		求函数 $F(x) = \begin{cases}
			\frac{x(\uppi + 2x)}{2 \cos x}, & x \leq 0,\\
			\sin\frac{1}{x^{2} - 1}, & x > 0
		\end{cases}$ 的间断点,并判断它们的类型.
	\end{ti}

	\begin{ti}
		设 $f(x) = \lim_{n \to \infty}\frac{\ee^{\frac{1}{x}} \arctan\frac{1}{1 + x}}{x^{2} + \ee^{nx}}$,求 $f(x)$ 的间断点并判定其类型.
	\end{ti}

	\begin{ti}
		设 $f(x) = \begin{cases}
			\ee^{\frac{1}{x - 1}}, & x > 0,\\
			\ln(1 + x), & -1 < x < 0,
		\end{cases}$ 求 $f(x)$ 的间断点,并说明间断点的类型.
	\end{ti}

	\begin{ti}
		设 $f(x;t) = \left( \frac{x - 1}{t - 1} \right)^{\frac{t}{x - t}}((x - 1)(t - 1)>0, x \ne t)$,函数 $f(x)$ 由表达式
		\[
			f(x) = \lim_{t \to x}f(x;t)
		\]
		确定,求 $f(x)$ 的连续区间和间断点,并判定间断点的类型.
	\end{ti}

	\begin{ti}
		设函数 $f(x)$ 在 $[a,b]$ 上连续,$x_{1},x_{2},\cdots,x_{n},\cdots$ 是 $[a,b]$ 上的一个点列,求 $\lim_{n \to \infty} \sqrt[n]{\frac{1}{n}\sum_{k=1}^{n}\ee^{f(x_{k})}}$.
	\end{ti}

	\begin{ti}
		\begin{enumerate}
			\item 求函数 $f(x) = \lim_{n \to \infty} \sqrt[n]{1 + (2x)^{n} + x^{2n}}(x \geq 0)$ 的表达式;
			\item 讨论函数 $f(x)$ 的连续性.
		\end{enumerate}
	\end{ti}

	\begin{ti}
		已知 $f(x) = \lim_{n \to \infty} \frac{x^{2n-1} + ax^{2} + bx}{x^{2n} + 1}$ 是连续函数,求 $a,b$ 的值.
	\end{ti}

	\begin{ti}
		求函数 $f(x) = \frac{x^{3} + 1}{|x + 1|\left( x^{2} - x \right)} \sin\left( \frac{|x - 1|}{x + 2}\uppi \right)$ 的所有间断点,并判断它们的类型.
	\end{ti}
	% \ctexset{
	section={
		format={\zihao{4}\bfseries\raggedright},
		name={,、},
		aftername={\hspace{0em}},
		number=\chinese{section},
	},
}
\chapter{线性代数}
	线性代数是硕士研究生招生考试考查内容之一,主要考查考生对线性代数的基本概念基本理论、基本运算的理解和掌握以及考生的抽象思维能力、逻辑推理能力、空间想象能力、综合运用能力和解决实际问题的能力。在考研数学一试卷中分值为 34 分,约占 \SI{22}{\percent}。
	\section{行列式}

	\begin{titwo}
		设 $\begin{vsmallmatrix}
			a_{11} & a_{12} & a_{13} & a_{14}\\
			a_{21} & a_{22} & a_{23} & a_{24}\\
			a_{31} & a_{32} & a_{33} & a_{34}\\
			a_{41} & a_{42} & a_{43} & a_{44}
		\end{vsmallmatrix} = m, c \ne 0$,则 
		\[
			\begin{vsmallmatrix}
				a_{11} & a_{12}c & a_{13}c^{2} & a_{14}c^{3}\\
				a_{21}c^{-1} & a_{22} & a_{23}c & a_{24}c^{2}\\
				a_{31}c^{-2} & a_{32}c^{-1} & a_{33} & a_{34}c\\
				a_{41}c^{-3} & a_{42}c^{-2} & a_{43}c^{-1} & a_{44}
			\end{vsmallmatrix}
		\]
		等于\kuo.

		\fourch{$c^{-2}m$}{$m$}{$cm$}{$c^{3}m$}
	\end{titwo}

	\begin{titwo}
		$\begin{vsmallmatrix}
			a & b & c & d \\
			x & 0 & 0 & y \\
			y & 0 & 0 & x \\
			d & c & b & a
		\end{vsmallmatrix} = $\htwo.
	\end{titwo}

	\begin{titwo}
		设 $a, b, a + b$ 均非零,则行列式 $\begin{vsmallmatrix}
			a & b & a + b \\
			b & a + b & a \\
			a + b & a & b \\
		\end{vsmallmatrix} = $
		
		\noindent\htwo.
	\end{titwo}

	\begin{titwo}
		设 $n$ 阶矩阵 $\bm A = \begin{bsmallmatrix}
			0 & 1 & 1 & \cdots & 1 & 1 \\
			1 & 0 & 1 & \cdots & 1 & 1 \\
			1 & 1 & 0 & \cdots & 1 & 1 \\
			\vdots & \vdots & \vdots &  & \vdots & \vdots \\
			1 & 1 & 1 & \cdots & 0 & 1 \\
			1 & 1 & 1 & \cdots & 1 & 0 
		\end{bsmallmatrix}$,则 $|\bm A| = $\htwo.
	\end{titwo}

	\begin{titwo}
		计算 $n$ 阶行列式 $\begin{bsmallmatrix}
			a & b & 0 & \cdots & 0 & 0 \\
			0 & a & b & \cdots & 0 & 0 \\
			0 & 0 & a & \cdots & 0 & 0 \\
			\vdots & \vdots & \vdots &  & \vdots & \vdots \\
			0 & 0 & 0 & \cdots & a & b \\
			b & 0 & 0 & \cdots & 0 & a 
		\end{bsmallmatrix}$.
	\end{titwo}

	\begin{titwo}
		计算行列式 $\begin{vsmallmatrix}
			x + 1 & x & x & \cdots & x \\
			x & x + \frac{1}{2} & x & \cdots & x \\
			x & x & x + \frac{1}{3} & \cdots & x \\
			\vdots & \vdots & \vdots &  & \vdots \\
			x & x & x & \cdots & x + \frac{1}{n}
		\end{vsmallmatrix}$.
	\end{titwo}

	\begin{titwo}
		计算行列式 $\begin{vsmallmatrix}
			1 - x & x & 0 & 0 & 0 \\
			-1 & 1 - x & x & 0 & 0 \\
			0 & -1 & 1 - x & x & 0 \\
			0 & 0 & -1 & 1 - x & x \\
			0 & 0 & 0 & -1 & 1 - x
		\end{vsmallmatrix}$.
	\end{titwo}

	\begin{titwo}
		计算行列式 $\begin{vsmallmatrix}
			a & b & c & d \\
			-b & a & -d & c \\
			-c & d & a & -b \\
			-d & -c & b & a \\
		\end{vsmallmatrix}$.
	\end{titwo}

	\begin{titwo}
		行列式 $D_{n+1} = \begin{vsmallmatrix}
			a^{n} & (a + 1)^{n} & \cdots & (a + n)^{n} \\
			a^{n - 1} & (a + 1)^{n - 1} & \cdots & (a + n)^{n - 1} \\
			\vdots & \vdots &  & \vdots \\
			a & a + 1 & \cdots & a + n \\
			1 & 1 & \cdots & 1
		\end{vsmallmatrix} = $\htwo.
	\end{titwo}

	\begin{titwo}
		设 $n$ 阶行列式
		\[
			D_{n} = \begin{vsmallmatrix}
				2 & 1 & 0 & \cdots & 0 & 0 \\
				1 & 2 & 1 & \cdots & 0 & 0 \\
				0 & 1 & 2 & \cdots & 0 & 0 \\
				\vdots & \vdots & \vdots &  & \vdots & \vdots \\
				0 & 0 & 0 & \cdots & 2 & 1 \\
				0 & 0 & 0 & \cdots & 1 & 2
			\end{vsmallmatrix},
		\]
		则 $\sum_{i=1}^{n} D_{i} = $\htwo.
	\end{titwo}

	\begin{titwo}
		设 $D_{n} = \begin{vsmallmatrix}
			a + 2 & 2a & 0 & \cdots & 0 & 0 \\
			1 & a + 2 & 2a & \cdots & 0 & 0 \\
			0 & 1 & a + 2 & \cdots & 0 & 0 \\
			\vdots & \vdots & \vdots &  & \vdots & \vdots \\
			0 & 0 & 0 & \cdots & a + 2 & 2a \\
			0 & 0 & 0 & \cdots & 1 & a + 2 
		\end{vsmallmatrix}$,其中 $n \geq 3$.

		\noindent 则 $\frac{D_{n} - a D_{n - 1}}{D_{n - 1} - a D_{n - 2}} = $\htwo.
	\end{titwo}

	\begin{titwo}
		设 $\bm \alpha_{1},\bm \alpha_{2},\bm \alpha_{3},\bm \beta_{1},\bm \beta_{2}$ 都是 $4$ 维列向量,且 $4$ 阶行列式 $\bigl|\bm \alpha_{1},\bm \alpha_{2},\bm \alpha_{3},\bm \beta_{1}\bigr| = m,\bigl|\bm \alpha_{1},\bm \alpha_{2},\bm \beta_{2},\bm \alpha_{3}\bigr| = n$,则 $4$ 阶行列式 $\bigl|\bm \alpha_{3},\bm \alpha_{2},\bm \alpha_{1},\bm \beta_{1} + \bm \beta_{2}\bigr|$ 等于\kuo.

		\fourch{$m + n$}{$- (m + n)$}{$n - m$}{$m - n$}
	\end{titwo}

	\begin{titwo}
		设 $\bm A = [\bm \alpha_{1},\bm \alpha_{2},\bm \alpha_{3}]$ 是 $3$ 阶矩阵,且 $|\bm A| = 4$,若
		\[
			\bm B = [\bm \alpha_{1} - 3 \bm \alpha_{2} + 2 \bm \alpha_{3}, \bm \alpha_{2} - 2 \bm \alpha_{3}, 2 \bm \alpha_{2} + \bm \alpha_{3}],
		\]
		则 $|\bm B| = $\htwo.
	\end{titwo}

	\begin{titwo}
		设 $\bm A$ 是 $m$ 阶矩阵,$\bm B$ 是 $n$ 阶矩阵,且
		\[
			|\bm A| = a, |\bm B| = b, \bm C = \begin{bsmallmatrix}
				\bm O & \bm A \\
				\bm B & \bm O
			\end{bsmallmatrix},
		\]
		则 $|\bm C| = $\htwo.
	\end{titwo}

	\begin{titwo}
		设 $\bm A$ 为奇数阶矩阵,且 $\bm A \bm A^{\TT} = \bm A^{\TT} \bm A = \bm E, |\bm A| > 0$,则 $|\bm A - \bm E| = $\htwo.
	\end{titwo}

	\begin{titwo}
		设 $\bm A$ 是 $n$ 阶矩阵,满足 $\bm A \bm A^{\TT} = \bm E$($\bm E$ 是 $n$ 阶单位矩阵,$\bm A^{\TT}$ 是 $\bm A$ 的转置矩阵),且 $|\bm A| < 0$,求 $|\bm A + \bm E|$.
	\end{titwo}
\subsection{矩阵}

	\begin{titwo}
		设 $n$ 维行向量 $\bm \alpha = \bigl[ \frac{1}{2}, 0 , \cdots , 0 , \frac{1}{2} \bigr]$,矩阵 $\bm A = \bm E - \bm \alpha^{\TT} \bm \alpha, \bm B = \bm E + 2 \bm \alpha^{\TT} \bm \alpha$,则 $\bm A \bm B = $\kuo.

		\fourch{$\bm O$}{$- \bm E$}{$\bm E$}{$\bm E + \bm \alpha^{\TT} \bm \alpha$}
	\end{titwo}

	\begin{titwo}
		已知 $\bm A, \bm B, \bm A + \bm B, \bm A^{-1} + \bm B^{-1}$ 均为 $n$ 阶可逆矩阵,则 $\bigl( \bm A^{-1} + \bm B^{-1} \bigr)^{-1}$ 等于\kuo.

		\twoch{$\bm A + \bm B$}{$\bm A^{-1} + \bm B^{-1}$}{$\bm A(\bm A + \bm B)^{-1} \bm B$}{$(\bm A + \bm B)^{-1}$}
	\end{titwo}

	\begin{titwo}
		设 $\bm A$ 是 $n$ 阶方阵,且 $\bm A^{3} = \bm O$,则\kuo.

		\onech{$\bm A$ 不可逆,且 $\bm E - \bm A$ 不可逆}{$\bm A$ 可逆,但 $\bm E + \bm A$ 不可逆}{$\bm A^{2} - \bm A + \bm E$ 及 $\bm A^{2} + \bm A + \bm E$ 均可逆}{$\bm A$ 不可逆,且必有 $\bm A^{2} = \bm O$}
	\end{titwo}

	\begin{titwo}
		设 $\bm A$ 为 $n$ 阶可逆矩阵,则下列等式中,不一定成立的是\kuo.

		\onech{$\bigl( \bm A + \bm A^{-1} \bigr)^{2} = \bm A^{2} + 2 \bm A \bm A^{-1} + \bigl( \bm A^{-1} \bigr)^{2}$}{$\bigl( \bm A + \bm A^{\TT} \bigr)^{2} = \bm A^{2} + 2 \bm A \bm A^{\TT} + \bigl( \bm A^{\TT} \bigr)^{2}$}{$\bigl( \bm A + \bm A^{\astt} \bigr)^{2} = \bm A^{2} + 2 \bm A \bm A^{\astt} + \bigl( \bm A^{\astt} \bigr)^{2}$}{$\bigl( \bm A + \bm E \bigr)^{2} = \bm A^{2} + 2 \bm A \bm E + \bm E^{2}$}
	\end{titwo}

	\begin{titwo}
		设 $\bm A = \begin{bsmallmatrix}
			0 & 1 & 1 & 1 \\
			1 & 0 & 1 & 1 \\
			1 & 1 & 0 & 1 \\
			1 & 1 & 1 & 0
		\end{bsmallmatrix}$,则 $\bm A^{-1}=$\htwo.
	\end{titwo}

	\begin{titwo}
		设 $\bm B = \begin{bsmallmatrix}
			0 & b_{1} & 0 & \cdots & 0 \\
			0 & 0 & b_{2} & \cdots & 0 \\
			\vdots & \vdots & \vdots &  & \vdots \\
			0 & 0 & 0 & \cdots & b_{n-1} \\
			b_{n} & 0 & 0 & \cdots & 0
		\end{bsmallmatrix}$,则 $\bm B^{-1} = $\htwo.
	\end{titwo}

	\begin{titwo}
		设 $\bm B = 2 \bm A - \bm E$,证明:$\bm B^{2} = \bm E$ 的充分必要条件是 $\bm A^{2} = \bm A$.
	\end{titwo}

	\begin{titwo}
		设 $\bm A = \begin{bsmallmatrix}
			a & b \\
			c & d
		\end{bsmallmatrix}$.
		\begin{enumerate}
			\item 计算 $\bm A^{2}$,并将 $\bm A^{2}$ 用 $\bm A$ 和 $\bm E$ 线性表出;
			\item 证明:当 $k > 2$ 时,$\bm A^{k} = \bm O$ 的充分必要条件为 $\bm A^{2} = \bm O$.
		\end{enumerate}
	\end{titwo}

	\begin{titwo}
		设 $\bm M = \begin{bsmallmatrix}
			\bm A & \bm B \\
			\bm O & \bm D
		\end{bsmallmatrix}$ 可逆,其中 $\bm A, \bm D$ 皆为方阵,证明 $\bm A, \bm D$ 可逆,并求 $\bm M^{-1}$.
	\end{titwo}

	\begin{titwo}
		设 $\bm A$ 为 $n$ 阶非奇异矩阵,$\bm \alpha$ 为 $n$ 维列向量,$b$ 为常数. 记分块矩阵
		\[
			\bm P = \begin{bsmallmatrix}
				\bm E & \bm 0 \\
				- \bm \alpha^{\TT} \bm A^{\astt} & |\bm A|
			\end{bsmallmatrix},
			\bm Q = \begin{bsmallmatrix}
				\bm A & \bm \alpha \\
				\bm \alpha^{\TT} & b
			\end{bsmallmatrix},
		\]
		其中 $\bm A^{\astt}$ 是矩阵 $\bm A$ 的伴随矩阵,$\bm E$ 为 $n$ 阶单位矩阵.
		\begin{enumerate}
			\item 计算并化简 $\bm P \bm Q$;
			\item 证明:矩阵 $\bm Q$ 可逆的充分必要条件是 $\bm \alpha^{\TT} \bm A^{-1} \bm \alpha \ne b$.
		\end{enumerate}
	\end{titwo}

	\begin{titwo}
		已知 $\bm A$ 是 $n$ 阶方阵,$\bm E$ 是 $n$ 阶单位矩阵,且
		\[
			\bm A^{3} = \bm E,
		\]
		则 $\begin{bsmallmatrix}
			\bm O & - \bm E \\
			\bm A & \bm O
		\end{bsmallmatrix}^{98} = $\kuo.

		\fourch{$\begin{bsmallmatrix}
			\bm A & \bm E \\
			\bm O & \bm A
		\end{bsmallmatrix}$}{$\begin{bsmallmatrix}
			\bm A & \bm O \\
			\bm E & \bm A
		\end{bsmallmatrix}$}{$\begin{bsmallmatrix}
			\bm A & \bm O \\
			\bm O & \bm A
		\end{bsmallmatrix}$}{$\begin{bsmallmatrix}
			- \bm A & \bm O \\
			\bm O & - \bm A
		\end{bsmallmatrix}$}
	\end{titwo}

	\begin{titwo}
		下列命题正确的是\kuo.

		\onech{若 $\bm A \bm B = \bm E$,则 $\bm A$ 必可逆,且 $\bm A^{-1} = \bm B$}{若 $\bm A, \bm B$ 均为 $n$ 阶可逆矩阵,则 $\bm A + \bm B$ 必可逆}{若 $\bm A, \bm B$ 均为 $n$ 阶不可逆矩阵,则 $\bm A - \bm B$ 必不可逆}{若 $\bm A, \bm B$ 均为 $n$ 阶不可逆矩阵,则 $\bm A \bm B$ 必不可逆}
	\end{titwo}

	\begin{titwo}
		设 $\bm A$ 为 $3$ 阶非零矩阵,且满足
		\[
			a_{ij} = A_{ij} (i,j = 1,2,3),
		\]
		其中 $A_{ij}$ 为 $a_{ij}$ 的代数余子式,则下列结论中:\circled{1} $\bm A$ 是可逆矩阵; \circled{2} $\bm A$ 是对称矩阵; \circled{3} $\bm A$ 是不可逆矩阵; \circled{4} $\bm A$ 是正交矩阵. 正确的个数为\kuo.

		\fourch{$1$}{$2$}{$3$}{$4$}
	\end{titwo}

	\begin{titwo}
		设 $n$ 阶矩阵 $\bm A, \bm B$ 等价,则下列说法中,不一定成立的是\kuo.

		\onech{如果 $|\bm A| > 0$,则 $|\bm B| > 0$}{如果 $\bm A$ 可逆,则存在可逆矩阵 $\bm P$,使得 $\bm P \bm B = \bm E$}{如果 $\bm A, \bm E$ 等价,则 $|\bm B| \ne 0$}{存在可逆矩阵 $\bm P$ 与 $\bm Q$,使得 $\bm P \bm A \bm Q = \bm B$}
	\end{titwo}

	\begin{titwo}
		设 $\bm A = \begin{bsmallmatrix}
			1 & 0 & 0 & 0 \\
			-2 & 3 & 0 & 0 \\
			0 & -4 & 5 & 0 \\
			0 & 0 & -6 & 7 \\
		\end{bsmallmatrix}, \bm B = (\bm E + \bm A)^{-1} (\bm E - \bm A)$,则 $(\bm E + \bm B)^{-1} = $\htwo.
	\end{titwo}

	\begin{titwo}
		设
		\[
			\bm B = \begin{bsmallmatrix}
				0 & 1 & 0 & 0 \\
				0 & 0 & 1 & 0 \\
				0 & 0 & 0 & 1 \\
				0 & 0 & 0 & 0
			\end{bsmallmatrix},
		\]
		证明 $\bm A = \bm E + \bm B$ 可逆,并求 $\bm A^{-1}$.
	\end{titwo}

	\begin{titwo}
		证明:方阵 $\bm A$ 与所有同阶对角矩阵可交换的充分必要条件是 $\bm A$ 是对角矩阵.
	\end{titwo}

	\begin{titwo}
		设 $\bm \alpha, \bm \beta$ 为 $n$ 维单位列向量,$\bm P$ 是 $n$ 阶可逆矩阵,则下列矩阵中可逆的是\kuo.

		\twoch{$\bm A = \bm E - \bm \alpha \bm \alpha^{\TT}$}{$\bm B = \bm \alpha^{\TT} \bm P \bm \alpha \bm P^{-1} - \bm \alpha \bm \alpha^{\TT}$}{$\bm C = \bm \alpha^{\TT} \bm P^{-1} \bm \beta \bm P - \bm \beta \bm \alpha^{\TT}$}{$\bm D = \bm E + \bm \beta \bm \beta^{\TT}$}
	\end{titwo}

	\begin{titwo}
		设 $\bm A$ 是 $m \times n$ 矩阵,$\bm B$ 是 $n \times m$ 矩阵,已知 $\bm E_{m} + \bm A \bm B$ 可逆.
		\begin{enumerate}
			\item 验证 $\bm E_{n} + \bm B \bm A$ 可逆,且
			\[
				(\bm E_{n} + \bm B \bm A)^{-1} = \bm E_{n} - \bm B (\bm E_{m} + \bm A \bm B)^{-1} \bm A;
			\]
			\item 设 $\bm W = \begin{bsmallmatrix}
				1 + a_{1} b_{1} & a_{1} b_{2} & a_{1} b_{3} \\
				a_{2} b_{1} & 1 + a_{2} b_{2} & a_{2} b_{3} \\
				a_{3} b_{1} & a_{3} b_{2} & 1 + a_{3} b_{3}
			\end{bsmallmatrix}$,其中 $a_{1} b_{1} + a_{2} b_{2} + a_{3} b_{3} = 0$. 证明:$\bm W$ 可逆,并求 $\bm W^{-1}$.
		\end{enumerate}
	\end{titwo}

	\begin{titwo}
		设
		\[
			\bm \alpha = [1,2,3], \bm \beta = \Biggl[ 1,\frac{1}{2},\frac{1}{3} \Biggr], \bm A = \bm \alpha^{\TT} \bm \beta,
		\]
		则 $\bm A^{n} = $\htwo.
	\end{titwo}

	\begin{titwo}
		设 $\bm A = \begin{bsmallmatrix}
			1 & 2 & 3 \\
			0 & 1 & 4 \\
			0 & 0 & 1
		\end{bsmallmatrix}$,求 $\bm A^{n} (n \geq 3)$.
	\end{titwo}

	\begin{titwo}
		设
		\[
			\bm A = \begin{bsmallmatrix}
				1 & 0 & 1 \\
				0 & 2 & 0 \\
				1 & 0 & 1
			\end{bsmallmatrix},
		\]
		$n \geq 2$ 为正整数,则 $\bm A^{n} - 2 \bm A^{n-1} = $\htwo.
	\end{titwo}

	\begin{titwo}
		设 $\bm A = \begin{bsmallmatrix}
			1 & 0 & 0 \\
			1 & 0 & 1 \\
			0 & 1 & 0
		\end{bsmallmatrix}$.
		\begin{enumerate}
			\item 证明当 $n \geq 3$ 时,有 $\bm A^{n} = \bm A^{n-2} + \bm A^{2} - \bm E$;
			\item 求 $\bm A^{100}$.
		\end{enumerate}
	\end{titwo}

	\begin{titwo}
		已知 $\bm A = \begin{bsmallmatrix}
			3 & 1 & 0 & 0 & 0 \\
			0 & 3 & 1 & 0 & 0 \\
			0 & 0 & 3 & 0 & 0 \\
			0 & 0 & 0 & 3 & -1 \\
			0 & 0 & 0 & -9 & 3
		\end{bsmallmatrix}$,求 $\bm A^{n}(n \geq 2)$.
	\end{titwo}

	\begin{titwo}
		设 $\bm \alpha = [a_{1},a_{2},\cdots,a_{n}]^{\TT} \ne \bm 0, \beta = [b_{1},b_{2},\cdots,b_{n}]^{\TT} \ne \bm 0$,且 $\bm \alpha^{\TT} \bm \beta = 0, \bm A = \bm E + \bm \alpha \bm \beta^{\TT}$,试计算:
		\begin{enumerate}
			\item $|\bm A|$;
			\item $\bm A^{n}$;
			\item $\bm A^{-1}$.
		\end{enumerate}
	\end{titwo}

	\begin{titwo}
		设 $\bm A$ 是 $n(n \geq 2)$ 阶方阵,$\bm A^{\astt}$ 是 $\bm A$ 的伴随矩阵,则 $|\bm A^{\astt}| = $\kuo.

		\fourch{$|\bm A|$}{$|\bm A^{-1}|$}{$|\bm A^{n-1}|$}{$|\bm A^{n}|$}
	\end{titwo}

	\begin{titwo}
		设 $\bm A$ 是 $n(n \geq 2)$ 阶方阵,$|\bm A| = 3$,则 $|( A^{\astt} )^{\astt}| = $
		
		\noindent\kuo.

		\fourch{$3^{(n-1)^{2}}$}{$3^{n^{2} - 1}$}{$3^{n^{2} - n}$}{$3^{n-1}$}
	\end{titwo}

	\begin{titwo}
		设 $\bm A$ 是 $n(n \geq 2)$ 阶可逆方阵,$\bm A^{\astt}$ 是 $\bm A$ 的伴随矩阵,则 $(\bm A^{\astt})^{\astt} = $\kuo.

		\fourch{$|\bm A|^{n-1} \bm A$}{$|\bm A|^{n+1} \bm A$}{$|\bm A|^{n-2} \bm A$}{$|\bm A|^{n+2} \bm A$}
	\end{titwo}

	\begin{titwo}
		设 $\bm A_{n \times n}$ 是正交矩阵,则\kuo.

		\twoch{$\bm A^{\astt} (\bm A^{\astt})^{\TT} = |\bm A| \bm E$}{$(\bm A^{\astt})^{\TT} \bm A^{\astt} = |\bm A^{\astt}| \bm E$}{$\bm A^{\astt} (\bm A^{\astt})^{\TT} = \bm E$}{$(\bm A^{\astt})^{\TT} \bm A^{\astt} = - \bm E$}
	\end{titwo}

	\begin{titwo}
		设 $\bm A = \frac{1}{2} \begin{bsmallmatrix}
			2 & 0 & 0 \\
			0 & 0 & 1 \\
			0 & 3 & 0
		\end{bsmallmatrix}$,则 $(\bm A^{\astt})^{-1} = $\htwo.
	\end{titwo}

	\begin{titwo}
		证明:若 $\bm A$ 为 $n$ 阶可逆方阵,$\bm A^{\astt}$ 为 $\bm A$ 的伴随矩阵,则 $(\bm A^{\astt})^{\TT} = \bigl( \bm A^{\TT} \bigr)^{\astt}$.
	\end{titwo}

	\begin{titwo}
		证明:若 $\bm A$ 为 $n(n \geq 2)$ 阶方阵,则有
		\[
			|\bm A^{\astt}| = |(- \bm A)^{\astt}|.
		\]
	\end{titwo}

	\begin{titwo}
		设 $\bm A$ 是 $n$ 阶矩阵,则 $\left| -2 \begin{bsmallmatrix}
			\bm A^{\astt} & \bm O \\
			\bm A + \bm A^{\astt} & \bm A
		\end{bsmallmatrix} \right| = $\kuo.

		\twoch{$(-2)^{n} |\bm A|^{n}$}{$(4 |\bm A|)^{n}$}{$(-2)^{2n} |\bm A^{\astt}|^{n}$}{$|4 \bm A|^{n}$}
	\end{titwo}

	\begin{titwo}
		设 $\bm A$ 是 $n$ 阶矩阵,$|\bm A| = 5$,则 $|( 2 \bm A )^{\astt}| = $\htwo.
	\end{titwo}

	\begin{titwo}
		$|\bm A|$ 是 $n$ 阶行列式,其中有一行(列)元素全是 $1$,证明:这个行列式的全部代数余子式的和等于该行列式的值.
	\end{titwo}

	\begin{titwo}
		$\bm A$ 为 $n(n \geq 3)$ 阶非零实矩阵,$A_{ij}$ 为 $|\bm A|$ 中元素 $a_{ij}$ 的代数余子式,试证明:
		\begin{enumerate}
			\item $a_{ij} = A_{ij} \Leftrightarrow \bm A^{\TT} \bm A = \bm E$,且 $|\bm A| = 1$;
			\item $a_{ij} = -A_{ij} \Leftrightarrow \bm A^{\TT} \bm A = \bm E$,且 $|\bm A| = -1$.
		\end{enumerate}
	\end{titwo}

	\begin{titwo}
		设 $\bm A = \begin{bsmallmatrix}
			1 & 2 & 3 & 4 \\
			0 & 1 & 2 & 3 \\
			0 & 0 & 1 & 2 \\
			0 & 0 & 0 & 1
		\end{bsmallmatrix}$,求 $|\bm A|$ 的所有代数余子式之和.
	\end{titwo}

	\begin{titwo}
		证明:$n > 3$ 的非零实方阵 $\bm A$,若它的每个元素等于自己的代数余子式,则 $\bm A$ 是正交矩阵.
	\end{titwo}

	\begin{titwo}
		已知 $\bm A, \bm B$ 均是 $3$ 阶矩阵,将 $\bm A$ 中第 $3$ 行的 $-2$ 倍加到第 $2$ 行得矩阵 $\bm A_{1}$,将 $\bm B$ 中第 $1$ 列和第 $2$ 列对换得到 $\bm B_{1}$,又 $\bm A_{1} \bm B_{1} = \begin{bsmallmatrix}
			1 & 1 & 1 \\
			1 & 0 & 2 \\
			2 & 1 & 3
		\end{bsmallmatrix}$,则 $\bm A \bm B = $\htwo.
	\end{titwo}

	\begin{titwo}
		已知 $\bm A = \begin{bsmallmatrix}
			0 & 1 & 0 \\
			1 & 0 & 0 \\
			0 & 0 & 1
		\end{bsmallmatrix}^{5} \begin{bsmallmatrix}
			1 & 0 & 0 \\
			0 & 5 & 0 \\
			0 & 0 & 3
		\end{bsmallmatrix} \begin{bsmallmatrix}
			1 & 0 & 0 \\
			0 & 1 & 1 \\
			0 & 0 & 1
		\end{bsmallmatrix}^{4}$,则 $\bm A^{-1} = $\htwo.
	\end{titwo}

	\begin{titwo}
		设 $\bm A$ 是 $n$ 阶可逆矩阵,将 $\bm A$ 的第 $i$ 行和第 $j$ 行对换得到的矩阵记为 $\bm B$. 证明 $\bm B$ 可逆,并推导 $\bm A^{-1}$ 和 $\bm B^{-1}$ 的关系.
	\end{titwo}

	\begin{titwo}
		设
		\begin{gather*}
			\bm A = \begin{bsmallmatrix}
				a_{11} & a_{12} & a_{13} \\
				a_{21} & a_{22} & a_{23} \\
				a_{31} & a_{32} & a_{33}
			\end{bsmallmatrix},
			\bm B = \begin{bsmallmatrix}
				a_{21} & a_{22} & a_{23} \\
				a_{11} & a_{12} & a_{13} \\
				a_{31} + a_{11} & a_{32} + a_{12} & a_{33} + a_{13}
			\end{bsmallmatrix}, \\
			\bm P_{1} = \begin{bsmallmatrix}
				0 & 1 & 0 \\
				1 & 0 & 0 \\
				0 & 0 & 1
			\end{bsmallmatrix},
			\bm P_{2} = \begin{bsmallmatrix}
				1 & 0 & 0 \\
				0 & 1 & 0 \\
				1 & 0 & 1
			\end{bsmallmatrix},
		\end{gather*}
		则必有\kuo.
		
		\twoch{$\bm A \bm P_{1} \bm P_{2} = \bm B$}{$\bm A \bm P_{2} \bm P_{1} = \bm B$}{$\bm P_{1} \bm P_{2} \bm A = \bm B$}{$\bm P_{2} \bm P_{1} \bm A = \bm B$}
	\end{titwo}

	\begin{titwo}
		设
		\begin{gather*}
			\bm A = \begin{bsmallmatrix}
				a_{11} & a_{12} & a_{13} & a_{14} \\
				a_{21} & a_{22} & a_{23} & a_{24} \\
				a_{31} & a_{32} & a_{33} & a_{34} \\
				a_{41} & a_{42} & a_{43} & a_{44}
			\end{bsmallmatrix},
			\bm B = \begin{bsmallmatrix}
				a_{14} & a_{13} & a_{12} & a_{11} \\
				a_{24} & a_{23} & a_{22} & a_{21} \\
				a_{34} & a_{33} & a_{32} & a_{31} \\
				a_{44} & a_{43} & a_{42} & a_{41}
			\end{bsmallmatrix}, \\
			\bm P_{1} = \begin{bsmallmatrix}
				0 & 0 & 0 & 1 \\
				0 & 1 & 0 & 0 \\
				0 & 0 & 1 & 0 \\
				1 & 0 & 0 & 0
			\end{bsmallmatrix},
			\bm P_{2} = \begin{bsmallmatrix}
				1 & 0 & 0 & 0 \\
				0 & 0 & 1 & 0 \\
				0 & 1 & 0 & 0 \\
				0 & 0 & 0 & 1
			\end{bsmallmatrix}.
		\end{gather*}
		其中 $\bm A$ 可逆,则 $\bm B^{-1}$ 等于\kuo.

		\twoch{$\bm A^{-1} \bm P_{1} \bm P_{2}$}{$\bm P_{1} \bm A^{-1} \bm P_{2}$}{$\bm P_{1} \bm P_{2} \bm A^{-1}$}{$\bm P_{2} \bm A^{-1} \bm P_{1}$}
	\end{titwo}

	\begin{titwo}
		设 $\bm A, \bm B$ 是 $n$ 阶方阵,则下列结论正确的是
		
		\noindent\kuo.

		\onech{$\bm A \bm B = \bm O \Leftrightarrow \bm A = \bm O $ 或 $\bm B = \bm O$}{$|\bm A| = 0 \Leftrightarrow \bm A = \bm O$}{$|\bm A \bm B| = 0 \Leftrightarrow |\bm A| = 0$ 或 $|\bm B| = 0$}{$\bm A = \bm E \Leftrightarrow |\bm A| = 1$}
	\end{titwo}

	\begin{titwo}
		已知 $n$ 阶方阵 $\bm A$ 满足矩阵方程 $\bm A^{2} - 3 \bm A - 2 \bm E = \bm O$. 证明 $\bm A$ 可逆,并求出其逆矩阵 $\bm A^{-1}$.
	\end{titwo}

	\begin{titwo}
		已知对于 $n$ 阶方阵 $\bm A$,存在自然数 $k$,使得 $\bm A^{k} = \bm O$. 证明矩阵 $\bm E - \bm A$ 可逆,并写出其逆矩阵的表达式($\bm E$ 为 $n$ 阶单位矩阵).
	\end{titwo}

	\begin{titwo}
		设矩阵 $\bm A = \begin{bsmallmatrix}
			1 & 0 & 1 \\
			0 & 2 & 0 \\
			1 & 0 & 1
		\end{bsmallmatrix}$,矩阵 $\bm X$ 满足 $\bm A \bm X + \bm E = \bm A^{2} + \bm X$,其中 $\bm E$ 为 $3$ 阶单位矩阵. 求矩阵 $\bm X$.
	\end{titwo}

	\begin{titwo}
		设 $\bm A, \bm B$ 均是 $n$ 阶矩阵,且 $\bm A \bm B = \bm A + \bm B$. 证明 $\bm A - \bm E$ 可逆,并求 $(\bm A - \bm E)^{-1}$.
	\end{titwo}

	\begin{titwo}
		已知 $\bm A, \bm B$ 是 $3$ 阶方阵,$\bm A \ne \bm O, \bm A \bm B = \bm O$,证明:$\bm B$ 不可逆.
	\end{titwo}

	\begin{titwo}
		设 $\bm A, \bm B$ 均为 $n$ 阶矩阵,且 $\bm A \bm B = \bm A + \bm B$,则下列命题中:

		{\raggedright
		\begin{tabular}{l}
			\circled{1} 若 $\bm A$ 可逆,则 $\bm B$ 可逆;\\
			\circled{2} 若 $\bm A + \bm B$ 可逆,则 $\bm B$ 可逆;\\
			\circled{3} 若 $\bm B$ 可逆,则 $\bm A + \bm B$ 可逆;\\
			\circled{4} $\bm A - \bm E$ 恒可逆.
		\end{tabular}}

		\noindent 正确的个数为\kuo.

		\fourch{$1$}{$2$}{$3$}{$4$}
	\end{titwo}

	\begin{titwo}
		设 $3$ 阶方阵 $\bm A, \bm B$ 满足关系式 $\bm A^{-1} \bm B \bm A = 6 \bm A + \bm B \bm A$,且 $\bm A = \begin{bsmallmatrix}
			\frac{1}{3} & 0 & 0 \\
			0 & \frac{1}{4} & 0 \\
			0 & 0 & \frac{1}{7}
		\end{bsmallmatrix}$,则 $\bm B = $\htwo.
	\end{titwo}

	\begin{titwo}
		已知 $\bm A^{2} - 2 \bm A + \bm E = \bm O$,则 $(\bm A + \bm E)^{-1} = $\htwo.
	\end{titwo}

	\begin{titwo}
		设 $\bigl( 2 \bm E - \bm C^{-1} \bm B \bigr) \bm A^{\TT} = \bm C^{-1}$,其中 $\bm E$ 是 $4$ 阶单位矩阵,$\bm A^{\TT}$ 是 $4$ 阶矩阵 $\bm A$ 的转置矩阵,且
		\[
			\bm B = \begin{bsmallmatrix}
				1 & 2 & -3 & -2 \\
				0 & 1 & 2 & -3 \\
				0 & 0 & 1 & 2 \\
				0 & 0 & 0 & 1
			\end{bsmallmatrix},
			\bm C = \begin{bsmallmatrix}
				1 & 2 & 0 & 1 \\
				0 & 1 & 2 & 0 \\
				0 & 0 & 1 & 2 \\
				0 & 0 & 0 & 1
			\end{bsmallmatrix},
		\]
		求 $\bm A$.
	\end{titwo}

	\begin{titwo}
		设 $\bm A$ 是主对角元素为 $0$ 的 $4$ 阶实对称矩阵,$\bm E$ 是 $4$ 阶单位矩阵,$\bm B = \begin{bsmallmatrix}
			0 &  &  &  \\
			 & 0 &  &  \\
			 &  & 2 &  \\
			 &  &  & 2 
		\end{bsmallmatrix}$,且 $\bm E + \bm A \bm B$ 是不可逆的对称矩阵,求 $\bm A$.
	\end{titwo}

	\begin{titwo}
		设矩阵 $\bm A$ 的伴随矩阵 $\bm A^{\astt} = \begin{bsmallmatrix}
			1 & 0 & 0 & 0 \\
			0 & 1 & 0 & 0 \\
			1 & 0 & 1 & 0 \\
			0 & -3 & 0 & 8
		\end{bsmallmatrix}$,且
		\[
			\bm A \bm B \bm A^{-1} = \bm B \bm A^{-1} + 3 \bm E,
		\]
		求 $\bm B$.
	\end{titwo}

	\begin{titwo}
		已知 $\bm Q = \begin{bsmallmatrix}
			1 & 2 & 3 \\
			2 & 4 & t \\
			3 & 6 & 9
		\end{bsmallmatrix}, \bm P$ 为 $3$ 阶非零矩阵,且满足 $\bm P \bm Q = \bm O$,则\kuo.

		\onech{当 $t = 6$ 时,$\bm P$ 的秩必为 $1$}{当 $t = 6$ 时,$\bm P$ 的秩必为 $2$}{当 $t \ne 6$ 时,$\bm P$ 的秩必为 $1$}{当 $t \ne 6$ 时,$\bm P$ 的秩必为 $2$}
	\end{titwo}

	\begin{titwo}
		设 $\bm A = \begin{bsmallmatrix}
			1 & 1 & 1 & 1 \\
			0 & 1 & -1 & a \\
			2 & 3 & a & 4 \\
			3 & 5 & 1 & 9
		\end{bsmallmatrix}$,若 $r(\bm A^{\astt}) = 1$,则 $a = $\kuo.

		\twoch{$1$}{$3$}{$1$ 或 $3$}{无法确定}
	\end{titwo}

	\begin{titwo}
		设 $n(n \geq 3)$ 阶矩阵 $\bm A = \begin{bsmallmatrix}
			1 & a & \cdots & a \\
			a & 1 & \cdots & a \\
			\vdots & \vdots &  & \vdots \\
			a & a & \cdots & 1
		\end{bsmallmatrix}$,若矩阵 $\bm A$ 的秩为 $n - 1$,则 $a$ 必为\kuo.

		\fourch{$1$}{$\frac{1}{1 - n}$}{$-1$}{$\frac{1}{n - 1}$}
	\end{titwo}

	\begin{titwo}
		设 $\bm A$ 是 $5$ 阶方阵,且 $\bm A^{2} = \bm O$,则 $r(\bm A^{\astt}) = $\htwo.
	\end{titwo}

	\begin{titwo}
		设有两个非零矩阵
		\[
			\bm A = [a_{1},a_{2},\cdots,a_{n}]^{\TT}, \bm B = [b_{1},b_{2},\cdots,b_{n}]^{\TT}.
		\]
		\begin{enumerate}
			\item 计算 $\bm A \bm B^{\TT}$ 与 $\bm A^{\TT} \bm B$;
			\item 求矩阵 $\bm A \bm B^{\TT}$ 的秩 $r(\bm A \bm B^{\TT})$;
			\item 设 $\bm C = \bm E - \bm A \bm B^{\TT}$,其中 $\bm E$ 为 $n$ 阶单位矩阵. 证明:
			\[
				\bm C^{\TT} \bm C = \bm E - \bm B \bm A^{\TT} - \bm A \bm B^{\TT} + \bm B \bm B^{\TT}
			\]
			的充要条件是 $\bm A^{\TT} \bm A = 1$.
		\end{enumerate}
	\end{titwo}

	\begin{titwo}
		已知 $\bm A$ 是 $m \times n$ 矩阵,$r(\bm A) = r < \min\{ m,n \}$,则 $\bm A$ 中\kuo.

		\onech{没有等于零的 $r - 1$ 阶子式,至少有一个不为零的 $r$ 阶子式}{有不等于零的 $r$ 阶子式,所有 $r + 1$ 阶子式全为零}{有等于零的 $r$ 阶子式,没有不等于零的 $r + 1$ 阶子式}{所有 $r$ 阶子式不等于零,所有 $r + 1$ 阶子式全为零}
	\end{titwo}

	\begin{titwo}
		设 $\bm A$ 是 $n$ 阶实矩阵,证明:$\tr(\bm A \bm A^{\TT}) = 0$ 的充分必要条件是 $\bm A = \bm O$.
	\end{titwo}

	\begin{titwo}
		设 $\bm A = (a_{ij})_{n \times n}$,且 $\sum_{j=1}^{n} a_{ij} = 0, i = 1,2,\cdots,n$,求 $r(\bm A^{\astt})$ 及 $\bm A^{\astt}$ 的表示形式.
	\end{titwo}

	\begin{titwo}
		设 $\bm A, \bm B$ 均是 $3$ 阶非零矩阵,满足 $\bm A \bm B = \bm O$,其中 $\bm B = \begin{bsmallmatrix}
			1 & -1 & 1 \\
			2a & 1 - a & 2a \\
			a & - a & a^{2} - 2
		\end{bsmallmatrix}$,则\kuo.

		\onech{$a = -1$ 时,必有 $r(\bm A) = 1$}{$a \ne -1$ 时,必有 $r(\bm A) = 2$}{$a = 2$ 时,必有 $r(\bm A) = 1$}{$a \ne 2$ 时,必有 $r(\bm A) = 2$}
	\end{titwo}
\section{向量组的线性相关和线性无关}

	\begin{titwo}
		$n$ 维向量组 $\bm \alpha_{1}, \bm \alpha_{2}, \cdots, \bm \alpha_{s} (3 \leq s \leq n)$ 线性无关的充要条件是\kuo.

		\onech{存在一组全为零的数 $k_{1},k_{2},\cdots,k_{s}$,使 $k_{1} \bm \alpha_{1} + k_{2} \bm \alpha_{2} + \cdots + k_{s} \bm \alpha_{s} = \bm 0$}{$\bm \alpha_{1}, \bm \alpha_{2}, \cdots, \bm \alpha_{s}$ 中任意两个向量都线性无关}{$\bm \alpha_{1}, \bm \alpha_{2}, \cdots, \bm \alpha_{s}$ 中任意一个向量都不能由其余向量线性表出}{存在一组不全为零的数 $k_{1},k_{2},\cdots,k_{s}$,使 $k_{1} \bm \alpha_{1} + k_{2} \bm \alpha_{2} + \cdots + k_{s} \bm \alpha_{s} \ne \bm 0$}
	\end{titwo}

	\begin{titwo}
		已知向量组 $\bm \alpha_{1} , \bm \alpha_{2} , \bm \alpha_{3} , \bm \alpha_{4}$ 线性无关,则向量组 $2 \bm \alpha_{1} + \bm \alpha_{3} + \bm \alpha_{4}, \bm \alpha_{2} - \bm \alpha_{4}, \bm \alpha_{3} + \bm \alpha_{4}, \bm \alpha_{2} + \bm \alpha_{3}, 2 \bm \alpha_{1} + \bm \alpha_{2} + \bm \alpha_{3}$ 的秩是\kuo.

		\fourch{$1$}{$2$}{$3$}{$4$}
	\end{titwo}

	\begin{titwo}
		已知 $3$ 维向量组 $\bm \alpha_{1} , \bm \alpha_{2} , \bm \alpha_{3}$ 线性无关,则向量组 $\bm \alpha_{1} - \bm \alpha_{2}, \bm \alpha_{2} - k \bm \alpha_{3}, \bm \alpha_{3} - \bm \alpha_{1}$ 也线性无关的充要条件是\htwo.
	\end{titwo}

	\begin{titwo}
		设有两个 $n$ 维向量组
		\begin{align*}
			(\text{\Rmnum{1}})&\bm \alpha_{1}, \bm \alpha_{2}, \cdots, \bm \alpha_{s},\\
			(\text{\Rmnum{2}})&\bm \beta_{1}, \bm \beta_{2}, \cdots, \bm \beta_{s},
		\end{align*}
		若存在两组不全为零的数 $k_{1},k_{2},\allowbreak \cdots,\allowbreak k_{s},\allowbreak \lambda_{1},\allowbreak \lambda_{2},\allowbreak \cdots,\allowbreak \lambda_{s}$,使 $(k_{1} + \lambda_{1}) \bm \alpha_{1} + (k_{2} + \lambda_{2}) \bm \alpha_{2} + \cdots + (k_{s} + \lambda_{s}) \bm \alpha_{s} + (k_{1} - \lambda_{1}) \bm \beta_{1} + \cdots + (k_{s} - \lambda_{s}) \bm \beta_{s} = \bm 0$,则\kuo.

		\onech{$\bm \alpha_{1} + \bm \beta_{1}, \cdots, \bm \alpha_{s} + \bm \beta_{s}, \bm \alpha_{1} - \bm \beta_{1}, \cdots, \bm \alpha_{s} - \bm \beta_{s}$ 线性相关}{$\bm \alpha_{1} + \bm \beta_{1}, \cdots, \bm \alpha_{s} + \bm \beta_{s}, \bm \alpha_{1} - \bm \beta_{1}, \cdots, \bm \alpha_{s} - \bm \beta_{s}$ 线性无关}{$\bm \alpha_{1}, \cdots, \bm \alpha_{s}$ 及 $\bm \beta_{1}, \cdots, \bm \beta_{s}$ 均线性相关}{$\bm \alpha_{1}, \cdots, \bm \alpha_{s}$ 及 $\bm \beta_{1}, \cdots, \bm \beta_{s}$ 均线性无关}
	\end{titwo}

	\begin{titwo}
		已知 $n$ 维向量组 $\bm \alpha_{1}, \bm \alpha_{2}, \cdots, \bm \alpha_{s}$ 线性无关,则向量组 $\bm \alpha_{1}', \bm \alpha_{2}', \cdots, \bm \alpha_{s}'$ 可能线性相关的是\kuo.

		\onech{$\bm \alpha_{i}'(i = 1,2,\cdots,s)$ 是 $\bm \alpha_{i}(i = 1,2,\cdots,s)$ 中第一个分量加到第 $2$ 个分量得到的向量}{$\bm \alpha_{i}'(i = 1,2,\cdots,s)$ 是 $\bm \alpha_{i}(i = 1,2,\cdots,s)$ 中第一个分量改变成其相反数的向量}{$\bm \alpha_{i}'(i = 1,2,\cdots,s)$ 是 $\bm \alpha_{i}(i = 1,2,\cdots,s)$ 中第一个分量改为 $0$ 的向量}{$\bm \alpha_{i}'(i = 1,2,\cdots,s)$ 是 $\bm \alpha_{i}(i = 1,2,\cdots,s)$ 中第 $n$ 个分量后再增添一个分量的向量}
	\end{titwo}

	\begin{titwo}
		设向量组 $\bm \alpha_{1}, \bm \alpha_{2}, \cdots, \bm \alpha_{s}(s \geq 2)$ 线性无关,且
		\begin{gather*}
			\bm \beta_{1} = \bm \alpha_{1} + \bm \alpha_{2},
			\bm \beta_{2} = \bm \alpha_{2} + \bm \alpha_{3},\\
			\cdots,\\
			\bm \beta_{s - 1} = \bm \alpha_{s - 1} + \bm \alpha_{s},
			\bm \beta_{s} = \bm \alpha_{s} + \bm \alpha_{1}.
		\end{gather*}
		讨论向量组 $\bm \beta_{1}, \bm \beta_{2}, \cdots, \bm \beta_{s}$ 的线性相关性.
	\end{titwo}

	\begin{titwo}
        已知向量组 $\bm \alpha_{1}, \bm \alpha_{2}, \cdots, \bm \alpha_{s + 1}(s > 1)$ 线性无关,
        \[
            \bm \beta_{i} = \bm \alpha_{i} + t \bm \alpha_{i+1}, i = 1,2,\cdots,s.
        \]
        证明:向量组 $\bm \beta_{1},\bm \beta_{2},\cdots,\bm \beta_{s}$ 线性无关.
	\end{titwo}

	\begin{titwo}
		设 $\bm A$ 是 $3 \times 3$ 矩阵,$\bm \alpha_{1},\bm \alpha_{2},\bm \alpha_{3}$ 是 $3$ 维列向量,且线性无关,已知
		\[
			\bm A \bm \alpha_{1} = \bm \alpha_{2} + \bm \alpha_{3},
			\bm A \bm \alpha_{2} = \bm \alpha_{1} + \bm \alpha_{3},
			\bm A \bm \alpha_{3} = \bm \alpha_{1} + \bm \alpha_{2}.
		\]
		\begin{enumerate}
			\item 证明 $\bm A \bm \alpha_{1},\bm A \bm \alpha_{2},\bm A \bm \alpha_{3}$ 线性无关;
			\item 求 $|\bm A|$.
		\end{enumerate}
	\end{titwo}

	\begin{titwo}
		已知 $\bm A$ 是 $n$ 阶矩阵,$\bm \alpha_{1}, \bm \alpha_{2}, \cdots, \bm \alpha_{s}$ 是 $n$ 维线性无关向量组,若 $\bm A \bm \alpha_{1}, \bm A \bm \alpha_{2}, \cdots, \bm A \bm \alpha_{s}$ 线性相关. 证明:$\bm A$ 不可逆.
	\end{titwo}

	\begin{titwo}
		设 $\bm A$ 是 $n \times m$ 矩阵,$\bm B$ 是 $m \times n$ 矩阵,$\bm E$ 是 $n$ 阶单位矩阵. 若 $\bm A \bm B = \bm E$,证明:$\bm B$ 的列向量组线性无关.
	\end{titwo}

	\begin{titwo}
		设 $\bm A$ 为 $n$ 阶正定矩阵,$\bm \alpha_{1}, \bm \alpha_{2}, \cdots, \bm \alpha_{n}$ 为 $n$ 维非零列向量,且满足
		\[
			\bm \alpha_{i}^{\TT} \bm A^{-1} \bm \alpha_{j} = 0(i \ne j; i,j = 1,2,\cdots,n).
		\]
		试证:向量组 $\bm \alpha_{1}, \bm \alpha_{2}, \cdots, \bm \alpha_{n}$ 线性无关.
	\end{titwo}

	\begin{titwo}
		设 $\bm A, \bm B, \bm C$ 均是 $3$ 阶矩阵,满足 $\bm A \bm B = - 2 \bm B,\allowbreak  \bm C \bm A^{\TT} = 2 \bm C$. 其中
		\[
			\bm B = \begin{bsmallmatrix}
				1 & 2 & 3 \\
				-1 & 1 & 0 \\
				2 & -1 & 1
			\end{bsmallmatrix},
			\bm C = \begin{bsmallmatrix}
				1 & -2 & 1 \\
				-2 & 4 & -2 \\
				-1 & 2 & -1
			\end{bsmallmatrix}.
		\]
		\begin{enumerate}
			\item 求 $\bm A$;
			\item 证明:对任何 $3$ 维向量 $\bm \xi$,$\bm A^{100} \bm \xi$ 与 $\bm \xi$ 必线性相关.
		\end{enumerate}
	\end{titwo}
\subsection{向量组的线性表示}

	\begin{titwo}
		设向量组
		\[
			(\text{\Rmnum{1}})\bm \alpha_{1}, \bm \alpha_{2}, \cdots, \bm \alpha_{s}
		\]
		线性无关,(\Rmnum{2})$\bm \beta_{1}, \bm \beta_{2}, \cdots, \bm \beta_{t}$ 线性无关,且 $\bm \alpha_{i}(i = 1,2,\cdots,s)$ 不能由(\Rmnum{2})$\bm \beta_{1}, \bm \beta_{2}, \cdots, \bm \beta_{t}$ 线性表出,$\bm \beta_{j}(j = 1,2,\cdots,t)$ 不能由(\Rmnum{1})$\bm \alpha_{1}, \bm \alpha_{2}, \cdots, \bm \alpha_{s}$ 线性表出,则向量组 $\bm \alpha_{1}, \bm \alpha_{2}, \cdots, \bm \alpha_{s}, \bm \beta_{1}, \bm \beta_{2}, \cdots, \bm \beta_{t}$\kuo.

		\onech{必线性相关}{必线性无关}{可能线性相关,也可能线性无关}{以上都不正确}
	\end{titwo}

	\begin{titwo}
		设
		\begin{gather*}
			\bm \alpha_{1} = [1,0,-1,2]^{\TT}, \bm \alpha_{2} = [2,-1,-2,6]^{\TT},\\
			\bm \alpha_{3} = [3,1,t,4]^{\TT}, \bm \beta = [4,-1,-5,10]^{\TT},
		\end{gather*}
		已知 $\bm \beta$ 不能由 $\bm \alpha_{1}, \bm \alpha_{2}, \bm \alpha_{3}$ 线性表出,则 $t = $\htwo.
	\end{titwo}

	\begin{titwo}
		已知
		\begin{gather*}
			\bm \alpha_{1} = [1,-1,1]^{\TT},\bm \alpha_{2} = [1,t,-1]^{\TT},\\
			\bm \alpha_{3} = [t,1,2]^{\TT},\bm \beta = \bigl[4,t^{2},-4\bigr]^{\TT},
		\end{gather*}
		若 $\bm \beta$ 可由 $\bm \alpha_{1},\bm \alpha_{2},\bm \alpha_{3}$ 线性表示,且表示法不唯一,求 $t$ 及 $\bm \beta$ 的表达式.
	\end{titwo}

	\begin{titwo}
		已知 $\bm \alpha_{1},\bm \alpha_{2},\bm \alpha_{3},\bm \alpha_{4}$ 为 $3$ 维非零列向量,则下列结论:\\
		\circled{1}如果 $\bm \alpha_{4}$ 不能由 $\bm \alpha_{1},\bm \alpha_{2},\bm \alpha_{3}$ 线性表出,则 $\bm \alpha_{1},\bm \alpha_{2},\bm \alpha_{3}$ 线性相关;\\
		\circled{2}如果 $\bm \alpha_{1},\bm \alpha_{2},\bm \alpha_{3}$ 线性相关,$\bm \alpha_{2},\bm \alpha_{3},\bm \alpha_{4}$ 线性相关,则 $\bm \alpha_{1},\bm \alpha_{2},\bm \alpha_{4}$ 也线性相关;\\
		\circled{3}如果 $r(\bm \alpha_{1}, \bm \alpha_{1} + \bm \alpha_{2}, \bm \alpha_{2} + \bm \alpha_{3}) = r(\bm \alpha_{4}, \bm \alpha_{1} + \bm \alpha_{4}, \bm \alpha_{2} + \bm \alpha_{4}, \bm \alpha_{3} + \bm \alpha_{4})$,则 $\bm \alpha_{4}$ 可以由 $\bm \alpha_{1},\bm \alpha_{2},\bm \alpha_{3}$ 线性表出.\\
		其中正确的个数为\kuo.

		\fourch{$0$}{$1$}{$2$}{$3$}
	\end{titwo}

	\begin{titwo}
		向量组(\Rmnum{1}) $\bm \alpha_{1}, \bm \alpha_{2}, \cdots, \bm \alpha_{s}$,其秩为 $r_{1}$,向量组(\Rmnum{2}) $\bm \beta_{1}, \bm \beta_{2}, \cdots, \bm \beta_{s}$,其秩为 $r_{2}$,且 $\bm \beta_{i} (i = 1,2,\cdots,s)$ 均可由向量组(\Rmnum{1})$\bm \alpha_{1}, \bm \alpha_{2}, \cdots, \bm \alpha_{s}$ 线性表出,则必有\kuo.

		\onech{$\bm \alpha_{1} + \bm \beta_{1}, \bm \alpha_{2} + \bm \beta_{2}, \cdots, \bm \alpha_{s} + \bm \beta_{s}$ 的秩为 $r_{1} + r_{2}$}{$\bm \alpha_{1} - \bm \beta_{1}, \bm \alpha_{2} - \bm \beta_{2}, \cdots, \bm \alpha_{s} - \bm \beta_{s}$ 的秩为 $r_{1} - r_{2}$}{$\bm \alpha_{1}, \bm \alpha_{2}, \cdots, \bm \alpha_{s}, \bm \beta_{1}, \bm \beta_{2}, \cdots, \bm \beta_{s}$ 的秩为 $r_{1} + r_{2}$}{$\bm \alpha_{1}, \bm \alpha_{2}, \cdots, \bm \alpha_{s}, \bm \beta_{1}, \bm \beta_{2}, \cdots, \bm \beta_{s}$ 的秩为 $r_{1}$}
	\end{titwo}

	\begin{titwo}
		已知向量组
		\[
			\bm \alpha_{1} = \begin{bsmallmatrix}
				1 \\
				-1 \\
				2
			\end{bsmallmatrix},
			\bm \alpha_{2} = \begin{bsmallmatrix}
				0 \\
				3 \\
				1
			\end{bsmallmatrix},
			\bm \alpha_{3} = \begin{bsmallmatrix}
				3 \\
				0 \\
				7
			\end{bsmallmatrix}
		\]
		与向量组
		\[
			\bm \beta_{1} = \begin{bsmallmatrix}
				1 \\
				-2 \\
				2
			\end{bsmallmatrix},
			\bm \beta_{2} = \begin{bsmallmatrix}
				2 \\
				1 \\
				5
			\end{bsmallmatrix},
			\bm \beta_{3} = \begin{bsmallmatrix}
				x \\
				3 \\
				3
			\end{bsmallmatrix}
		\]
		等秩,则 $x = $\htwo.
	\end{titwo}

	\begin{titwo}
		已知 $\bm \alpha_{1} = [1,2,-3,1]^{\TT}, \bm \alpha_{2} = [5,-5,a,11]^{\TT}, \bm \alpha_{3} = [1,-3,6,3]^{\TT}, \bm \alpha_{4} = [2,-1,3,a]^{\TT}$. 问:
		\begin{enumerate}
			\item 当 $a$ 为何值时,向量组 $\bm \alpha_{1},\bm \alpha_{2},\bm \alpha_{3},\bm \alpha_{4}$ 线性相关;
			\item 当 $a$ 为何值时,向量组 $\bm \alpha_{1},\bm \alpha_{2},\bm \alpha_{3},\bm \alpha_{4}$ 线性无关;
			\item 当 $a$ 为何值时,$\bm \alpha_{4}$ 能由 $\bm \alpha_{1},\bm \alpha_{2},\bm \alpha_{3}$ 线性表出,并写出它的表出式.
		\end{enumerate}
	\end{titwo}

	\begin{titwo}
		已知
		\[
			\bm \alpha_{1} = \begin{bsmallmatrix}
				1 + \lambda \\
				1 \\
				1
			\end{bsmallmatrix},
			\bm \alpha_{2} = \begin{bsmallmatrix}
				1 \\
				1 + \lambda \\
				1
			\end{bsmallmatrix},
			\bm \alpha_{3} = \begin{bsmallmatrix}
				1 \\
				1 \\
				1 + \lambda
			\end{bsmallmatrix},
			\bm \beta = \begin{bsmallmatrix}
				0 \\
				\lambda \\
				\lambda^{2}
			\end{bsmallmatrix}.
		\]
		问 $\lambda$ 取何值时,有:
		\begin{enumerate}
			\item $\beta$ 可由 $\bm \alpha_{1},\bm \alpha_{2},\bm \alpha_{3}$ 线性表出,且表达式唯一;
			\item $\beta$ 可由 $\bm \alpha_{1},\bm \alpha_{2},\bm \alpha_{3}$ 线性表出,但表达式不唯一;
			\item $\beta$ 不能由 $\bm \alpha_{1},\bm \alpha_{2},\bm \alpha_{3}$ 线性表出.
		\end{enumerate}
	\end{titwo}

	\begin{titwo}
		已知 $\bm \alpha_{1}, \bm \alpha_{2}, \cdots, \bm \alpha_{s}$ 线性无关,$\beta$ 可由 $\bm \alpha_{1}, \bm \alpha_{2},$ $\cdots, \bm \alpha_{s}$ 线性表出,且表达式的系数全不为零. 证明:$\bm \alpha_{1}, \bm \alpha_{2}, \cdots, \bm \alpha_{s},\beta$ 中任意 $s$ 个向量均线性无关.
	\end{titwo}
\subsection{向量组的等价}
	
	\begin{titwo}
		已知向量组(\Rmnum{1}) $\bm \alpha_{1},\bm \alpha_{2},\bm \alpha_{3},\bm \alpha_{4}$ 线性无关,则与(\Rmnum{1})等价的向量组是\kuo.

		\onech{$\bm \alpha_{1} + \bm \alpha_{2}, \bm \alpha_{2} + \bm \alpha_{3}, \bm \alpha_{3} + \bm \alpha_{4}, \bm \alpha_{4} + \bm \alpha_{1}$}{$\bm \alpha_{1} - \bm \alpha_{2}, \bm \alpha_{2} - \bm \alpha_{3}, \bm \alpha_{3} - \bm \alpha_{4}, \bm \alpha_{4} - \bm \alpha_{1}$}{$\bm \alpha_{1} + \bm \alpha_{2}, \bm \alpha_{2} - \bm \alpha_{3}, \bm \alpha_{3} + \bm \alpha_{4}, \bm \alpha_{4} - \bm \alpha_{1}$}{$\bm \alpha_{1} + \bm \alpha_{2}, \bm \alpha_{2} - \bm \alpha_{3}, \bm \alpha_{3} - \bm \alpha_{4}, \bm \alpha_{4} - \bm \alpha_{1}$}
	\end{titwo}

	\begin{titwo}
		已知向量组(\Rmnum{1})与向量组(\Rmnum{2}),若(\Rmnum{1})可由(\Rmnum{2})线性表示,且 $r(\text{\Rmnum{1}}) = r(\text{\Rmnum{2}}) = r$. 证明:(\Rmnum{1})与(\Rmnum{2})等价.
	\end{titwo}

	\begin{titwo}
		设 $n$ 维列向量组 $\bm \alpha_{1},\bm \alpha_{2},\cdots,\bm \alpha_{m}(m < n)$ 线性无关,则 $n$ 维列向量组 $\bm \beta_{1},\bm \beta_{2},\cdots,\bm \beta_{m}$ 线性无关的充分必要条件为\kuo.

		\onech{向量组 $\bm \alpha_{1},\bm \alpha_{2},\cdots,\bm \alpha_{m}$ 可由向量组 $\bm \beta_{1},\bm \beta_{2},\cdots,\bm \beta_{m}$ 线性表出}{向量组 $\bm \beta_{1},\bm \beta_{2},\cdots,\bm \beta_{m}$ 可由向量组 $\bm \alpha_{1},\bm \alpha_{2},\cdots,\bm \alpha_{m}$ 线性表出}{向量组 $\bm \alpha_{1},\bm \alpha_{2},\cdots,\bm \alpha_{m}$ 与向量组 $\bm \beta_{1},\bm \beta_{2},\cdots,\bm \beta_{m}$ 等价}{矩阵 $\bm A = [\bm \alpha_{1},\bm \alpha_{2},\cdots,\bm \alpha_{m}]$ 与矩阵 $\bm B = \bigl[\bm \beta_{1},$ $\bm \beta_{2},$ $\cdots,$ $\bm \beta_{m}\bigr]$ 等价}
	\end{titwo}
\subsection{向量空间}

	\begin{titwo}
		已知 $\mathbb{R}^{3}$ 的两个基分别为
		\[
			\bm \alpha_{1} = \begin{bsmallmatrix}
				1 \\
				1 \\
				1
			\end{bsmallmatrix},
			\bm \alpha_{2} = \begin{bsmallmatrix}
				1 \\
				0 \\
				-1
			\end{bsmallmatrix},
			\bm \alpha_{3} = \begin{bsmallmatrix}
				1 \\
				0 \\
				1
			\end{bsmallmatrix}
		\]
		与
		\[
			\bm \beta_{1} = \begin{bsmallmatrix}
				1 \\
				2 \\
				1
			\end{bsmallmatrix},
			\bm \beta_{2} = \begin{bsmallmatrix}
				2 \\
				3 \\
				4
			\end{bsmallmatrix},
			\bm \beta_{3} = \begin{bsmallmatrix}
				3 \\
				4 \\
				3
			\end{bsmallmatrix},
		\]
		求由基 $\bm \alpha_{1},\bm \alpha_{2},\bm \alpha_{3}$ 到基 $\bm \beta_{1},\bm \beta_{2},\bm \beta_{3}$ 的过渡矩阵 $\bm P$.
	\end{titwo}

	\begin{titwo}
		设 $\bm \alpha_{1} = [1,0,1]^{\TT},\bm \alpha_{2} = [1,1,-1]^{\TT},\bm \alpha_{3} = [1,-1,1]^{\TT};$ $\bm \beta_{1} = [3,0,1]^{\TT}, \bm \beta_{2} = [2,0,0]^{\TT}, \bm \beta_{3} = [0,2,-2]^{\TT}$ 是 $\mathbb{R}^{3}$ 的两个基. 若向量 $\bm \xi$ 在基 $\bm \beta_{1},\bm \beta_{2},\bm \beta_{3}$ 下的坐标为 $[1,2,0]^{\TT}$,则 $\bm \xi$ 在基 $\bm \alpha_{1},\bm \alpha_{2},\bm \alpha_{3}$ 下的坐标为\kuo.

		\twoch{$[1,3,3]^{\TT}$}{$[-1,3,3]^{\TT}$}{$[-1,-3,3]^{\TT}$}{$[-1,3,-3]^{\TT}$}
	\end{titwo}

	\begin{titwo}
		设 $\mathbb{R}^{3}$ 中两个基
		\begin{gather*}
			\bm \alpha_{1} = [1,1,0]^{\TT}, \bm \alpha_{2} = [0,1,1]^{\TT}, \bm \alpha_{3} = [1,0,1]^{\TT}; \\
			\bm \beta_{1} = [1,0,0]^{\TT}, \bm \beta_{2} = [1,1,0]^{\TT}, \bm \beta_{3} = [1,1,1]^{\TT}.
		\end{gather*}
		\begin{enumerate}
			\item 求 $\bm \beta_{1},\bm \beta_{2},\bm \beta_{3}$ 到 $\bm \alpha_{1},\bm \alpha_{2},\bm \alpha_{3}$ 的过渡矩阵;
			\item 已知 $\bm \xi$ 在基 $\bm \beta_{1},\bm \beta_{2},\bm \beta_{3}$ 下的坐标为 $[1,0,2]^{\TT}$,求 $\bm \xi$ 在基 $\bm \alpha_{1},\bm \alpha_{2},\bm \alpha_{3}$ 下的坐标;
			\item 求在上述两个基下有相同坐标的向量.
		\end{enumerate}
	\end{titwo}
\subsection{方程组}

	\begin{titwo}
		设 $\bm \alpha_{1},\bm \alpha_{2},\bm \alpha_{3}$ 均为线性方程组 $\bm A \bm x = \bm b$ 的解,则下列向量
		\[
			\bm \alpha_{1} - \bm \alpha_{2},
			\bm \alpha_{1} - 2\bm \alpha_{2} + \bm \alpha_{3},
			\frac{1}{4}( \bm \alpha_{1} - \bm \alpha_{3} ),
			\bm \alpha_{1} + 3 \bm \alpha_{2} - 4 \bm \alpha_{3},
		\]
		其中是相应的齐次方程组 $\bm A \bm x = \bm 0$ 的解向量的个数为\kuo.

		\fourch{$4$}{$3$}{$2$}{$1$}
	\end{titwo}

	\begin{titwo}
		设 $\bm A$ 是秩为 $n - 1$ 的 $n$ 阶矩阵,$\bm \alpha_{1},\bm \alpha_{2}$ 是方程组 $\bm A \bm x = \bm 0$ 的两个不同的解向量,$k$ 是任意常数,则 $\bm A \bm x = \bm 0$ 的通解必定是\kuo.

		\twoch{$\bm \alpha_{1} + \bm \alpha_{2}$}{$k \bm \alpha_{1}$}{$k ( \bm \alpha_{1} + \bm \alpha_{2} )$}{$k ( \bm \alpha_{1} - \bm \alpha_{2} )$}
	\end{titwo}

	\begin{titwo}
		齐次线性方程组的系数矩阵 $\bm A_{4 \times 5} = \bigl[ \bm \beta_{1}, \bm \beta_{2},$ $\bm \beta_{3}, \bm \beta_{4}, \bm \beta_{5} \bigr]$ 经过初等行变换化成阶梯形矩阵为
		\[
			\bm A = \bigl[ \bm \beta_{1}, \bm \beta_{2}, \bm \beta_{3}, \bm \beta_{4}, \bm \beta_{5} \bigr] \xrightarrow{\text{初等行变换}} \begin{bsmallmatrix}
				1 & 2 & -1 & 5 & 2 \\
				0 & 1 & 2 & 6 & 0 \\
				0 & 0 & 0 & 4 & 0 \\
				0 & 0 & 0 & 0 & 0
			\end{bsmallmatrix},
		\]
		则\kuo.

		\onech{$\bm \beta_{1}$ 不能由 $\bm \beta_{3},\bm \beta_{4},\bm \beta_{5}$ 线性表出}{$\bm \beta_{2}$ 不能由 $\bm \beta_{1},\bm \beta_{3},\bm \beta_{5}$ 线性表出}{$\bm \beta_{3}$ 不能由 $\bm \beta_{1},\bm \beta_{2},\bm \beta_{5}$ 线性表出}{$\bm \beta_{4}$ 不能由 $\bm \beta_{1},\bm \beta_{2},\bm \beta_{3}$ 线性表出}
	\end{titwo}

	\begin{titwo}
		设 $\bm A$ 为 $m \times n$ 矩阵,则齐次线性方程组 $\bm A \bm X = \bm 0$ 仅有零解的充分条件是\kuo.

		\twoch{$\bm A$ 的列向量线性无关}{$\bm A$ 的列向量线性相关}{$\bm A$ 的行向量线性无关}{$\bm A$ 的行向量线性相关}
	\end{titwo}

	\begin{titwo}
		已知 $\bm \beta_{1},\bm \beta_{2}$ 是 $\bm A \bm X = \bm b$ 的两个不同的解,$\bm \alpha_{1},\bm \alpha_{2}$ 是相应的齐次方程组 $\bm A \bm X = \bm 0$ 的基础解系,$k_{1},k_{2}$ 是任意常数,则 $\bm A \bm X = \bm b$ 的通解是\kuo.

		\onech%
		{$k_{1} \bm \alpha_{1} + k_{2} ( \bm \alpha_{1} + \bm \alpha_{2} ) + \frac{\bm \beta_{1} - \bm \beta_{2}}{2}$}%
		{$k_{1} \bm \alpha_{1} + k_{2} ( \bm \alpha_{1} - \bm \alpha_{2} ) + \frac{\bm \beta_{1} + \bm \beta_{2}}{2}$}%
		{$k_{1} \bm \alpha_{1} + k_{2} ( \bm \beta_{1} - \bm \beta_{2} ) + \frac{\bm \beta_{1} - \bm \beta_{2}}{2}$}%
		{$k_{1} \bm \alpha_{1} + k_{2} ( \bm \beta_{1} - \bm \beta_{2} ) + \frac{\bm \beta_{1} + \bm \beta_{2}}{2}$}
	\end{titwo}

	\begin{titwo}
		设 $\bm \alpha_{1},\bm \alpha_{2},\bm \alpha_{3}$ 是四元非齐次线性方程组 $\bm A \bm X = \bm b$ 的三个解向量,且 $r(\bm A) = 3,\bm \alpha_{1}  = [1,2,3,4]^{\TT},\bm \alpha_{2} + \bm \alpha_{3} = [0,1,2,3]^{\TT},k$ 是任意常数,则方程组 $\bm A \bm X = \bm b$ 的通解是\kuo.

		\twoch%
		{$\begin{bsmallmatrix}
			1 \\
			2 \\
			3 \\
			4
		\end{bsmallmatrix} + k \begin{bsmallmatrix}
			1 \\
			1 \\
			1 \\
			1
		\end{bsmallmatrix}$}
		{$\begin{bsmallmatrix}
			1 \\
			2 \\
			3 \\
			4
		\end{bsmallmatrix} + k \begin{bsmallmatrix}
			0 \\
			1 \\
			2 \\
			3
		\end{bsmallmatrix}$}%
		{$\begin{bsmallmatrix}
			1 \\
			2 \\
			3 \\
			4
		\end{bsmallmatrix} + k \begin{bsmallmatrix}
			2 \\
			3 \\
			4 \\
			5
		\end{bsmallmatrix}$}%
		{$\begin{bsmallmatrix}
			1 \\
			2 \\
			3 \\
			4
		\end{bsmallmatrix} + k \begin{bsmallmatrix}
			3 \\
			4 \\
			5 \\
			6
		\end{bsmallmatrix}$}
	\end{titwo}

	\begin{titwo}
		设 $\bm A = \begin{bsmallmatrix}
			1 & 1 & a \\
			1 & a & 1 \\
			a & 1 & 1 \\
			2 & a + 1 & a + 3
		\end{bsmallmatrix},\bm B$ 是 $3$ 阶非零矩阵,且 $\bm A \bm B = \bm O$,则 $\bm A \bm x = \bm 0$ 的通解是\htwo.
	\end{titwo}

	\begin{titwo}
		求齐次线性方程组 $\begin{cases}
			x_{1} + x_{2} + x_{5} = 0,\\
			x_{1} + x_{2} - x_{3} = 0,\\
			x_{3} + x_{4} + x_{5} = 0
		\end{cases}$ 的基础解系.
	\end{titwo}

	\begin{titwo}
		已知 $r(\bm A) = r_{1}$,且方程组 $\bm A \bm X = \bm \alpha$ 有解,$r(\bm B) = r_{2}$,且 $\bm B \bm Y = \bm \beta$ 无解,设
		\[
			\bm A = [\bm \alpha_{1},\bm \alpha_{2},\cdots,\bm \alpha_{n}], \bm B = \bigl[\bm \beta_{1},\bm \beta_{2},\cdots,\bm \beta_{n}\bigr],
		\]
		且 $r\bigl(\bm \alpha_{1},\bm \alpha_{2},\cdots,\bm \alpha_{n},\bm \alpha,\bm \beta_{1},\bm \beta_{2},\cdots,\bm \beta_{n},\bm \beta\bigr) = r$,则\kuo.

		\twoch{$r = r_{1} + r_{2}$}{$r > r_{1} + r_{2}$}{$r = r_{1} + r_{2} + 1$}{$r \leq r_{1} + r_{2} + 1$}
	\end{titwo}

	\begin{titwo}
		设 $\bm A$ 是 $m \times n$ 矩阵,则方程组 $\bm A \bm X = \bm b$ 有唯一解的充分必要条件是\kuo.

		\onech{$m = n$ 且 $|\bm A| \ne 0$}{$\bm A \bm X = \bm 0$ 有唯一零解}{$\bm A$ 的列向量组 $\bm \alpha_{1},\bm \alpha_{2},\cdots,\bm \alpha_{n}$ 和 $\bm \alpha_{1},\bm \alpha_{2},\cdots,\bm \alpha_{n},\bm b$ 是等价向量组}{$r(\bm A) = n$,$\bm b$ 可由 $\bm A$ 的列向量线性表出}
	\end{titwo}

	\begin{titwo}
		设 $\bm A$ 是 $4 \times 5$ 矩阵,且 $\bm A$ 的行向量组线性无关,则下列说法不正确的是\kuo.
		
		\onech{$\bm A^{\TT} \bm X = \bm 0$ 只有零解}{$\bm A^{\TT} \bm A \bm X = \bm 0$ 必有无穷多解}{对任意的 $\bm b$,$\bm A^{\TT} \bm X = \bm b$ 有唯一解}{对任意的 $\bm b$,$\bm A \bm X = \bm b$ 有无穷多解}
	\end{titwo}

	\begin{titwo}
		已知 $n$ 阶矩阵 $\bm A$ 的各行元素之和均为零,且 $r(\bm A) = n - 1$,则线性方程组 $\bm A \bm X = \bm 0$ 的通解是\htwo.
	\end{titwo}

	\begin{titwo}
		已知非齐次线性方程组
		\begin{equation}\label{eq:120}
			\bm A_{3 \times 4} \bm X = \bm b
		\end{equation}
		有通解
		\[
			k_{1} [1,2,0,-2]^{\TT} + k_{2} [4,-1,-1,-1]^{\TT} + [1,0,-1,1]^{\TT},
		\]
		则满足方程组~\eqref{eq:120} 且满足条件 $x_{1} = x_{2},x_{3} = x_{4}$ 的解是\htwo.
	\end{titwo}

	\begin{titwo}
		已知 $4$ 阶方阵 $\bm A = [\bm \alpha_{1},\bm \alpha_{2},\bm \alpha_{3},\bm \alpha_{4}],$ $\bm \alpha_{1},$ $\bm \alpha_{2},$ $\bm \alpha_{3},$ $\bm \alpha_{4}$ 均为 $4$ 维列向量,其中 $\bm \alpha_{1},\bm \alpha_{2}$ 线性无关,若
		\begin{align*}
			\bm \beta &= \bm \alpha_{1} + 2 \bm \alpha_{2} - \bm \alpha_{3} \\
			&= \bm \alpha_{1} + \bm \alpha_{2} + \bm \alpha_{3} + \bm \alpha_{4} \\
			&= \bm \alpha_{1} + 3 \bm \alpha_{2} + \bm \alpha_{3} + 2 \bm \alpha_{4},
		\end{align*}
		则 $\bm A \bm x = \bm \beta$ 的通解为\htwo.
	\end{titwo}

	\begin{titwo}
		与 $\bm \alpha_{1} = [1,2,3,-1]^{\TT}, \bm \alpha_{2} = [0,1,1,2]^{\TT}, \bm \alpha_{3} = [2,1,$ $3,0]^{\TT}$ 都正交的单位向量是\htwo.
	\end{titwo}

	\begin{titwo}
		设向量组 $\bm \alpha_{1},\bm \alpha_{2},\cdots,\bm \alpha_{t}$ 是齐次线性方程组 $\bm A \bm x = \bm 0$ 的一个基础解系,向量 $\bm \beta$ 不是方程组 $\bm A \bm x = \bm 0$ 的解,即 $\bm A \bm \beta \ne \bm 0$. 证明:向量组 $\bm \beta,\bm \beta + \bm \alpha_{1},\bm \beta + \bm \alpha_{2},\cdots,\bm \beta + \bm \alpha_{t}$ 线性无关. 
	\end{titwo}

	\begin{titwo}
		设向量组
		\begin{gather*}
			\bm \alpha_{1} = [a_{11},a_{21},\cdots,a_{n1}]^{\TT},\\
			\bm \alpha_{2} = [a_{12},a_{22},\cdots,a_{n2}]^{\TT},\\
			\cdots,\\
			\bm \alpha_{s} = [a_{1s},a_{2s},\cdots,a_{ns}]^{\TT}.
		\end{gather*}
		证明:向量组 $\bm \alpha_{1},\bm \alpha_{2},\cdots,\bm \alpha_{s}$ 线性相关(线性无关)的充要条件是齐次线性方程组
		\[
			\begin{cases}
				a_{11}x_{1} + a_{12}x_{2} + \cdots + a_{1s}x_{s} = 0,\\
				a_{21}x_{1} + a_{22}x_{2} + \cdots + a_{2s}x_{s} = 0,\\
				\cdots\cdots\\
				a_{n1}x_{1} + a_{n2}x_{2} + \cdots + a_{ns}x_{s} = 0
			\end{cases}
		\]
		有非零解(唯一零解).
	\end{titwo}

	\begin{titwo}
		求下述线性方程组的解空间的维数:
		\[
			\begin{cases}
				x_{1} + 2x_{2} - 2x_{3} + 2x_{4} - x_{5} = 0,\\
				x_{1} + 2x_{2} - x_{3} + 3x_{4} - 2x_{5} = 0,\\
				2x_{1} + 4x_{2} - 7x_{3} + x_{4} + x_{5} = 0.
			\end{cases}
		\]
		并判断 $\bm \xi_{1} = [9,-1,2,-1,1]^{\TT}$ 是否属于该解空间.
	\end{titwo}

	\begin{titwo}
		已知线性方程组
		\[
			\begin{cases}
				a_{11}x_{1} + a_{12}x_{2} + a_{13}x_{3} + a_{14}x_{4} = a_{15},\\
				a_{21}x_{1} + a_{22}x_{2} + a_{23}x_{3} + a_{24}x_{4} = a_{25},\\
				a_{31}x_{1} + a_{32}x_{2} + a_{33}x_{3} + a_{34}x_{4} = a_{35},\\
				a_{41}x_{1} + a_{42}x_{2} + a_{43}x_{3} + a_{44}x_{4} = a_{45}
			\end{cases}
		\]
		的通解为 $[2,1,0,1]^{\TT} + k [1,-1,2,0]^{\TT}$. 记
		\[
			\bm \alpha_{j} = [a_{1j},a_{2j},a_{3j},a_{4j}]^{\TT}, j = 1,2,\cdots,5.
		\]
		问:
		\begin{enumerate}
			\item $\bm \alpha_{4}$ 能否由 $\bm \alpha_{1},\bm \alpha_{2},\bm \alpha_{3},\bm \alpha_{5}$ 线性表出,说明理由;
			\item $\bm \alpha_{4}$ 能否由 $\bm \alpha_{1},\bm \alpha_{2},\bm \alpha_{3}$ 线性表出,说明理由.
		\end{enumerate}
	\end{titwo}

	\begin{titwo}
		已知 $4$ 阶方阵 $\bm A = [\bm \alpha_{1},\bm \alpha_{2},\bm \alpha_{3},\bm \alpha_{4}],\bm \alpha_{1},\bm \alpha_{2},$ $\bm \alpha_{3},$ $\bm \alpha_{4}$ 均为 $4$ 维列向量,其中 $\bm \alpha_{2},\bm \alpha_{3},\bm \alpha_{4}$ 线性无关,$\bm \alpha_{1} = 2\bm \alpha_{2} - \bm \alpha_{3}$,如果 $\bm \beta = \bm \alpha_{1} + \bm \alpha_{2} + \bm \alpha_{3} + \bm \alpha_{4}$,求线性方程组 $\bm A \bm X = \bm \beta$ 的通解.
	\end{titwo}

	\begin{titwo}
		设 $\bm A_{m \times n}, r(\bm A) = m, \bm B_{n \times (n - m)}, r(\bm B) = n - m$,且满足关系式 $\bm A \bm B = \bm O$. 证明:若 $\bm \eta$ 是齐次线性方程组 $\bm A \bm X = \bm 0$ 的解,则必存在唯一的 $\bm \xi$,使得 $\bm B \bm \xi = \bm \eta$.
	\end{titwo}

	\begin{titwo}
		设三元非齐次线性方程组的系数矩阵 $\bm A$ 的秩为 $1$,已知 $\bm \eta_{1},\bm \eta_{2},\bm \eta_{3}$ 是它的三个解向量,且 $\bm \eta_{1} + \bm \eta_{2} = [1,2,3]^{\TT},\bm \eta_{2} + \bm \eta_{3} = [2,-1,1]^{\TT},\bm \eta_{3} + \bm \eta_{1} = [0,2,0]^{\TT}$,求该非齐次方程的通解.
	\end{titwo}

	\begin{titwo}
		设三元线性方程有通解
		\[
			k_{1} \begin{bsmallmatrix}
				-1\\
				3\\
				2
			\end{bsmallmatrix} + k_{2} \begin{bsmallmatrix}
				2\\
				-3\\
				1
			\end{bsmallmatrix} + \begin{bsmallmatrix}
				1\\
				-1\\
				3
			\end{bsmallmatrix},
		\]
		求原方程.
	\end{titwo}

	\begin{titwo}
		设 $\bm B$ 是秩为 $2$ 的 $5 \times 4$ 矩阵,$\bm \alpha_{1} = [1,$ $1,$ $2,$ $3]^{\TT}, \bm \alpha_{2} = [-1,1,4,-1]^{\TT}, \bm \alpha_{3} = [5,-1,-8,9]^{\TT}$ 是齐次线性方程组 $\bm B \bm x = \bm 0$ 的解向量,求 $\bm B \bm x = \bm 0$ 的解空间的一个标准正交基.
	\end{titwo}

	\begin{titwo}
		设 $\bm A$ 是 $3 \times 3$ 矩阵,$\bm \beta_{1},\bm \beta_{2},\bm \beta_{3}$ 是互不相同的 $3$ 维列向量,且都不是方程组 $\bm A \bm x = \bm 0$ 的解,记 $\bm B = \bigl[\bm \beta_{1},\bm \beta_{2},\bm \beta_{3}\bigr]$,且满足 $r(\bm A \bm B) < r(\bm A),r(\bm A \bm B) < r(\bm B)$. 则 $r(\bm A \bm B)$ 等于\kuo.
		
		\fourch{0}{1}{2}{3}
	\end{titwo}

	\begin{titwo}
		已知 $\bm \xi_{1},\bm \xi_{2},\cdots,\bm \xi_{r}(r \geq 3)$ 是 $\bm A \bm x = \bm 0$ 的基础解系,则下列向量组也是 $\bm A \bm x = \bm 0$ 的基础解系的是\kuo.
		\onech{$\bm \alpha_{1} = - \bm \xi_{2} - \bm \xi_{3} - \cdots - \bm \xi_{r},\bm \alpha_{2} = \bm \xi_{1} - \bm \xi_{3} - \bm \xi_{4} - \cdots - \bm \xi_{r},\bm \alpha_{3} = \bm \xi_{1} + \bm \xi_{2} - \bm \xi_{4} - \cdots - \bm \xi_{r},\cdots,\bm \alpha_{r} = \bm \xi_{1} + \bm \xi_{2} + \cdots + \bm \xi_{r-1}$}
		{$\bm \beta_{1} = \bm \xi_{2} + \bm \xi_{3} + \cdots + \bm \xi_{r},\bm \beta_{2} = \bm \xi_{1} + \bm \xi_{3} + \bm \xi_{4} + \cdots + \bm \xi_{r},\bm \beta_{3} = \bm \xi_{1} + \bm \xi_{2} + \bm \xi_{4} + \cdots + \bm \xi_{r},\cdots,\bm \beta_{r} = \bm \xi_{1} + \bm \xi_{2} + \cdots + \bm \xi_{r-1}$}%
		{$\bm \xi_{1},\bm \xi_{2},\cdots,\bm \xi_{r}$ 的一个等价向量组}%
		{$\bm \xi_{1},\bm \xi_{2},\cdots,\bm \xi_{r}$ 的一个等秩向量组}
	\end{titwo}

	\begin{titwo}
		设齐次线性方程组
		\[
			\begin{cases}
				a_{11} x_{1} + a_{12} x_{2} + a_{13} x_{3} + a_{14} x_{4} = 0, \\
				a_{21} x_{1} + a_{22} x_{2} + a_{23} x_{3} + a_{24} x_{4} = 0
			\end{cases}
		\]
		有基础解系 $\bm \beta_{1} = [b_{11},b_{12},b_{13},b_{14}]^{\TT},\bm \beta_{2} = [b_{21},b_{22},b_{23},$ $b_{24}]^{\TT}$,记 $\bm \alpha_{1} = [a_{11},a_{12},a_{13},a_{14}]^{\TT},\bm \alpha_{2} = [a_{21},a_{22},a_{23},$ $a_{24}]^{\TT}$.

		证明:向量组 $\bm \alpha_{1},\bm \alpha_{2},\bm \beta_{1},\bm \beta_{2}$ 线性无关.
	\end{titwo}

	\begin{titwo}
		设 $\bm A$ 是 $3$ 阶矩阵,$\bm b = [9,18,-18]^{\TT}$,方程 $\bm A \bm x = \bm b$ 有通解 $k_{1} [-2,1,0]^{\TT} + k_{2} [2,0,1]^{\TT} + [1,2,-2]^{\TT}$,其中 $k_{1},k_{2}$ 是任意常数,求 $\bm A$ 及 $\bm A^{100}$.
	\end{titwo}

	\begin{titwo}
		已知线性方程组
		\[
			\begin{cases}
				b x_{1} - a x_{2} = -2 ab, \\
				-2 cx_{2} + 3 bx_{3} = bc, \\
				cx_{1} + ax_{3} = 0,
			\end{cases}
		\]
		则\kuo.
		
		\onech{当 $a,b,c$ 为任意实数时,方程组均有解}{当 $a = 0$ 时,方程组无解}{当 $b = 0$ 时,方程组无解}{当 $c = 0$ 时,方程组无解}
	\end{titwo}

	\begin{titwo}
		设 $a_{1},a_{2},\cdots,a_{n}$ 是互不相同的实数,且
		\[
			\bm A = \begin{bsmallmatrix}
				1 & a_{1} & a_{1}^{2} & \cdots & a_{1}^{n-1} \\
				1 & a_{2} & a_{2}^{2} & \cdots & a_{2}^{n-1} \\
				\vdots & \vdots & \vdots & & \vdots \\
				1 & a_{n} & a_{n}^{2} & \cdots & a_{n}^{n-1} \\
			\end{bsmallmatrix},
			\bm X = \begin{bsmallmatrix}
				x_{1} \\
				x_{2} \\
				\vdots \\
				x_{n}
			\end{bsmallmatrix},
			\bm b = \begin{bsmallmatrix}
				1 \\
				1 \\
				\vdots \\
				1
			\end{bsmallmatrix},
		\]
		求线性方程组 $\bm A \bm X = \bm b$ 的解.
	\end{titwo}

	\begin{titwo}
		问 $\lambda$ 为何值时,线性方程组
		\[
			\begin{cases}
				x_{1} + x_{3} = \lambda, \\
				4x_{1} + x_{2} + 2x_{3} = \lambda + 2, \\
				6x_{1} + x_{2} + 4x_{3} = 2 \lambda + 3
			\end{cases}
		\]
		有解,并求出解的一般形式.
	\end{titwo}

	\begin{titwo}
		问 $\lambda$ 为何值时,方程组
		\[
			\begin{cases}
				2x_{1} + \lambda x_{2} - x_{3} = 1, \\
				\lambda x_{1} - x_{2} + x_{3} = 2, \\
				4x_{1} + 5x_{2} - 5x_{3} = -1
			\end{cases}
		\]
		无解,有唯一解或有无穷多解?并在有无穷多解时写出方程组的通解.
	\end{titwo}

	\begin{titwo}
		已知线性方程组
		\[
			\begin{cases}
				x_{1} + x_{2} + x_{3} + x_{4} + x_{5} = a, \\
				3x_{1} + 2x_{2} + x_{3} + x_{4} - 3x_{5} = 0, \\
				x_{2} + 2x_{3} + 2x_{4} + 6x_{5} = b, \\
				5x_{1} + 4x_{2} + 3x_{3} + 3x_{4} - x_{5} = 2.
			\end{cases}
		\]
		\begin{enumerate}
			\item $a,b$ 为何值时,方程组有解;
			\item 方程组有解时,求方程组的导出组的基础解系;
			\item 方程组有解时,求方程组的全部解.
		\end{enumerate}
	\end{titwo}

	\begin{titwo}
		齐次线性方程组
		\[
			\begin{cases}
				\lambda x_{1} + x_{2} + \lambda^{2} x_{3} = 0, \\
				x_{1} + \lambda x_{2} + x_{3} = 0, \\
				x_{1} + x_{2} + \lambda x_{3} = 0
			\end{cases}
		\]
		的系数矩阵为 $\bm A$,若存在 $3$ 阶矩阵 $\bm B \ne \bm O$,使得 $\bm A \bm B = \bm O$,则 \kuo

		\twoch{$\lambda = -2$ 且 $|\bm B| = 0$}{$\lambda = -2$ 且 $|\bm B| \ne 0$}{$\lambda = 1$ 且 $|\bm B| = 0$}{$\lambda = 1$ 且 $|\bm B| \ne 0$}
	\end{titwo}

	\begin{titwo}
		方程组
		\[
			\begin{cases}
				x_{1} - x_{2} = a_{1}, \\
				x_{2} - x_{3} = a_{2}, \\
				x_{3} - x_{4} = a_{3}, \\
				x_{4} - x_{5} = a_{4}, \\
				x_{5} - x_{1} = a_{5}
			\end{cases}
		\]
		有解的充要条件是 \htwo.
	\end{titwo}

	\begin{titwo}
		设线性方程组
		\[
			\begin{cases}
				x_{1} - x_{2} - x_{3} - x_{4} = \lambda x_{1}, \\
				-x_{1} + x_{2} - x_{3} - x_{4} = \lambda x_{2}, \\
				-x_{1} - x_{2} + x_{3} - x_{4} = \lambda x_{3}, \\
				-x_{1} - x_{2} - x_{3} + x_{4} = \lambda x_{4}.
			\end{cases}
		\]
		则当 $\lambda$ 为何值时,方程组有解,有解时,求出所有的解.
	\end{titwo}

	\begin{titwo}
		已知 $\bm \eta_{1} = [-3,2,0]^{\TT}, \bm \eta_{2} = [-1,0,-2]^{\TT}$ 是线性方程组
		\[
			\begin{cases}
				a x_{1} + b x_{2} + c x_{3} = 2, \\
				x_{1} + 2x_{2} - x_{3} = 1, \\
				2x_{1} + x_{2} + x_{3} = -4
			\end{cases}
		\]
		的两个解向量,试求方程组的通解,并确定参数 $a,b,c$.
	\end{titwo}

	\begin{titwo}
		已知方程组 (\Rmnum{1})
		\[
			\begin{cases}
				x_{1} + x_{4} = 1, \\
				x_{2} - 2x_{4} = 2, \\
				x_{3} + x_{4} = -1
			\end{cases}
		\]
		与方程组 (\Rmnum{2})
		\[
			\begin{cases}
				-2x_{1} + x_{2} + ax_{3} - 5x_{4} = 1, \\
				x_{1} + x_{2} - x_{3} + bx_{4} = 4, \\
				3x_{1} + x_{2} + x_{3} + 2x_{4} = c
			\end{cases}
		\]
		是同解方程组,试确定参数 $a,b,c$.
	\end{titwo}

	\begin{titwo}
		设方程组
		\[
			\begin{cases}
				x_{1} + x_{2} - x_{3} + 2x_{4} - x_{5} = 1, \\
				x_{1} - x_{2} + 5x_{4} + ax_{5} = b, \\
				2x_{1} + 4x_{2} - 3x_{3} + x_{4} - 4x_{5} = 1, \\
				x_{1} + 3x_{2} - x_{3} - x_{4} = -1, \\
				2x_{1} - x_{3} + ax_{4} - 4x_{5} = -1.
			\end{cases}
		\]
		问:
		\begin{enumerate}
			\item $a,b$ 为何值时,方程组有唯一解;
			\item $a,b$ 为何值时,方程组无解;
			\item $a,b$ 为何值时,方程组有无穷多解,并求其通解.
		\end{enumerate}
	\end{titwo}

	\begin{titwo}
		已知方程组 (\Rmnum{1}) $\begin{cases}
			x_{1} + 3x_{2} - 3x_{4} = 1,\\
			-7x_{2} + 3x_{3} + x_{4} = -3
		\end{cases}$ 及方程组 (\Rmnum{2}) 的通解为
		\[
			k_{1} [-1,1,1,0]^{\TT} + k_{2} [2,-1,0,1]^{\TT} + [-2,-3,0,0]^{\TT}.
		\]
		求方程组 (\Rmnum{1}), (\Rmnum{2}) 的公共解.
	\end{titwo}

	\begin{titwo}
		设 $\bm A$ 是 $n$ 阶矩阵,对于齐次线性方程组 (\Rmnum{1})~$\bm A^{n} \bm x = \bm 0$ 和 (\Rmnum{2}) $\bm A^{n + 1} \bm x = \bm 0$,现有命题\\
		\circled{1}(\Rmnum{1}) 的解必是 (\Rmnum{2}) 的解;\\
		\circled{2}(\Rmnum{2}) 的解必是 (\Rmnum{1}) 的解;\\
		\circled{3}(\Rmnum{1}) 的解不一定是 (\Rmnum{2}) 的解;\\
		\circled{4}(\Rmnum{2}) 的解不一定是 (\Rmnum{1}) 的解.\\
		其中正确的是 \kuo.

		\fourch{\circled{1}\circled{4}}{\circled{1}\circled{2}}{\circled{2}\circled{3}}{\circled{3}\circled{4}}
	\end{titwo}

	\begin{titwo}
		设 $\bm A$ 是 $n$ 阶实矩阵,则对线性方程组 (\Rmnum{1})~$\bm A \bm X = \bm 0$ 和 (\Rmnum{2})~$\bm A^{\TT} \bm A \bm X = \bm 0$,必有 \kuo.
		
		\onech{(\Rmnum{2}) 的解是 (\Rmnum{1}) 的解,(\Rmnum{1}) 的解也是 (\Rmnum{2}) 的解}%
		{(\Rmnum{2}) 的解是 (\Rmnum{1}) 的解,但 (\Rmnum{1}) 的解不是 (\Rmnum{2}) 的解}%
		{(\Rmnum{1}) 的解不是 (\Rmnum{2}) 的解,(\Rmnum{2}) 的解也不是 (\Rmnum{1}) 的解}%
		{(\Rmnum{1}) 的解是 (\Rmnum{2}) 的解,但 (\Rmnum{2}) 的解不是 (\Rmnum{1}) 的解}
	\end{titwo}

	\begin{titwo}
		设 $\bm A$ 是 $m \times s$ 矩阵,$\bm B$ 是 $s \times n$ 矩阵,则齐次线性方程组 $\bm B \bm X = \bm 0$ 和 $\bm A \bm B \bm X = \bm 0$ 是同解方程组的一个充分条件是 \kuo.

		\twoch{$r(\bm A) = m$}{$r(\bm A) = s$}{$r(\bm B) = s$}{$r(\bm B) = n$}
	\end{titwo}

	\begin{titwo}
		设四元齐次线性方程组 (\Rmnum{1}) 为 $\begin{cases}
			x_{1} + x_{2} = 0, \\
			x_{2} - x_{4} = 0,
		\end{cases}$ 又已知某齐次线性方程组 (\Rmnum{2}) 的通解为
		\[
			k_{1} [0,1,1,0]^{\TT} + k_{2} [-1,2,2,1]^{\TT}.
		\]
		\begin{enumerate}
			\item 求线性方程组 (\Rmnum{1}) 的基础解系;
			\item 问线性方程组 (\Rmnum{1}) 和 (\Rmnum{2}) 是否有非零公共解?若有,则求出所有的非零公共解. 若没有,则说明理由.
		\end{enumerate}
	\end{titwo}

	\begin{titwo}
		已知齐次线性方程组 (\Rmnum{1}) 的基础解系为 $\bm \xi_{1} = [1,0,1,1]^{\TT}$, $\bm \xi_{2} = [2,1,0,-1]^{\TT}$, $\bm \xi_{3} = [0,2,1,-1]^{\TT}$,添加两个方程
		\[
			\begin{cases}
				x_{1} + x_{2} + x_{3} + x_{4} = 0, \\
				x_{1} + 2x_{2} + 2x_{4} = 0
			\end{cases}
		\]
		后组成齐次线性方程组 (\Rmnum{2}),求 (\Rmnum{2}) 的基础解系.
	\end{titwo}

	\begin{titwo}
		已知线性方程组 (\Rmnum{1}) $\begin{cases}
			3x_{1} - x_{2} + 8x_{3} + x_{4} = 0, \\
			x_{1} + 3x_{2} - 9x_{3} + 7x_{4} = 0
		\end{cases}$ 及线性方程组 (\Rmnum{2}) 的基础解系
		\[
			\bm \xi_{1} = [-3,7,2,0]^{\TT}, \bm \xi_{2} = [-1,-2,0,1]^{\TT}.
		\]
		求方程组 (\Rmnum{1}) 和 (\Rmnum{2}) 的公共解.
	\end{titwo}

	\begin{titwo}
        已知齐次线性方程组 (\Rmnum{1}) 为
        \[
            \begin{cases}
                x_{1} + x_{2} - x_{3} = 0, \\
                x_{2} + x_{3} - x_{4} = 0,
            \end{cases}
        \]
        齐次线性方程组 (\Rmnum{2}) 的基础解系为
		\[
			\bm \xi_{1} = [-1,1,2,4]^{\TT}, \bm \xi_{2} = [1,0,1,1]^{\TT}.
		\]
		\begin{enumerate}
			\item 求方程组 (\Rmnum{1}) 的基础解系;
			\item 求方程组 (\Rmnum{1}) 与 (\Rmnum{2}) 的全部非零公共解,并将非零公共解分别由方程组 (\Rmnum{1}), (\Rmnum{2}) 的基础解系线性表示.
		\end{enumerate}
	\end{titwo}

	\begin{titwo}
		设 $\bm A$ 是 $4$ 阶方阵,则下列线性方程组是同解方程组的是 \kuo.

		\twoch{$\bm A \bm x = \bm 0$; $\bm A^{2} \bm x = \bm 0$}{$\bm A^{2} \bm x = \bm 0$; $\bm A^{3} \bm x = \bm 0$}{$\bm A^{3} \bm x = \bm 0$; $\bm A^{4} \bm x = \bm 0$}{$\bm A^{4} \bm x = \bm 0$; $\bm A^{5} \bm x = \bm 0$}
	\end{titwo}

	\begin{titwo}
		设线性方程组
		\begin{equation}\label{eq:156.1}
			\begin{cases}
				x_{1} + 3x_{3} + 5x_{4} = 0, \\
				x_{1} - x_{2} - 2x_{3} + 2x_{4} = 0, \\
				2x_{1} - x_{2} + x_{3} + 3x_{4} = 0,
			\end{cases}
		\end{equation}
		添加一个方程 $a x_{1} + 2x_{2} + b x_{3} - 5x_{4} = 0$ 后,成为方程组
		\begin{equation}\label{eq:156.2}
			\begin{cases}
				x_{1} + 3x_{3} + 5x_{4} = 0, \\
				x_{1} - x_{2} - 2x_{3} + 2x_{4} = 0, \\
				2x_{1} - x_{2} + x_{3} + 3x_{4} = 0, \\
				ax_{1} + 2x_{2} + bx_{3} - 5x_{4} = 0.
			\end{cases}
		\end{equation}
		\begin{enumerate}
			\item 求方程组~\eqref{eq:156.1} 的通解;
			\item $a,b$ 满足什么条件时,\eqref{eq:156.1}~\eqref{eq:156.2} 是同解方程组.
		\end{enumerate}
	\end{titwo}
\section{特征值与特征向量}
	\begin{titwo}
		已知 $\bm \alpha_{1} = [-1,1,a,4]^{\TT}$, $\bm \alpha_{2} = [-2,1,5,a]^{\TT}$, $\bm \alpha_{3} = [a,2,10,1]^{\TT}$ 是 $4$ 阶方阵 $\bm A$ 的 $3$ 个不同特征值对应的特征向量,则 $a$ 的取值范围为 \kuo.

		\twoch{$a \ne 5$}{$a \ne -4$}{$a \ne -3$}{$a \ne -3$ 且 $a \ne -4$}
	\end{titwo}

	\begin{titwo}
		设 $\bm A$, $\bm B$ 为 $n$ 阶矩阵,且 $\bm A$ 与 $\bm B$ 相似,$\bm E$为 $n$ 阶单位矩阵,则 \kuo.

		\onech{$\lambda \bm E - \bm A = \lambda \bm E - \bm B$}{$\bm A$ 与 $\bm B$ 有相同的特征值和特征向量}{$\bm A$ 与 $\bm B$ 都相似于一个对角矩阵}{对任意常数 $t$, $t \bm E - \bm A$ 与 $t \bm E - \bm B$ 相似}
	\end{titwo}

	\begin{titwo}
		已知 $3$ 阶矩阵 $\bm A$ 有特征值 $\lambda_{1} = 1$, $\lambda_{2} = 2$, $\lambda_{3} = 3$,则 $2 \bm A^{\astt}$ 的特征值是 \kuo.

		\fourch{$1,2,3$}{$4,6,12$}{$2,4,6$}{$8,16,24$}
	\end{titwo}

	\begin{titwo}
		已知 $\bm \xi_{1}$, $\bm \xi_{2}$ 是方程 $(\lambda E - \bm A) \bm X = \bm 0$ 的两个不同的解向量,则下列向量中必是 $\bm A$ 的对应于特征值 $\lambda$ 的特征向量的是 \kuo.

		\fourch{$\bm \xi_{1}$}{$\bm \xi_{2}$}{$\bm \xi_{1} - \bm \xi_{2}$}{$\bm \xi_{1} + \bm \xi_{2}$}
	\end{titwo}

	\begin{titwo}
		设
		\[
			\bm A = \begin{bsmallmatrix}
				-1 & 2 & 3 \\
				2 & -1 & 0 \\
				3 & 3 & 1
			\end{bsmallmatrix},
		\]
		则下列选项中是 $\bm A$ 的特征向量的是 \kuo.

		\twoch{$\bm \xi_{1} = [1,2,1]^{\TT}$}{$\bm \xi_{2} = [1,-2,1]^{\TT}$}{$\bm \xi_{3} = [2,1,2]^{\TT}$}{$\bm \xi_{4} = [2,1,-2]^{\TT}$}
	\end{titwo}

	\begin{titwo}
		已知 $\bm A$, $\bm B$ 为 $3$ 阶相似矩阵,$\lambda_{1} = 1$, $\lambda_{2} = 2$ 为 $\bm A$ 的两个特征值,$|\bm B| = 2$,则行列式 $\begin{vsmallmatrix}
			(\bm A + \bm E)^{-1} & \bm O \\
			\bm O & (2 \bm B)^{\astt}
		\end{vsmallmatrix} = $ \htwo.
	\end{titwo}

	\begin{titwo}
		已知 $-2$ 是 $\bm A = \begin{bsmallmatrix}
			0 & -2 & -2 \\
			2 & x & -2 \\
			-2 & 2 & b
		\end{bsmallmatrix}$ 的特征值,其中 $b$ ($b \ne 0$) 是任意常数,则 $x = $ \htwo.
	\end{titwo}

	\begin{titwo}
		设 $\bm A$ 是 $3$ 阶矩阵,$|\bm A| = 3$,且满足 $|\bm A^{2} + 2 \bm A| = 0$,$|2 \bm A^{2} + \bm A| = 0$,则 $\bm A^{\astt}$ 的特征值是 \htwo.
	\end{titwo}

	\begin{titwo}
		设 $\bm A$ 是 $n$ 阶实对称矩阵,$\lambda_{1}$, $\lambda_{2}$, $\cdots$, $\lambda_{n}$ 是 $\bm A$ 的 $n$ 个互不相同的特征值,$\bm \xi_{1}$ 是 $\bm A$ 的对应于 $\lambda_{1}$ 的一个单位特征向量,则矩阵 $\bm B = \bm A - \lambda_{1} \bm \xi_{1} \bm \xi_{1}^{\TT}$ 的特征值是 \htwo.
	\end{titwo}

	\begin{titwo}
		设 $\bm A$ 是 $3$ 阶矩阵,$\bm \xi_{1}$, $\bm \xi_{2}$, $\bm \xi_{3}$ 是三个线性无关的 $3$ 维列向量,满足 $\bm A \bm \xi_{i} = \bm \xi_{i}$, $i = 1,2,3$,则 $\bm A = $ \htwo.
	\end{titwo}

	\begin{titwo}
		设 $\bm A$ 为 $n$ 阶矩阵,$\lambda_{1}$ 和 $\lambda_{2}$ 是 $\bm A$ 的两个不同的特征值,$\bm x_{1}$, $\bm x_{2}$ 是分别属于 $\lambda_{1}$ 和 $\lambda_{2}$ 的特征向量. 证明:$\bm x_{1} + \bm x_{2}$ 不是 $\bm A$ 的特征向量.
	\end{titwo}

	\begin{titwo}
		已知 $\bm \alpha = [1,k,1]^{\TT}$ 是 $\bm A^{-1}$ 的特征向量,其中 $\bm A = \begin{bsmallmatrix}
			2 & 1 & 1 \\
			1 & 2 & 1 \\
			1 & 1 & 2
		\end{bsmallmatrix}$,求 $k$ 及 $\bm \alpha$ 所对应的 $\bm A$ 的特征值.
	\end{titwo}

	\begin{titwo}
		设 $\bm A$ 是 $n$ 阶方阵,$2,4,\cdots,2n$ 是 $\bm A$ 的 $n$ 个特征值,$\bm E$ 是 $n$ 阶单位阵,计算行列式 $|\bm A - 3 \bm E|$ 的值.
	\end{titwo}

	\begin{titwo}
		设 $\lambda_{1}, \lambda_{2}$ 是 $n$ 阶矩阵 $\bm A$ 的特征值,$\bm \alpha_{1}, \bm \alpha_{2}$ 分别是 $\bm A$ 的对应于 $\lambda_{1}, \lambda_{2}$ 的特征向量,则 \kuo.

		\onech{当 $\lambda_{1} = \lambda_{2}$ 时,$\bm \alpha_{1}, \bm \alpha_{2}$ 对应分量必成比例}{当 $\lambda_{1} = \lambda_{2}$ 时,$\bm \alpha_{1}, \bm \alpha_{2}$ 对应分量不成比例}{当 $\lambda_{1} \ne \lambda_{2}$ 时,$\bm \alpha_{1}, \bm \alpha_{2}$ 对应分量必成比例}{当 $\lambda_{1} \ne \lambda_{2}$ 时,$\bm \alpha_{1}, \bm \alpha_{2}$ 对应分量必不成比例}
	\end{titwo}

	\begin{titwo}
		设 $\bm A$ 为 $n$ 阶矩阵,则下列命题正确的是 \kuo.

		\onech{若 $\bm \alpha$ 为 $\bm A^{\TT}$ 的特征向量,那么 $\bm \alpha$ 为 $\bm A$ 的特征向量}{若 $\bm \alpha$ 为 $\bm A^{\astt}$ 的特征向量,那么 $\bm \alpha$ 为 $\bm A$ 的特征向量}{若 $\bm \alpha$ 为 $\bm A^{2}$ 的特征向量,那么 $\bm \alpha$ 为 $\bm A$ 的特征向量}{若 $\bm \alpha$ 为 $2 \bm A$ 的特征向量,那么 $\bm \alpha$ 为 $\bm A$ 的特征向量}
	\end{titwo}

	\begin{titwo}
		已知 $\bm A$ 是 $3$ 阶矩阵,$r(\bm A) = 1$,则 $\lambda = 0$ \kuo.

		\onech{必是 $\bm A$ 的二重特征值}{至少是 $\bm A$ 的二重特征值}{至多是 $\bm A$ 的二重特征值}{一重、二重、三重特征值都可能}
	\end{titwo}

	\begin{titwo}
		$\bm A$ 是 $n$ 阶矩阵,则 $\bm A$ 相似于对角矩阵的充分必要条件是 \kuo.

		\onech{$\bm A$ 有 $n$ 个不同的特征值}{$\bm A$ 有 $n$ 个不同的特征向量}{对 $\bm A$ 的每个 $r_{i}$ 重特征值 $\lambda_{i}$,都有 $r(\lambda_{i} \bm E - \bm A) = n - r_{i}$}{$\bm A$ 是实对称矩阵}
	\end{titwo}

	\begin{titwo}
		已知 $\bm P^{-1} \bm A \bm P = \begin{bsmallmatrix}
			2 & 0 & 0 \\
			0 & 6 & 0 \\
			0 & 0 & 6
		\end{bsmallmatrix}$,$\bm \alpha_{1}$ 是矩阵 $\bm A$ 属于特征值 $\lambda = 2$ 的特征向量,$\bm \alpha_{2}, \bm \alpha_{3}$ 是矩阵 $\bm A$ 属于特征值 $\lambda = 6$ 的线性无关的特征向量,那么矩阵 $\bm P$ 不能是 \kuo.

		\twoch{$[ \bm \alpha_{1}, - \bm \alpha_{2}, \bm \alpha_{3} ]$}{$[ \bm \alpha_{1}, \bm \alpha_{2} + \bm \alpha_{3}, \bm \alpha_{2} - 2 \bm \alpha_{3} ]$}{$[ \bm \alpha_{1}, \bm \alpha_{3}, \bm \alpha_{2} ]$}{$[\bm \alpha_{1} + \bm \alpha_{2}, \bm \alpha_{1} - \bm \alpha_{2}, \bm \alpha_{3}]$}
	\end{titwo}

	\begin{titwo}
		设 $\bm A, \bm B$ 为 $3$ 阶相似矩阵,且 $|2 \bm E + \bm A| = 0$,$\lambda_{1} = 1,$ $\lambda_{2} = -1$ 为 $\bm B$ 的两个特征值,则行列式 $|\bm A + 2 \bm A \bm B| = $ \htwo.
	\end{titwo}

	\begin{titwo}
		设 $\bm A = \bm E + \bm \alpha \bm \beta^{\TT}$,其中 $\bm \alpha, \bm \beta$ 均为 $n$ 维列向量,$\bm \alpha^{\TT}  \bm \beta = 3$,则 $|\bm A + 2 \bm E| = $ \htwo.
	\end{titwo}

	\begin{titwo}
		矩阵 $\bm A = \begin{bsmallmatrix}
			1 & 1 & 1 & 1 \\
			1 & 1 & 1 & 1 \\
			1 & 1 & 1 & 1 \\
			1 & 1 & 1 & 1
		\end{bsmallmatrix}$ 的非零特征值是 \htwo.
	\end{titwo}

	\begin{titwo}
		设 $\bm A$ 是 $3$ 阶矩阵,已知 $|\bm A + \bm E| = 0$,$|\bm A + 2 \bm E| = 0$,$|\bm A + 3 \bm E| = 0$,则 $|\bm A + 4 \bm E| = $ \htwo.
	\end{titwo}

	\begin{titwo}
		设 $n$ 阶矩阵 $\bm A$ 的元素全是 $1$,则 $\bm A$ 的 $n$ 个特征值是 \htwo.
	\end{titwo}

	\begin{titwo}
		已知 $\bm \alpha = [\alpha,1,1]^{\TT}$ 是矩阵 $\bm A = \begin{bsmallmatrix}
			-1 & 2 & 2 \\
			2 & a & -2 \\
			2 & -2 & -1
		\end{bsmallmatrix}$ 的逆矩阵的特征向量,那么 $a = $ \htwo.
	\end{titwo}

	\begin{titwo}
		\begin{enumerate}
			\item 设 $\lambda_{1},\lambda_{2},\cdots,\lambda_{n}$ 是 $n$ 阶矩阵 $\bm A$ 的互异特征值,$\bm \alpha_{1},\bm \alpha_{2},\cdots,\bm \alpha_{n}$ 是 $\bm A$ 的分别对应于这些特征值的特征向量,证明 $\bm \alpha_{1},\bm \alpha_{2},\cdots,\bm \alpha_{n}$ 线性无关;
			\item 设 $\bm A, \bm B$ 为 $n$ 阶方阵,$|\bm B| \ne 0$,若方程 $|\bm A - \lambda \bm B| = 0$ 的全部根 $\lambda_{1},\lambda_{2},\cdots,\lambda_{n}$ 互异,$\bm \alpha_{i}$ 分别是方程组 $(\bm A - \lambda_{i} \bm B) \bm x = \bm 0$ 的非零解,$i = 1,2,\cdots,n$. 证明 $\bm \alpha_{1},$ $\bm \alpha_{2},\cdots,\bm \alpha_{n}$ 线性无关.
		\end{enumerate}
	\end{titwo}

	\begin{titwo}
		设 $\bm A$ 是 $n \times n$ 矩阵,对任何 $n$ 维列向量 $\bm X$ 都有 $\bm A \bm X = \bm 0$,证明:$\bm A = \bm O$.
	\end{titwo}

	\begin{titwo}
		设有 $4$ 阶方阵 $\bm A$ 满足条件 $|3 \bm E + \bm A| = 0$,$\bm A \bm A^{\TT} = 2 \bm E$,$|\bm A| < 0$,其中 $\bm E$ 是 $4$ 阶单位矩阵. 求方阵 $\bm A$ 的伴随矩阵 $\bm A^{\astt}$ 的一个特征值.
	\end{titwo}

	\begin{titwo}
		已知 $\bm B$ 是 $n$ 阶矩阵,满足 $\bm B^{2} = \bm E$ (此时矩阵 $\bm B$ 称为对合矩阵). 求 $\bm B$ 的特征值的取值范围.
	\end{titwo}

	\begin{titwo}
		设 $\bm A, \bm B$ 是 $n$ 阶方阵,证明:$\bm A \bm B, \bm B \bm A$ 有相同的特征值.
	\end{titwo}

	\begin{titwo}
		已知 $n$ 阶矩阵 $\bm A$ 的每行元素之和为 $a$,求 $\bm A$ 的一个特征值. 当 $k$ 是自然数时,求 $\bm A^{k}$ 的每行元素之和.
	\end{titwo}

	\begin{titwo}
		设 $\bm A$ 是 $3$ 阶矩阵,$\lambda_{1},\lambda_{2},\lambda_{3}$ 是三个不同的特征值,$\bm \xi_{1},\bm \xi_{2},\bm \xi_{3}$ 是相应的特征向量. 证明:向量组 $\bm A(\bm \xi_{1} + \bm \xi_{2})$, $\bm A(\bm \xi_{2} + \bm \xi_{3})$, $\bm A(\bm \xi_{3} + \bm \xi_{1})$ 线性无关的充要条件是 $\bm A$ 是可逆矩阵.
	\end{titwo}

	\begin{titwo}
		设 $\bm A$ 是 $3$ 阶实矩阵,$\lambda_{1},\lambda_{2},\lambda_{3}$ 是 $\bm A$ 的三个不同的特征值,$\bm \xi_{1},\bm \xi_{2},\bm \xi_{3}$ 是三个对应的特征向量. 证明:当 $\lambda_{2} \lambda_{3} \ne 0$ 时,向量组 $\bm \xi_{1},\bm A (\bm \xi_{1} + \bm \xi_{2}),\bm A^{2} (\bm \xi_{1} + \bm \xi_{2} + \bm \xi_{3})$ 线性无关.
	\end{titwo}

	\begin{titwo}
		设 $\bm A$ 是 $n$ 阶实矩阵,有 $\bm A \bm \xi = \lambda \bm \xi$, $\bm A^{\TT} \bm \eta = \mu \bm \eta$,其中 $\lambda$, $\mu$ 是实数,且 $\lambda \ne \mu$,$\bm \xi$, $\bm \eta$ 是 $n$ 维非零向量. 证明:$\bm \xi$, $\bm \eta$ 正交.
	\end{titwo}

	\begin{titwo}
		设矩阵 $\bm A = \begin{bsmallmatrix}
			a & -1 & c \\
			5 & b & 3 \\
			1 - c & 0 & -a
		\end{bsmallmatrix}$,且 $|\bm A| = -1$,$\bm A$ 的伴随矩阵 $\bm A^{\astt}$ 有特征值 $\lambda_{0}$,属于 $\lambda_{0}$ 的特征向量维 $\bm \alpha = [-1,-1,1]^{\TT}$,求 $a$, $b$, $c$ 及 $\lambda_{0}$ 的值.
	\end{titwo}

	\begin{titwo}
		设 $\bm A$ 是 $3$ 阶实对称矩阵,$\lambda_{1} = -1$, $\lambda_{2} = \lambda_{3} = 1$ 是 $\bm A$ 的特征值,对应于 $\lambda_{1}$ 的特征向量为 $\bm \xi_{1} = [0,1,1]^{\TT}$,求 $\bm A$.
	\end{titwo}

	\begin{titwo}
		设 $\bm A$ 为 $3$ 阶矩阵,$\lambda_{1}$, $\lambda_{2}$, $\lambda_{3}$ 是 $\bm A$ 的三个不同特征值,对应的特征向量为 $\bm \alpha_{1}$, $\bm \alpha_{2}$, $\bm \alpha_{3}$,令 $\bm \beta = \bm \alpha_{1} + \bm \alpha_{2} + \bm \alpha_{3}$.
		\begin{enumerate}
			\item 证明 $\bm \beta$, $\bm A \bm \beta$, $\bm A^{2} \bm \beta$ 线性无关;
			\item 若 $\bm A^{3} \bm \beta = \bm A \bm \beta$,求秩 $r(\bm A - \bm E)$ 及行列式 $|\bm A + 2 \bm E|$.
		\end{enumerate}
	\end{titwo}

	\begin{titwo}
		设 $\bm A$ 是 $3$ 阶矩阵,$\lambda_{1} = 1$, $\lambda_{2} = 2$, $\lambda_{3} = 3$ 是 $\bm A$ 的特征值,对应的特征向量分别是
		\[
			\bm \xi_{1} = [2,2,-1]^{\TT}, \bm \xi_{2} = [-1,2,2]^{\TT}, \bm \xi_{3} = [2,-1,2]^{\TT}.
		\]
		又 $\bm \beta = [1,2,3]^{\TT}$. 计算:
		\begin{enumerate}
			\item $\bm A^{n} \bm \xi_{1}$;
			\item $\bm A^{n} \bm \beta$.
		\end{enumerate}
	\end{titwo}

	\begin{titwo}
		设 $\bm A$ 是 $2$ 阶实对称矩阵,有特征值 $\lambda_{1} = 4$, $\lambda_{2} = -1$, $\bm \xi_{1} = [-2,1]^{\TT}$ 是 $\bm A$ 对应于 $\lambda_{1}$ 的特征向量,$\bm \beta = [3,1]^{\TT}$,则 $\bm A \bm \beta = $ \htwo.
	\end{titwo}

	\begin{titwo}
		设 $\bm A$ 是 $3$ 阶实对称矩阵,已知 $\bm A$ 的每行元素之和为 $3$,且有二重特征值 $\lambda_{1} = \lambda_{2} = 1$. 求 $\bm A$ 的全部特征值、特征向量,并求 $\bm A^{n}$.
	\end{titwo}

	\begin{titwo}
		设 $\bm A$, $\bm B$ 均为 $n$ 阶矩阵,$\bm A$ 可逆且 $\bm A \sim \bm B$,则下列命题中:\circled{1} $\bm A \bm B \sim \bm B \bm A$; \circled{2} $\bm A^{2} \sim \bm B^{2}$; \circled{3} $\bm A^{\TT} \sim \bm B^{\TT}$; \circled{4} $\bm A^{-1} \sim \bm B^{-1}$. 正确命题的个数为 \kuo.

		\fourch{1}{2}{3}{4}
	\end{titwo}

	\begin{titwo}
		设矩阵 $\bm A = \begin{bsmallmatrix}
			3 & 2 & -2 \\
			-k & -1 & k \\
			4 & 2 & -3
		\end{bsmallmatrix}$,问 $k$ 为何值时,存在可逆矩阵 $\bm P$,使得 $\bm P^{-1} \bm A \bm P = \bm \varLambda$,求出 $\bm P$ 及相应的对角矩阵.
	\end{titwo}

	\begin{titwo}
		设矩阵 $\bm A = \begin{bsmallmatrix}
			1 & -1 & 1 \\
			x & 4 & y \\
			-3 & -3 & 5
		\end{bsmallmatrix}$ 有三个线性无关的特征向量,$\lambda = 2$ 是 $\bm A$ 的二重特征值,试求可逆矩阵 $\bm P$ 使得 $\bm P^{-1} \bm A \bm P = \bm \varLambda$,其中 $\bm \varLambda$ 是对角矩阵.
	\end{titwo}

	\begin{titwo}
		已知 $\bm \xi = [1,1,-1]^{\TT}$ 是矩阵 $\bm A = \begin{bsmallmatrix}
			2 & -1 & 2 \\
			5 & a & 3 \\
			-1 & b & -2
		\end{bsmallmatrix}$ 的一个特征向量.
		\begin{enumerate}
			\item 确定参数 $a$, $b$ 及 $\bm \xi$ 对应的特征值 $\lambda$;
			\item $\bm A$ 是否相似于对角矩阵,说明理由.
		\end{enumerate}
	\end{titwo}
\section{相似}
	\begin{titwo}
		下列矩阵中能相似于对角矩阵的是 \kuo.

		\twoch{$\bm A = \begin{bsmallmatrix}
			1 & 2 & 0 \\
			0 & 1 & 0 \\
			0 & 0 & 2
		\end{bsmallmatrix}$}{$\bm B = \begin{bsmallmatrix}
			1 & 0 & 2 \\
			0 & 2 & 0 \\
			0 & 0 & 1
		\end{bsmallmatrix}$}{$\bm C = \begin{bsmallmatrix}
			1 & 2 & 0 \\
			0 & 2 & 0 \\
			0 & 0 & 1
		\end{bsmallmatrix}$}{$\bm D = \begin{bsmallmatrix}
			1 & 1 & 1 \\
			0 & 1 & 0 \\
			0 & 0 & 2
		\end{bsmallmatrix}$}
	\end{titwo}

	\begin{titwo}
		下列矩阵中不能相似于对角矩阵的是 \kuo.

		\twoch{$\bm A = \begin{bsmallmatrix}
			1 & 1 & 0 \\
			0 & 1 & 0 \\
			0 & 0 & 2
		\end{bsmallmatrix}$}{$\bm B = \begin{bsmallmatrix}
			1 & 0 & 0 \\
			0 & 1 & 1 \\
			0 & 0 & 2
		\end{bsmallmatrix}$}{$\bm C = \begin{bsmallmatrix}
			1 & 1 & 0 \\
			0 & 2 & 1 \\
			0 & 0 & 3
		\end{bsmallmatrix}$}{$\bm D = \begin{bsmallmatrix}
			1 & 0 & 0 \\
			0 & 1 & 1 \\
			0 & 1 & 2
		\end{bsmallmatrix}$}
	\end{titwo}

	\begin{titwo}
		设 $\bm A = \begin{bsmallmatrix}
			3 & 0 & 0 \\
			0 & 1 & 2 \\
			2 & 0 & 5
		\end{bsmallmatrix}$, $\bm B = \begin{bsmallmatrix}
			2 & 1 & 0 \\
			0 & 2 & 0 \\
			3 & 0 & 5
		\end{bsmallmatrix}$, $\bm C = \begin{bsmallmatrix}
			3 & 0 & 2 \\
			0 & 1 & 0 \\
			2 & 0 & 5
		\end{bsmallmatrix}$, $\bm D = \begin{bsmallmatrix}
			2 & 0 & 0 \\
			0 & 2 & 0 \\
			3 & 1 & 5
		\end{bsmallmatrix}$,其中与对角矩阵相似的有 \kuo.

		\fourch{$\bm A,\bm B,\bm C$}{$\bm B,\bm D$}{$\bm A,\bm C,\bm D$}{$\bm A,\bm C$}
	\end{titwo}

	\begin{titwo}
		设 $\bm A$ 是 $n$ 阶矩阵,满足 $\bm A^{2} = \bm A$,且 $r(\bm A) = r (0 < r \leq n)$. 证明:
		\[
			\bm A \sim \begin{bsmallmatrix}
				\bm E_{r} & \bm O \\
				\bm O & \bm O
			\end{bsmallmatrix},
		\]
		其中 $\bm E_{r}$ 是 $r$ 阶单位矩阵.
	\end{titwo}

	\begin{titwo}
		设 $\bm A$, $\bm B$ 均为 $n$ 阶矩阵,$\bm A$ 有 $n$ 个互不相同的特征值,且 $\bm A \bm B = \bm B \bm A$. 证明:$\bm B$ 相似于对角矩阵.
	\end{titwo}

	\begin{titwo}
		设 $\bm A = \bm E + \bm \alpha \bm \beta^{\TT}$,其中 $\bm \alpha = [a_{1},a_{2},\cdots,a_{n}]^{\TT} \ne \bm 0$, $\bm \beta = [b_{1},b_{2},\cdots,b_{n}]^{\TT} \ne \bm 0$,且 $\bm \alpha^{\TT} \bm \beta = 2$.
		\begin{enumerate}
			\item 求 $\bm A$ 的特征值和特征向量;
			\item 求可逆矩阵 $\bm P$,使得 $\bm P^{-1} \bm A \bm P = \bm \varLambda$.
		\end{enumerate}
	\end{titwo}

	\begin{titwo}
		设向量 $\bm \alpha = [a_{1},a_{2},\cdots,a_{n}]^{\TT}$, $\bm \beta = [b_{1},\allowbreak b_{2},\allowbreak \cdots,\allowbreak b_{n}]^{\TT}$ 都是非零向量,且满足条件 $\bm \alpha^{\TT} \bm \beta = 0$,记 $n$ 阶矩阵 $\bm A = \bm \alpha \bm \beta^{\TT}$,求:
		\begin{enumerate}
			\item $\bm A^{2}$;
			\item $\bm A$ 的特征值和特征向量;
			\item $\bm A$ 能否相似于对角矩阵,说明理由.
		\end{enumerate}
	\end{titwo}

	\begin{titwo}
		设 $a_{0},a_{1},\cdots,a_{n-1}$ 是 $n$ 个实数,方阵
		\[
			\bm A = \begin{bsmallmatrix}
				0 & 1 & 0 & \cdots & 0 & 0 \\
				0 & 0 & 1 & \cdots & 0 & 0 \\
				\vdots & \vdots & \vdots & & \vdots & \vdots \\
				0 & 0 & 0 & \cdots & 0 & 1 \\
				-a_{0} & -a_{1} & -a_{2} & \cdots & -a_{n-2} & -a_{n-1}
			\end{bsmallmatrix}
		\]
		\begin{enumerate}
			\item 若 $\lambda$ 是 $\bm A$ 的特征值,证明 $\bm \xi = \bigl[1,\allowbreak \lambda,\allowbreak \lambda^{2},\allowbreak \cdots,\allowbreak \lambda^{n-1}\bigr]^{\TT}$ 是 $\bm A$ 的对应于特征值 $\lambda$ 的特征向量;
			\item 若 $\bm A$ 有 $n$ 个互异的特征值 $\lambda_{1},\lambda_{2},\cdots,\lambda_{n}$,求可逆矩阵 $\bm P$,使 $\bm P^{-1} \bm A \bm P = \bm \varLambda$.
		\end{enumerate}
	\end{titwo}

	\begin{titwo}
		设 $\bm A$ 是 $3$ 阶矩阵,$\bm \alpha_{1}$, $\bm \alpha_{2}$, $\bm \alpha_{3}$ 是 $3$ 维列向量,$\bm \alpha_{1} \ne \bm 0$,满足 $\bm A \bm \alpha_{1} = 2 \bm \alpha_{1}$, $\bm A \bm \alpha_{2} = \bm \alpha_{1} + 2 \bm \alpha_{2}$, $\bm A \bm \alpha_{3} = \bm \alpha_{2} + 2 \bm \alpha_{3}$.
		\begin{enumerate}
			\item 证明 $\bm \alpha_{1}$, $\bm \alpha_{2}$, $\bm \alpha_{3}$ 线性无关;
			\item $\bm A$ 能否相似于对角矩阵,说明理由.
		\end{enumerate}
	\end{titwo}

	\begin{titwo}
		设 $n$ 阶矩阵
		\[
			\bm A = \begin{bsmallmatrix}
				a_{1}b_{1} & a_{1}b_{2} & \cdots & a_{1}b_{n} \\
				a_{2}b_{1} & a_{2}b_{2} & \cdots & a_{2}b_{n} \\
				\vdots & \vdots &  & \vdots \\
				a_{n}b_{1} & a_{n}b_{2} & \cdots & a_{n}b_{n}
			\end{bsmallmatrix}.
		\]
		已知 $\tr(\bm A) = a \ne 0$. 证明:矩阵 $\bm A$ 相似于对角矩阵.
	\end{titwo}

	\begin{titwo}
		已知 $\bm A \sim \begin{bsmallmatrix}
			1 & 0 & 0 \\
			0 & 1 & 0 \\
			0 & 0 & -2
		\end{bsmallmatrix}$,则 $r(\bm A - \bm E) + r(2 \bm E + \bm A) = $ \htwo.
	\end{titwo}

	\begin{titwo}
		已知矩阵 $\bm A = \begin{bsmallmatrix}
			2 & 0 & 0 \\
			0 & 0 & 1 \\
			0 & 1 & x
		\end{bsmallmatrix}$ 与 $\bm B = \begin{bsmallmatrix}
			2 & 0 & 0 \\
			0 & y & 0 \\
			0 & 0 & -1
		\end{bsmallmatrix}$ 相似.
		\begin{enumerate}
			\item 求 $x$ 与 $y$;
			\item 求一个满足 $\bm P^{-1} \bm A \bm P = \bm B$ 的可逆矩阵 $\bm P$.
		\end{enumerate}
	\end{titwo}

	\begin{titwo}
		已知矩阵 $\bm A = \begin{bsmallmatrix}
			1 & 0 & -1 \\
			0 & 1 & 0 \\
			-2 & 1 & 0
		\end{bsmallmatrix}$ 与 $\bm B = \begin{bsmallmatrix}
			2 & 3 & 3 \\
			2 & 1 & 0 \\
			a & b & c
		\end{bsmallmatrix}$ 相似,求 $a$, $b$, $c$ 及可逆矩阵 $\bm P$,使 $\bm P^{-1} \bm A \bm P = \bm B$.
	\end{titwo}

	\begin{titwo}
		设 $\bm A = \begin{bsmallmatrix}
			8 & -2 & -2 \\
			-2 & 5 & -4 \\
			-2 & -4 & 5
		\end{bsmallmatrix}$,求实对称矩阵 $\bm B$,使 $\bm A = \bm B^{2}$.
	\end{titwo}

	\begin{titwo}
		证明:$\bm A \sim \bm B$,其中
		\[
			\bm A = \begin{bsmallmatrix}
				1 & & & & \\
				& 2 & & & \\
				& & \ddots & & \\
				& & & n-1 & \\
				& & & & n
			\end{bsmallmatrix},
			\bm B = \begin{bsmallmatrix}
				n & & & & \\
				& n-1 & & & \\
				& & \ddots & & \\
				& & & 2 & \\
				& & & & 1
			\end{bsmallmatrix}.
		\]
		并求可逆矩阵 $\bm P$,使得 $\bm P^{-1} \bm A \bm P = \bm B$.
	\end{titwo}

	\begin{titwo}
		设 $\bm \alpha = [a_{1},a_{2},\cdots,a_{n}]^{\TT} \ne \bm 0$, $\bm A = \bm \alpha \bm \alpha^{\TT}$,求可逆矩阵 $\bm P$,使 $\bm P^{-1} \bm A \bm P = \bm \varLambda$.
	\end{titwo}

	\begin{titwo}
		设
		\[
			\bm A = \begin{bsmallmatrix}
				2 & -2 & 0 \\
				-2 & 1 & -2 \\
				0 & -2 & 0
			\end{bsmallmatrix},
			\bm B = \begin{bsmallmatrix}
				1 & -2 & -2 \\
				-2 & 2 & 0 \\
				-2 & 0 & 0
			\end{bsmallmatrix},
		\]
		问 $\bm A$, $\bm B$ 是否相似,并说明理由.
	\end{titwo}

	\begin{titwo}
		设矩阵 $\bm A = \begin{bsmallmatrix}
			0 & 1 & 0 & 0 \\
			1 & 0 & 0 & 0 \\
			0 & 0 & k & 1 \\
			0 & 0 & 1 & 2
		\end{bsmallmatrix}$,已知 $\bm A$ 的一个特征值为 $3$.
		\begin{enumerate}
			\item 求 $k$;
			\item 求矩阵 $\bm P$,使 $(\bm A \bm P)^{\TT} (\bm A \bm P)$ 为对角矩阵.
		\end{enumerate}
	\end{titwo}
\section{二次型化标准形、规范形}
	\begin{titwo}
		二次型 $f(x_{1},x_{2},x_{3}) = x_{1}^{2} + 4x_{2}^{2} + 4x_{3}^{2} - 4x_{1}x_{2} + 4\*x_{1}x_{3} - 8x_{2}x_{3}$ 的规范形是 \kuo.

		\twoch{$z_{1}^{2} + z_{2}^{2} + z_{3}^{2}$}{$z_{1}^{2} - z_{2}^{2} - z_{3}^{2}$}{$z_{1}^{2} - z_{2}^{2}$}{$z_{1}^{2}$}
	\end{titwo}

	\begin{titwo}
		若二次型
		\[
			f(x_{1},x_{2},x_{3}) = x_{1}^{2} + ax_{2}^{2} + x_{3}^{2} + 2x_{1}x_{2} - 2x_{2}x_{3} - 2ax_{1}x_{3}
		\]
		的正、负惯性指数都是 $1$. 则 $a = $ \htwo.
	\end{titwo}

	\begin{titwo}
		已知 $f(x_{1},x_{2},x_{3}) = 5x_{1}^{2} + 5x_{2}^{2} + cx_{3}^{2} - 2x_{1}x_{2} + 6x_{1}x_{3} - 6x_{2}x_{3}$ 的秩为 $2$. 试确定参数 $c$ 及二次型对应矩阵的特征值,并问 $f(x_{1},x_{2},x_{3}) = 1$ 表示何种曲面.
	\end{titwo}

	\begin{titwo}
		设 $\bm A \sim \bm \varLambda = \begin{bsmallmatrix}
			1 & 0 & 0 \\
			0 & 2 & 0 \\
			0 & 0 & 3
		\end{bsmallmatrix}$, $f(x) = x^{3} - 6x^{2} + 11x - 5$,则 $f(\bm A) = $ \htwo.
	\end{titwo}

	\begin{titwo}
		设 $\bm A$ 是 $3$ 阶实对称矩阵,$\lambda = 5$ 是 $\bm A$ 的二重特征值. 对应的特征向量为 $\bm \xi_{1} = [1,-1,2]^{\TT}$, $\bm \xi_{2} = [1,2,1]^{\TT}$,则二次型 $f(x_{1},x_{2},x_{3}) = \bm x^{\TT} \bm A \bm x$ 在 $\bm x_{0} = [1,5,$ $0]^{\TT}$ 的值 $f(1,5,0) = $ \htwo.
	\end{titwo}

	\begin{titwo}
		已知二次型
		\[
			f(x_{1},x_{2},x_{3}) = 4x_{2}^{2} - 3x_{3}^{2} + 4x_{1}x_{2} - 4x_{1}x_{3} + 8x_{2}x_{3}.
		\]
		\begin{enumerate}
			\item 写出二次型 $f$ 的矩阵表达式;
			\item 用正交变换把二次型 $f$ 化为标准形,并写出相应的正交矩阵.
		\end{enumerate}
	\end{titwo}

	\begin{titwo}
		已知二次曲面方程
		\[
			x^{2} + ay^{2} + z^{2} + 2bxy + 2xz + 2yz = 4
		\]
		可以经过正交变换
		\[
			\begin{bsmallmatrix}
				x \\
				y \\
				z
			\end{bsmallmatrix} = \bm P
			\begin{bsmallmatrix}
				\xi \\
				\eta \\
				\zeta
			\end{bsmallmatrix}
		\]
		化为椭圆柱面方程 $\eta^{2} + 4 \zeta^{2} = 4$,求 $a$, $b$ 的值和正交矩阵 $\bm P$.
	\end{titwo}

	\begin{titwo}
		已知 $f(x,y) = x^{2} + 4xy + y^{2}$,求正交矩阵 $\begin{bsmallmatrix}
			x \\
			y
		\end{bsmallmatrix} = \bm P \begin{bsmallmatrix}
			u \\
			v
		\end{bsmallmatrix}$ 中的矩阵 $\bm P$,使得
		\[
			f(x,y) = 2u^{2} + 2\sqrt{3} uv.
		\]
	\end{titwo}
\section{合同}
	\begin{titwo}
		实二次型 $f(x_{1},x_{2},\cdots,x_{n})$ 的秩为 $r$,符号差为 $s$,且 $f$ 和 $-f$ 对应的矩阵合同,则必有 \kuo.

		\twoch{$r$ 是偶数,$s = 1$}{$r$ 是奇数,$s = 1$}{$r$ 是偶数,$s = 0$}{$r$ 是奇数,$s = 0$}
	\end{titwo}

	\begin{titwo}
		设方阵 $\bm A_{1}$ 与 $\bm B_{1}$ 合同,$\bm A_{2}$ 与 $\bm B_{2}$ 合同,证明:$\begin{bsmallmatrix}
			\bm A_{1} & \\
			& \bm A_{2}
		\end{bsmallmatrix}$ 与 $\begin{bsmallmatrix}
			\bm B_{1} & \\
			& \bm B_{2}
		\end{bsmallmatrix}$ 合同.
	\end{titwo}

	\begin{titwo}
		设 $\bm A$, $\bm B$ 是 $n$ 阶实对称可逆矩阵,则存在 $n$ 阶可逆阵 $\bm P$,使得下列关系式 \circled{1} $\bm P \bm A = \bm B$; \circled{2} $\bm P^{-1} \bm A \bm B \bm P = \bm B \bm A$; \circled{3} $\bm P^{-1} \bm A \bm P = \bm B$; \circled{4} $\bm P^{\TT} \bm A^{2} \bm P = \bm B^{2}$ 成立的个数是 \kuo.

		\fourch{1}{2}{3}{4}
	\end{titwo}

	\begin{titwo}
		有三组二次型\\
		\circled{1}~$f(x_{1},x_{2},x_{3}) = x_{1}^{2} + 4x_{1}x_{2} + x_{2}^{2} + x_{3}^{2}$, $g(y_{1},y_{2},y_{3}) = y_{1}^{2} + y_{2}^{2} + 2y_{2}y_{3} + y_{3}^{2}$;\\
		\circled{2}~$f(x_{1},x_{2},x_{3}) = \lambda_{1} x_{1}^{2} + \lambda_{2} x_{2}^{2} + \lambda_{3} x_{3}^{2}$, $g(y_{1},y_{2},y_{3}) = \lambda_{3} y_{1}^{2} + \lambda_{1} y_{2}^{2} + \lambda_{2} y_{3}^{2}$;\\
		\circled{3}~$f(x_{1},x_{2},x_{3}) = x_{1}^{2} + x_{2}^{2} + x_{3}^{2}$, $g(y_{1},y_{2},y_{3}) = y_{2}^{2} + 2y_{1}y_{3}$.\\
		二次型矩阵彼此合同的有 \kuo.

		\fourch{0 组}{1 组}{2 组}{3 组}
	\end{titwo}

	\begin{titwo}
		设 $3$ 阶实对称矩阵
		\[
			\bm A = \begin{bsmallmatrix}
				a_{1} + a_{2} + a_{3} & a_{2} + a_{3} & a_{3} \\
				a_{2} + a_{3} & a_{2} + a_{3} & a_{3} \\
				a_{3} & a_{3} & a_{3}
			\end{bsmallmatrix},
			\bm B = \begin{bsmallmatrix}
				k_{3}a_{1} & 0 & 0 \\
				0 & k_{2}a_{2} & 0 \\
				0 & 0 & k_{1}a_{3}
			\end{bsmallmatrix},
		\]
		其中 $k_{1}$, $k_{2}$, $k_{3}$ 为大于 $0$ 的任意常数. 证明 $\bm A$ 与 $\bm B$ 合同,并求出可逆矩阵 $\bm C$,使得 $\bm C^{\TT} \bm A \bm C = \bm B$.
	\end{titwo}
\section{正定}
	\begin{titwo}
		已知二次型 $f(x_{1},x_{2},x_{3}) = 2x_{1}^{2} + x_{2}^{2} + x_{3}^{2} + 2t\*x_{1}\*x_{2} + tx_{2}x_{3}$ 是正定的,则 $t$ 的取值范围是 \htwo.
	\end{titwo}

	\begin{titwo}
		设矩阵 $\bm A = \begin{bsmallmatrix}
			1 & 0 & 1 \\
			0 & 2 & 0 \\
			1 & 0 & 1
		\end{bsmallmatrix}$,矩阵 $\bm B = (k \bm E + \bm A)^{2}$,求对角矩阵 $\bm \varLambda$,使得 $\bm B$ 和 $\bm \varLambda$ 相似,并问 $k$ 为何值时,$\bm B$ 为正定矩阵.
	\end{titwo}

	\begin{titwo}
		设 $\bm A$ 为 $m$ 阶实对称矩阵且正定,$\bm B$ 为 $m \times n$ 实矩阵,$\bm B^{\TT}$ 为 $\bm B$ 的转置矩阵. 证明:$\bm B^{\TT} \bm A \bm B$ 为正定矩阵的充分必要条件是 $r(\bm B) = n$.
	\end{titwo}

	\begin{titwo}
		设 $\bm A$ 与 $\bm B$ 均为正交矩阵,并且 $|\bm A| + |\bm B| = 0$. 证明:$\bm A + \bm B$ 不可逆.
	\end{titwo}

	\begin{titwo}
		下列矩阵中,是正定矩阵的是 \kuo.

		\twoch{$\bm A = \begin{bsmallmatrix}
			1 & -1 & 0 \\
			-1 & 0 & 1 \\
			0 & 1 & 2
		\end{bsmallmatrix}$}{$\bm B = \begin{bsmallmatrix}
			1 & 1 & -1 \\
			1 & 5 & 0 \\
			-1 & 0 & -2
		\end{bsmallmatrix}$}{$\bm C = \begin{bsmallmatrix}
			1 & 0 & 0 \\
			0 & 4 & 2 \\
			0 & 2 & 1
		\end{bsmallmatrix}$}{$\bm D = \begin{bsmallmatrix}
			2 & 1 & 0 \\
			1 & 1 & -1 \\
			0 & -1 & 5
		\end{bsmallmatrix}$}
	\end{titwo}
	% \chapter{概率论与数理统计}
	概率论与数理统计是硕士研究生招在考试考查内容之一,主要考查考生对研究随机规律性的基本概念、基本理论和基本方法的理解,以及运用概率统计方法分析和解决实际问题的能力。在考研数学一试卷中分值为 $34$ 分,约占 \SI{22}{\percent}。

\section{事件与概率}
	\begin{titwo}
		设 $A$ 与 $B$ 是两随机事件,$P(B) = 0.6$ 且 $P(A|B) = 0.5$,则 $P \bigl( A \cup \overline{B} \bigr) = $ \kuo.

		\fourch{$0.1$}{$0.3$}{$0.5$}{$0.7$}
	\end{titwo}

	\begin{titwo}
		设 $10$ 件产品中有 $4$ 件不合格品,从中任取两件,已知所取两件产品中有一件是不合格品,则另一件也是不合格品的概率是 \kuo.

		\fourch{$\frac{1}{2}$}{$\frac{2}{3}$}{$\frac{1}{5}$}{$\frac{2}{5}$}
	\end{titwo}

	\begin{titwo}
		设 $A$, $B$ 是任意两个事件,且 $A \subset B$, $P(B) > 0$,则必有 \kuo.

		\twoch{$P(A) \leq P(A|B)$}{$P(A) < P(A|B)$}{$P(A) \geq P(A|B)$}{$P(A) > P(A|B)$}
	\end{titwo}

	\begin{titwo}
		设 $0 < P(B) < 1$, $P(A_{1}) P(A_{2}) > 0$ 且 $P(A_{1} \cup A_{2} | B) = P(A_{1}|B) + P(A_{2}|B)$,则下列等式成立的是 \kuo.

		\onech{$P \bigl( A_{1} \cup A_{2} | \overline{B} \bigr) = P \bigl( A_{1} | \overline{B} \bigr) + \bigl( A_{2} | \overline{B} \bigr)$}%
		{$P( A_{1}B \cup A_{2}B ) = P(A_{1}B) + P(A_{2}B)$}%
		{$P(A_{1} \cup A_{2}) = P(A_{1}|B) + P(A_{2}|B)$}%
		{$P(B) = P(A_{1}) P(B|A_{1}) + P(A_{2}) P(B|A_{2})$}
	\end{titwo}

	\begin{titwo}
		设事件 $A$, $B$ 满足 $A B = \varnothing$,则下列结论中一定正确的是 \kuo.

		\twoch{$\overline{A}$, $\overline{B}$ 互不相容}{$\overline{A}$, $\overline{B}$ 相容}{$P(AB) = P(A) P(B)$}{$P(A - B) = P(A)$}
	\end{titwo}

	\begin{titwo}
		以下结论,错误的是 \kuo.
		
		\onech{若 $0 < P(B) < 1$, $P(A|B) + P \bigl( \overline{A} | \overline{B} \bigr) = 1$,则 $A$, $B$ 相互独立}%
		{若 $A$, $B$ 满足 $P(B|A) = 1$,则 $P(A-B) = 0$}%
		{设 $A$, $B$ 是两个事件,则 $(A - B) \cup B = A \cup B$}%
		{若当事件 $A$, $B$ 同时发生时,事件 $C$ 必发生,则 $P(C) < P(A) + P(B) - 1$}
	\end{titwo}

	\begin{titwo}
		$A$, $B$, $C$ 为随机事件,$A$ 发生导致 $B$ 与 $C$ 最多有一个发生,则有 \kuo.

		\fourch{$A \subset BC$}{$A \supset BC$}{$A \subset \overline{BC}$}{$A \supset \overline{BC}$}
	\end{titwo}

	\begin{titwo}
		设 $P(B) > 0$, $A_{1}$, $A_{2}$ 互不相容,则下列各式中不一定正确的是 \kuo.

		\onech{$P(A_{1}A_{2} | B) = 0$}{$P(A_{1} \cup A_{2} | B) = P(A_{1} | B) + P(A_{2} | B)$}{$P \bigl( \overline{A_{1}} \overline{A_{2}} | B \bigr) = 1$}{$P \bigl( \overline{A_{1}} \cup \overline{A_{2}} | B \bigr) = 1$}
	\end{titwo}

	\begin{titwo}
		一种零件的加工由相互独立的两道工序组成,第一道工序的废品率为 $p_{1}$,第二道工序的废品 $p_{2}$,则该零件加工的成品率为 \kuo.

		\twoch{$1 - p_{1} - p_{2}$}{$1 - p_{1}p_{2}$}{$1 - p_{1} - p_{2} + p_{1} p_{2}$}{$(1 - p_{1}) + (1 - p_{2})$}
	\end{titwo}

	\begin{titwo}
		以下 $4$ 个结论:\\
		\circled{1} 教室中有 $r$ 个学生,则他们的生日都不相同的概率是 $\frac{\AA_{365}^{r}}{365^{r}}$;\\
		\circled{2} 教室中有 $4$ 个学生,则至少有两个人的生日在同一个月的概率是 $\frac{41}{96}$;\\
		\circled{3} 将 $C,C,E,E,I,N,S$ 共 $7$ 个字母随机地排成一行,恰好排成英文单词 $SCIENCE$ 的概率是 $\frac{1}{315}$;\\
		\circled{4} 袋中有编号为 $1$ 到 $10$ 的 $10$ 个球,今从袋中任取 $3$ 个球,则 $3$ 个球的最小号码为 $5$ 的概率为 $\frac{1}{12}$.\\
		正确的个数为 \kuo.

		\fourch{1}{2}{3}{4}
	\end{titwo}

	\begin{titwo}
		设两个相互独立的事件 $A$ 与 $B$ 至少有一个发生的概率为 $\frac{8}{9}$,且 $A$ 发生 $B$ 不发生的概率与 $B$ 发生 $A$ 不发生的概率相等,则 $P(A) = $ \htwo.
	\end{titwo}

	\begin{titwo}
		设事件 $A$, $B$, $C$ 两两独立,三个事件不能同时发生,且它们的概率相等,则 $P(A \cup B \cup C)$ 的最大值为 \htwo.
	\end{titwo}

	\begin{titwo}
		事件 $A$ 与 $B$ 相互独立,$P(A) = a$, $P(B) = b$,如果事件 $C$ 发生必然导致 $A$ 与 $B$ 同时发生,则 $A$, $B$, $C$ 都不发生的概率为 \htwo.
	\end{titwo}

	\begin{titwo}
		设 $A$, $B$ 是任意两个事件,则 $P \bigl[ \bigl( \overline{A} \cup B \bigr) \bigl( A \cup B \bigr) \bigl( \overline{A} \cup \overline{B} \bigr) \bigl( A \cup \overline{B} \bigr) \bigr] = $ \htwo.
	\end{titwo}

	\begin{titwo}
		设随机事件 $A$, $B$, $C$ 满足 $P(A|B) + P \bigl( \overline{A} | \overline{B} \bigr) = 1$,且 $P\bigl( A \overline{B} \bigr) = P\bigl( \overline{A} B \bigr) = \frac{1}{4}$, $C = A \cup B$,则 $P(AB|C) = $ \htwo.
	\end{titwo}

	\begin{titwo}
		一批产品共有 $10$ 个正品和 $2$ 个次品,任意抽取两次,每次抽一个,抽出后不再放回,则第二次抽出的是次品的概率为 \htwo.
	\end{titwo}

	\begin{titwo}
		设有大小相同、标号分别为 $1$, $2$, $3$, $4$, $5$ 的五个球,同时有标号为 $1$, $2$, $\cdots$, $10$ 的十个空盒. 将五个球随机放入这十个空盒中,设每个球放入任何一个盒子的可能性都是一样的,并且每个空盒可以放多个球,计算下列事件的概率:
		\begin{enumerate}
			\item $A = $ \{某指定的五个盒子中各有一个球\};
			\item $B = $ \{每个盒子中最多只有一个球\};
			\item $C = $ \{某个指定的盒子不空\}.
		\end{enumerate}
	\end{titwo}

	\begin{titwo}
		设 $AB \subset C$. 试证明:$P(A) + P(B) - P(C) \leq 1$.
	\end{titwo}

	\begin{titwo}
		设 $P(A) > 0$, $P(B) > 0$. 证明:$A$, $B$ 互不相容与 $A$, $B$ 相互独立不能同时成立.
	\end{titwo}

	\begin{titwo}
		证明:若三事件 $A$, $B$, $C$ 相互独立,则 $A \cup B$ 及 $A - B$ 都与 $C$ 相互独立.
	\end{titwo}

	\begin{titwo}
		袋中有 $5$ 只白球 $6$ 只黑球,从袋中一次取出 $3$ 个球,发现都是同一颜色,求这颜色是黑色的概率.
	\end{titwo}

	\begin{titwo}
		甲袋中有 $3$ 个白球 $2$ 个黑球,乙袋中有 $4$ 个白球 $4$ 个黑球,今从甲袋中任取 $2$ 球放入乙袋,再从乙袋中任取一球,求该球是白球的概率.
	\end{titwo}

	\begin{titwo}
		某彩票每周开奖一次,每次提供十万分之一的中奖机会,且各周开奖是相互独立的. 某彩民每周买一次彩票,坚持十年(每年 $52$ 周),那么他从未中奖的概率是多少?
	\end{titwo}

	\begin{titwo}
		随机地取两个正数 $x$ 和 $y$,这两个数中的每一个都不超过 $1$,试求 $x$ 与 $y$ 之和不超过 $1$,积不小于 $0.09$ 的概率.
	\end{titwo}

	\begin{titwo}
		设有甲、乙两名射击运动员,甲命中目标的概率是 $0.6$,乙命中目标的概率是 $0.5$,求下列事件的概率:
		\begin{enumerate}
			\item 从甲、乙中任选一人去射击,若目标被命中,则是甲命中的概率;
			\item 甲、乙两人各自独立射击,若目标被命中,则是甲命中的概率.
		\end{enumerate}
	\end{titwo}

	\begin{titwo}
		验收成箱包装的玻璃器皿,每箱 $24$ 只装. 统计资料表明,每箱最多有 $2$ 只残品,且含 $0$, $1$, $2$ 件残品的箱各占 \SI{80}{\percent}, \SI{15}{\percent}, \SI{5}{\percent}. 现在随机抽取一箱,随机检验其中 $4$ 只; 若未发现残品则通过验收,否则要逐一检验并更换. 试求:
		\begin{enumerate}
			\item 一次通过验收的概率;
			\item 通过验收的箱中确实无残品的概率.
		\end{enumerate}
	\end{titwo}

	\begin{titwo}
		甲、乙、丙三人向一架飞机进行射击,他们的命中率分别为 $0.4$, $0.5$, $0.7$. 设飞机中一弹而被击落的概率为 $0.2$,中两弹而被击落的概率为 $0.6$,中三弹必然被击落,今三人各射击一次,求飞机被击落的概率.
	\end{titwo}

	\begin{titwo}
		某学生想借张宇编著的《张宇高等数学 18 讲》,决定到三个图书馆去借,对每一个图书馆而言,有无这本书的概率均为 $0.5$; 若有,能否借到的概率也均为 $0.5$,假设这三个图书馆采购、出借图书相互独立,求该生能借到此书的概率.
	\end{titwo}

	\begin{titwo}
		假设有四张同样的卡片,其中三张上分别只印有 $a_{1}$, $a_{2}$, $a_{3}$,而另一张上同时印有 $a_{1}$, $a_{2}$, $a_{3}$. 现在随意抽取一张卡片,令 $A_{k} = $ \{卡片上印有 $a_{k}$\}. 证明:事件 $A_{1}$, $A_{2}$, $A_{3}$ 两两独立但不相互独立.
	\end{titwo}

	\begin{titwo}
		在电视剧《乡村爱情》中,谢广坤家中生了一对龙凤胎,专业上叫异性双胞胎. 假设双胞胎中第一个是男孩的概率为 \SI{51}{\percent},同性双胞胎是异性双胞胎的 $3$ 倍,已知一双胞胎第一个是男孩,试求第二个也是男孩的概率.
	\end{titwo}
	
	\section{数字特征}
	\begin{titwo}
		现有 10 张奖券,其中 8 张为 2 元的,2 张为 5 元的。今从中任取 3 张,则奖金的数学期望为 \kuo.

		\fourch{$6$}{$7.8$}{$9$}{$11.2$}
	\end{titwo}

	\begin{titwo}
		设 $X_{1}$, $X_{2}$, $X_{3}$ 相互独立,且均服从参数为 $\lambda$ 的泊松分布,令 $Y = \frac{1}{3} (X_{1} + X_{2} + X_{3})$,则 $Y^{2}$ 的数学期望为 \kuo.

		\fourch{$\frac{1}{3} \lambda$}{$\lambda^{2}$}{$\frac{1}{3} \lambda + \lambda^{2}$}{$\frac{1}{3} \lambda^{2} + \lambda$}
	\end{titwo}

	\begin{titwo}
		设 $X$ 为连续型随机变量,方差存在,则对任意常数 $C$ 和 $\varepsilon > 0$,必有 \kuo.

		\onech{$P\{ |X - C| \geq \varepsilon \} = E(|X - C|)/\varepsilon$}%
		{$P\{ |X - C| \geq \varepsilon \} \geq E(|X - C|)/\varepsilon$}%
		{$P\{ |X - C| \geq \varepsilon \} \leq E(|X - C|)/\varepsilon$}%
		{$P\{ |X - C| \geq \varepsilon \} \leq DX/\varepsilon^{2}$}
	\end{titwo}

	\begin{titwo}
		一袋中有 6 个正品 4 个次品,按下列方式抽样:每次取 1 个,取后放回,共取 $n(n \leq 10)$ 次,其中次品个数记为 $X$;若一次性取出 $n(n \leq 10)$ 个,其中次品个数记为 $Y$. 则下列正确的是 \kuo.

		\onech{$EX > EY$}{$EX < EY$}{$EX = EY$}{若 $n$ 不同,则 $EX$, $EY$ 大小不同}
	\end{titwo}

	\begin{titwo}
		设随机变量 $(X,Y)$ 的概率密度 $f(x,y)$ 满足 $f(x,$ $y) = f(-x,y)$,且 $\rho_{XY}$ 存在,则 $\rho_{XY} = $ \kuo.

		\fourch{$1$}{$0$}{$-1$}{$-1$ 或 $1$}
	\end{titwo}

	\begin{titwo}
		设随机变量 $(X,Y)$ 服从二维正态分布,其边缘分布为 $X \sim N(1,1)$, $Y \sim N(2,4)$, $X$ 与 $Y$ 的相关系数为 $\rho_{XY} = -\frac{1}{2}$,且概率 $P\{ aX + bY \leq 1 \} = \frac{1}{2}$,则\kuo.

		\twoch{$a = \frac{1}{2}$, $b = - \frac{1}{4}$}{$a = \frac{1}{4}$, $b = - \frac{1}{2}$}{$a = -\frac{1}{4}$, $b = \frac{1}{2}$}{$a = \frac{1}{2}$, $b = \frac{1}{4}$}
	\end{titwo}

	\begin{titwo}
		设 $a$ 为区间 $(0,1)$ 上一个定点,随机变量 $X$ 服从 $(0,1)$ 上的均匀分布. 以 $Y$ 表示点 $X$ 到 $a$ 的距离,当 $X$ 与 $Y$ 不相关时,$a = $ \kuo.

		\fourch{$0.1$}{$0.3$}{$0.5$}{$0.7$}
	\end{titwo}

	\begin{titwo}
		设 $X$ 是随机变量,$EX > 0$ 且 $E \bigl( X^{2} \bigr) = 0.7$, $DX = 0.2$,则以下各式成立的是 \kuo.

		\twoch{$P\bigl\{ - \frac{1}{2} < X < \frac{3}{2} \bigr\} \geq 0.2$}%
		{$P\bigl\{ X > \sqrt{2} \bigr\} \geq 0.6$}%
		{$P\bigl\{ 0 < X < \sqrt{2} \bigr\} \geq 0.6$}%
		{$P\bigl\{0 < X < \sqrt{2}\bigr\} \leq 0.6$}
	\end{titwo}
	
	\begin{titwo}
		设随机变量 $X_{1}$, $X_{2}$, $\cdots$, $X_{n}$ $(n > 1)$ 独立同分布,其方差 $\sigma^{2} > 0$,记 $\overline{X_{k}} = \frac{1}{k} \* \sum_{i=1}^{k} X_{i}$ $(1 \leq k \leq n)$,则 $\Cov\bigl( \overline{X_{s}},\overline{X_{t}} \bigr)$ $(1 \leq s,t \leq n)$ 的值等于 \kuo.

		\twoch{$\frac{\sigma^{2}}{\max\{ s,t \}}$}{$\frac{\sigma^{2}}{\min\{ s,t \}}$}{$\sigma^{2} \cdot \max\{s,t\}$}{$\sigma^{2} \cdot \min\{s,t\}$}
	\end{titwo}

	\begin{titwo}
		设随机变量 $X$ 的概率密度为
		\[
		f(x) = \begin{cases}
			\frac{3}{8}x^{2}, & 0 < x < 2, \\
			0, & \text{其他},
		\end{cases}
		\]
		则 $E\bigl( \frac{1}{X^{2}} \bigr) = $ \htwo.
	\end{titwo}

	\begin{titwo}
		设随机变量 $Y$ 服从参数为 $1$ 的指数分布,记
		\[
			X_{k} = \begin{cases}
				0, & Y \leq k, \\
				1, & Y > k,
			\end{cases} k = 1,2,
		\]
		则 $E(X_{1} + X_{2}) = $ \htwo.
	\end{titwo}

	\begin{titwo}
		已知离散型随机变量 $X$ 服从参数为 $2$ 的泊松分布,即 $P\{ X = k \} = \frac{ 2^{k}\ee^{-2} }{k!}$, $k = 0$, $1$, $2$, $\cdots$,则随机变量 $Z = 3X - 2$ 的数学期望 $EZ = $ \htwo.
	\end{titwo}

	\begin{titwo}
		设随机变量 $X_{1}$, $X_{2}$, $\cdots$, $X_{100}$ 独立同分布,且 $EX_{i} = 0$, $DX_{i} = 10$, $i = 1$, $2$, $\cdots$, $100$,令 $\overline{X} = \frac{1}{100} \* \sum_{i=1}^{100} X_{i}$,则 $E\Biggl[ \sum_{i=1}^{100} \bigl( X_{i} - \overline{X} \bigr)^{2} \Biggr] = $ \htwo.
	\end{titwo}

	\begin{titwo}
		设随机变量 $X$ 和 $Y$ 均服从 $B\bigl( 1,\frac{1}{2} \bigr)$,且 $D(X + Y) = 1$,则 $X$ 与 $Y$ 的相关系数 $\rho = $ \htwo.
	\end{titwo}

	\begin{titwo}
		已知随机变量 $X \sim N(-3,1)$, $Y \sim N(2,1)$,且 $X$, $Y$ 相互独立,设随机变量 $Z = X - 2Y + 7$,则 $Z \sim $ \htwo.
	\end{titwo}

	\begin{titwo}
		设相互独立的两个随机变量 $X$, $Y$ 具有同一分布律,且 $X$ 的分布律为
		\begin{center}
			\begin{tabular}{c|cc}
				\hline
				$X$ & $0$ & $1$ \\
				\hline
				$P$ & $\frac{1}{2}$ & $\frac{1}{2}$ \\
				\hline
			\end{tabular}
		\end{center}
		则随机变量 $Z = \max\{X,Y\}$ 的分布律为 \htwo.
	\end{titwo}

	\begin{titwo}
		设二维随机变量 $(X,Y)$ 的概率密度为
		\[
			f(x,y) = \begin{cases}
				\frac{1}{8} (x + y), & 0 \leq x \leq 2,0 \leq y \leq 2, \\
				0, & \text{其他}.
			\end{cases}
		\]
		则随机变量 $U = X + 2Y$, $V = -X$ 的协方差 $\Cov(U,$ $V) = $ \htwo.
	\end{titwo}

	\begin{titwo}
		一台设备由三个部件构成,在设备运转中各部件需要调整的概率分别为 $0.10$, $0.20$, $0.30$,设备部件状态相互独立,以 $X$ 表示同时需要调整的部件数,则 $X$ 的方差为 \htwo.
	\end{titwo}

	\begin{titwo}
		设 $(X,Y)$ 的概率密度为
		\[
		f(x,y) = \begin{cases}
			1, & 0 \leq |y| \leq x \leq 1, \\
			0, & \text{其他},
		\end{cases}
		\]
		则 $\Cov(X,Y) = $ \htwo.
	\end{titwo}

	\begin{titwo}
		若 $X_{1}$, $X_{2}$, $X_{3}$ 两两不相关,且 $DX_{i} = 1$ $(i = 1,2,$ $3)$,则 $D(X_{1} + X_{2} + X_{3}) = $ \htwo.
	\end{titwo}

	\begin{titwo}
		设随机变量 $X_{1}$, $X_{2}$, $X_{3}$ 相互独立,且 $X_{1} \sim B \bigl( 4,$ $\frac{1}{2} \bigr)$, $X_{2} \sim B \bigl( 6,\frac{1}{3} \bigr)$, $X_{3} \sim B\bigl( 6,\frac{1}{5} \bigr)$,则 $E[ X_{1} \* (X_{1} + X_{2} - X_{3}) ] = $ \htwo.
	\end{titwo}

	\begin{titwo}
		设随机变量 $X$ 与 $Y$ 的分布律为 \begin{tabular}{c|cc}
			\hline
			$X$ & $0$ & $1$ \\
			\hline
			$P$ & $\frac{1}{4}$ & $\frac{3}{4}$ \\
			\hline
		\end{tabular} 与 \begin{tabular}{c|cc}
			\hline
			$Y$ & $0$ & $1$ \\
			\hline
			$P$ & $\frac{1}{2}$ & $\frac{1}{2}$ \\
			\hline
		\end{tabular} 且相关系数 $\rho_{XY} = \frac{\sqrt{3}}{3}$,则 $(X,Y)$ 的分布律为 \htwo.
	\end{titwo}

	\begin{titwo}
		设二维随机变量 $(X,Y)$ 的分布律为
		\begin{center}
			\begin{tabular}{c|ccc}
				\hline
				\diagbox{$X$}{$Y$} & $-1$ & $0$ & $1$ \\
				\hline
				$-5$ & $0$ & $\frac{1}{9}$ & $\frac{1}{3}$ \\
				$-1$ & $\frac{1}{9}$ & $0$ & $\frac{2}{9}$ \\
				$1$ & $\frac{1}{9}$ & $\frac{1}{9}$ & $0$ \\
				\hline
			\end{tabular}
		\end{center}
		则 $X$ 与 $Y$ 的协方差为 \htwo.
	\end{titwo}

	\begin{titwo}
		设二维随机变量 $(X,Y)$ 的概率密度为
		\[
			f(x,y) = \begin{cases}
				3x, & 0 < x < 1,0 < y < x, \\
				0, & \text{其他},
			\end{cases}
		\]
		则随机变量 $Z = X - Y$ 的方差为 \htwo.
	\end{titwo}
\end{document}